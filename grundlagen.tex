\chapter{Mengentheoretische und topologische Grundlagen}

\section{Kompaktheitsbegriffe für topologische Räumen}

Das Auswahlaxiom oder Abschwächungen davon dienen oftmals als Bindeglied zwischen unterschiedlichen Konzepten in der Mathematik, indem sie ihre Äquivalenz zeigen. 
Ein für die vorliegende Ausarbeitung zentrales Konzept ist das der Kompaktheit topologischer Räume. 
Es exisiteren unterschiedliche Kompaktheitsbegriffe, die im Rahmen von \textbf{ZF} nicht notwendig äquivalent sind.

\begin{defn}
  Sei $(X,\tau)$ ein topologischer Raum.
  Ein Punkt $x \in X$ heißt \textit{Adhärenzpunkt} eines Filters $\varphi \in \mathcal{F}(X)$, falls es einen gegen $x$ bezüglich $\tau$ konvergenten Oberfilter $\psi \supseteq \varphi$ gibt.
\end{defn}

\begin{prop}
  \label{prop:adherence}
  Sei $(X,\tau)$ ein topologischer Raum.
  Ein Punkt $x \in X$ ist genau dann Adhärenzpunkt eines Filters $\varphi \in \mathcal{F}(X)$, falls
  \begin{displaymath}
    x \in \bigcap_{F \in \varphi} \overline{F}.
  \end{displaymath}
\end{prop}

\begin{proof}
  Sei zunächst $x$ Adhärenzpunkt eines Filters $\varphi$ auf $X$.
  Nach Definition existiert ein gegen $x$ konvergenter Oberfilter $\psi \supseteq \varphi$, also gilt $\psi \supseteq \dot x \cap \tau$
  Somit gilt aufgrund der Filtereigenschaft von $\psi$ für alle $F \in \varphi$ und alle $U \in x \cap \tau$, dass  $F \cap U \neq \emptyset$.
  Also besitzt jede Umgebung $U$ von $x$ mit allen $F \in \varphi$ einen nichtleeren Schnitt, was bedeutet, dass $x \in \overline F$ für alle $F  \in \varphi$

  Sei andererseits $x \in \bigcap_{F \in \varphi} \overline F$.
  Nach Definition des Abschlusses einer Menge, gilt somit für alle $U \in \dot x \cap \tau$, dass $F \cap U \neq \emptyset$.
  Es ist also $\varphi \cup (\dot x \cap \tau)$ eine Filtersubbasis.
  Man bezeichne mit $\psi$ den davon erzeugten Filter.
  Nach Konstruktion gelten 
  \begin{displaymath}
    \psi \supseteq \varphi \quad\text{und}\quad \psi \overset{\tau}{\to} x.\qedhere
  \end{displaymath}
\end{proof}

\begin{defn}
  Sei $(X,\tau)$ ein topologischer Raum.
  Dann heißt $x \in X$ \textit{vollständiger Häufungspunkt} einer Teilmenge $A \subseteq X$, wenn für alle Umgebungen $U$ von $x$, die Mengen $A$ und $A \cap U$ dieselbe Kardinalzahl besitzen.
\end{defn}

\begin{defn}
  Ein topologischer Raum $(X,\tau)$ heißt
  \begin{enumerate}[(1)]
    \item kompakt, \textit{falls sich aus jeder offenen Überdeckung von $X$ eine Teilüberdeckung auswählen lässt},
    \item filterkompakt, \textit{falls jeder Filter auf $X$ einen Adhärenzpunkt besitzt},
    \item ultrafilterkompakt, \textit{falls jeder Ultrafilter auf $X$ konvergiert},
    \item \textsc{Alexandroff}-\textsc{Urysohn}-kompakt, \textit{falls jede unendliche Teilmenge von $X$ einen vollständigen Häufungspunkt besitzt},
    \item \textsc{Tychonoff}-kompakt, \textit{falls $X$ homöomorph zu einem abgeschlossenen Teilraum eines Hilbert-Würfels $[0,1]^I$ ist}.
  \end{enumerate}
\end{defn}


\begin{thm}
\label{thm:compactness}
  Sei $(X,\tau)$ ein topologischer Raum. Dann gilt:
  \begin{enumerate}[(1)]
    \item Der Raum $X$ ist genau dann kompakt, wenn er filterkompakt ist. 
    \item Ist $X$ filterkompakt, dann auch ultrafilterkompakt.
  \end{enumerate}
\end{thm}

\begin{proof}
  $(1)$: 
  Sei $(X,\tau)$ kompakt und $\varphi$ ein Filter auf $X$.
  Angenommen, $\varphi$ besitze keinen Adhärenzpunkt.
  Dann gilt nach Proposition \ref{prop:adherence}
  \begin{displaymath}
    \bigcap_{F \in \varphi} \overline{F} = \emptyset.
  \end{displaymath}
  Folglich ist 
  \begin{displaymath}
    X = \bigcup_{F \in \varphi} X \setminus \overline{F} \subseteq \bigcup_{F \in \varphi} X \setminus F 
  \end{displaymath}
  eine offene Überdeckung von $X$.
  Da $X$ als kompakt vorausgesetzt wird, existieren $F_1,\dots,F_n \in \varphi$ mit
  \begin{displaymath}
    \bigcup_{i = 1}^n X \setminus F_i = X
  \end{displaymath}
  woraus wiederum
  \begin{displaymath}
    X \setminus \bigcup_{i = 1}^n F_i = \bigcap_{i = 1}^n F_i = \emptyset
  \end{displaymath}
  folgt. 
  Dies steht jedoch im Widerspruch zur Filtereigenschaft.

  Sei umgekehrt $X$ filterkompakt und $(O_i)_{i \in I}$ eine offene Überdeckung von $X$.
  Angenommen es existiere keine Teilüberdeckung.
  Man betrachte nun die Familie
  \begin{displaymath}
    \mathfrak{B} := \left\{ X \setminus \mathcal{O} \mid \mathcal{O} = \bigcup_{k = 1}^n O_{i_k}\right\}.
  \end{displaymath}
  Diese Familie ist nichtler, da $X \setminus O_i \in \mathcal{O}$ für alle $i \in I$ gilt und zudem abgeschlossen unter endlichen Schnitten.

  Folglich ist $\mathfrak{B}$ eine Filterbasis und es bezeichne $\varphi$ den durch Obermengenbildung davon erzeugten Filter.
  Nach Voraussetzung existiert ein Oberfilter $\psi \supseteq \varphi$ der bezüglich der Topologie $\tau$ gegen ein $x \in X$ konvergiert.
  Aus der Überdeckungseigenschaft folgt, dass zusätzlich ein $i_0 \in I$ existiert mit $x \in O_i$.
  Aufgrund der Konvergenz des Filters $\psi$ folgt $O_i \in \psi$.
  Dies steht jedoch im Widerspruch zu $\psi \supseteq \mathfrak{B}$, denn deswegen gilt bereits $X \setminus O_i \in \psi$.

  $(2)$: Da Ultrafilter bereits bezüglich Inklusion maximal sind, also keine echten Oberfilter besitzen, impliziert Filterkompaktheit bereits Ultrafilterkompaktheit.
\end{proof}

Eine wichtige Folgerung aus dem Auswahlaxiom ist der folgende Satz:
\begin{defn}
    \UFT, der Ultrafiltersatz: 
      \begin{addmargin}[2em]{2em}% 1em left, 2em right
        \textit{Jeder Filter auf einer Menge lässt sich zu einem Ultrafilter erweitern.}
      \end{addmargin}
\end{defn}

Ein Beweis dazu wird in Satz \ref{thm:acpit} für Ideale auf \textsc{Boole}schen Algebren erbracht. 
Zuvor soll jedoch die Beziehung der Gültigkeit von \UFT zur Übereinstimmung gewisser Kompaktheitsbegriffe analysiert werden.

\begin{lem}
  \label{lem:uftprod}
  Sei $(X_i,\tau_i)_{i \in I}$ eine Familie ultrafilterkompakter Räume. 
  Dann ist ihr Produkt $\prod_{i \in I} X_i$ wiederum ultrafilterkompakt.
\end{lem}

\begin{proof}
  Es sei $\varphi \in \mathcal{F}_0(\prod_{i \in I} X_i)$.
  Dann sind auch die Bilder $\pi_i(\varphi)$ Ultrafilter auf $X_i$.
  Aufgrund der Ultrafilterkompaktheit der einzelnen Faktoren konvergiert jeder Bildfilter gegen ein $x_i \in X_i$.
  Dies impliziert jedoch, dass $\varphi$ gegen $(x_i)_{i \in I}$ konvergiert.
\end{proof}

\begin{thm}
  \label{thm:uftcompact}
  Es sind äquivalent:
  \begin{enumerate}[(1)]
    \item Ein topologischer Raum ist genau dann kompakt, wenn er ultrafilterkompakt ist.
    \item \UFT.
  \end{enumerate}
\end{thm}

\begin{proof}
  (1)$\Rightarrow$(2):
  Nach Satz \ref{thm:pitequivalence}((4)$\Rightarrow$(3)), dies ist gerade die Implikation, welche nicht Bezug auf Satz \ref{thm:uftcompact} nimmt, reicht es zu zeigen, dass Produkte kompakter \textsc{Hausdorff}-Räume kompakt sind. 
  Produkte ultrafilterkompakter \textsc{Hausdorff}-Räume sind jedoch nach Lemma \ref{lem:uftprod} wieder ultrafilterkompakt, was nach Voraussetzung bedeutet, dass das Produkt kompakt ist.

  (2)$\Rightarrow$(1):
  Nach Satz \ref{thm:compactness}(1) genügt es zu zeigen, dass jeder Filter $\varphi$ auf einem ultrafilterkompakten Raum $X$ einen Adhärenzpunkt besitzt.
  Nach Voraussetzung existiert ein $\varphi$ umfassender Ultrafilter, welcher aufgrund der Ultrafilterkompaktheit von $X$ konvergiert.
  Damit ist ein konvergenter Oberfilter von $\varphi$ gefunden, also besitzt $\varphi$ einen Adhärenzpunkt.
\end{proof}

\begin{thm}
  Es sind äquivalent:
  \begin{enumerate}[(1)]
    \item Ein topologischer Raum ist genau dann kompakt, wenn er \textsc{Alexandroff}-\textsc{Urysohn}-kompakt ist.
    \item Ein topologischer Raum ist genau dann ultrafilterkompakt, wenn er \textsc{Alexandroff}-\textsc{Urysohn}-kompakt ist.
    \item \AC.
  \end{enumerate}
\end{thm}

\begin{proof}
  (1),(2)$\Rightarrow$(3):
  Man betrachte zwei unendliche Kardinalzahlen $a,b$.
  Dann existieren disjunkte Mengen $A,B$ mit $|A| = a$ und $|B| = b$.
  Man betrachte $X := A \cup B$ als topologischen Raum mit Topologie $\tau := \{\emptyset, A, B, A \cup B\}$.
  Dieser Raum ist sowohl kompakt, da die Topologie nur aus endlich vielen Mengen besteht, als auch ultrafilterkompakt. 
  Somit implizieren sowohl (1) als auch (2), dass $X$ \textsc{Alexandroff}-\textsc{Urysohn}-kompakt ist.
  Folglich besitzt $A \cup B$ einen vollständigen Häufungspunkt $x$.
  Angenommen $x \in A$, dann ist $a = |A| = |A \cup B| \geq b$. 
  Ist andererseits $x \in B$, dann ist $b = |B| = |A \cup B| \geq a$.

  (3)$\Rightarrow$(1),(2):
  Mit Satz \ref{thm:uftcompact} folgt aus \UFT bereits die Äquivalenz der Begriffe Kompaktheit und Ultrafilterkompaktheit.
  Es reicht also bereits zu zeigen, dass (1) gilt.

  Angenommen $X$ sei \textsc{Alexandroff}-\textsc{Urysohn}-kompakt aber nicht kompakt.
  Dann ist die Menge $\mathfrak{M}$ aller offenen Überdeckungen, die keine endliche Teilüberdeckung enthalten nicht leer und es existiert nach 1.3.5 Bartsch eine bezüglich Mächtigkeit minimale offene Überdeckung $\mathcal{O} = (O_i)_{i \in I}$.

  Nach dem Wohlordnungssatz lässt sich $I$ wohlordnen. 
  Man betrachte nun für $i \in I$ die Menge
  \begin{displaymath}
    A_i := \bigcup \,\{O_j \mid j < i, j \in I\}.
  \end{displaymath}
  Es gilt $A_i \neq X$ für alle $i \in I$, um nicht der Minimalität von $\mathcal{O}$ zu widersprechen.
  Weiterhin muss gelten $|X \setminus A_i| \geq |\mathcal{O}|$
\end{proof}

\section{Der \textsc{Boole}sche Primidealsatz}

\begin{defn}
 \item \AC, das \textit{Auswahlaxiom}.
 \item Lemma von \textsc{Zorn}.
\end{defn}

\begin{defn}
    \PIT, der \textit{\textsc{Boole}sche Primidealsatz}:
      \begin{addmargin}[2em]{2em}% 1em left, 2em right
        \textit{Jede \textsc{Boole}sche Algebra besitzt ein maximales Ideal.}
      \end{addmargin}
\end{defn}

\begin{thm}
  \label{thm:acpit}
  \AC impliziert \PIT.
\end{thm}

\begin{proof}
  Sei $X$ eine \textsc{Boole}sche Algebra und $I$ ein Ideal. 
  Man betrachte die Menge
  \begin{displaymath}
    \mathcal{A} := \{J \subsetneq X \mid \text{ $J$ ist ein $I$ umfassendes Ideal } \}.
  \end{displaymath}
  Um das \textsc{Zorn}sche Lemma anwenden zu können, betrachtet man nun eine Kette $\mathcal{K} \subseteq \mathcal{A}$.
  Es sei 
  \begin{displaymath}
    B := \bigcup_{J \in \mathcal{K}} J.
  \end{displaymath}

  Es soll nun gezeigt werden, dass $B \in \mathcal{A}$ gilt.
  Da nach Voraussetzung $\mathcal{K}$ nur aus $I$ umfassenden Idealen besteht, gilt auch $I \subseteq B$.
  Sind $x,y \in B$ so existieren $J_x, J_y \in \mathcal{K}$ mit $x \in J_x$ und $y \in J_y$.
  Da $\mathcal{K}$ eine Kette ist, kann man aus Symmetriegründen ohne Beschränkung der Allgemeinheit annehmen $J_x \subseteq J_y$ und damit $x \in J_y$.
  Da $J_y$ nach Voraussetzung ein Ideal ist gilt $x + y \in J_y \subseteq B$.
  Ist andererseits $x \in B$ und $a \in X$ mit $a \leq x$, so existiert wie zuvor ein $J \in \mathcal{K}$ mit $x \in J$ und, da $J$ ein Ideal ist, gilt auch $a \in J$ und damit $a \in B$.
  Also ist $B$ ein Ideal, welches $I$ umfasst, das bedeutet $B \in \mathcal{A}$.

  Nach Konstruktion gilt $J \subseteq B$ für alle $J \in \mathcal{K}$, und damit ist $B$ eine obere Schranke von $\mathcal{K}$ in $\mathcal{A}$.
  Das \textsc{Zorn}sche Lemma garantiert nun die Existenz maximaler Elemente in $\mathcal{A}$, also die Existenz eines maximalen, $I$ umfassenden Ideals.
\end{proof}

\begin{bem}
  Auf den ersten Blick scheint der bewiesene Satz \ref{thm:acpit} stärker zu sein als \PIT.
  Es hätte bereits genügt, anstatt $\mathcal{A}$ die Menge aller echten Ideale in $X$ zu betrachten.
  Tatsächlich sind die beiden Aussagen jedoch äquivalent.
\end{bem}

\begin{proof}
  
  Um zu zeigen, dass aus \PIT bereits zu jedem Ideal die Existenz maximaler umfassender Ideale folgt, betrachte man eine \textsc{Boole}sche Algebra $X$ mit Ideal $I$.
  
  Es induziert $I$ folgendermaßen eine Äquivalenzrelation auf $X$.
  Es soll für $x,y \in X$ gelten, dass 
  \begin{displaymath}
    x \sim y \quad \text{genau dann, wenn} \quad (x \cdot -y) + (y \cdot -x) \in I
  \end{displaymath}
  gilt.
  Auf der Menge der Restklassen 
  \begin{displaymath}
    X / I := \{[x] \mid x \in X\}
  \end{displaymath}
  lassen sich nun über die Repräsentanten folgende Operationen definieren:
  \begin{displaymath}
    [x] + [y] := [x+y], \quad
    [x] \cdot [y] := [x \cdot y], \quad
    -[x] := [-x]
  \end{displaymath}
  Damit wird $X/I$ zu einer \textsc{Boole}schen Algebra.
  Setzt man \PIT voraus, so existiert ein maximales Ideal $K \subseteq X/I$
  Betrachtet man nun die Menge
  \begin{displaymath}
    J := \{x \in X \mid [x] \in K\},
  \end{displaymath}
  so zeigt man nun, dass $J$ ein maximales Ideal in $X$ ist, welches $I$ umfasst.
  
  Dass $J$ ein Ideal ist, folgt sofort aus der Definition der Verknüpfungsoperationen auf $X/I$ und der Tatsache, dass $K$ ein Ideal ist.
  Für alle $x \in I$ gilt zudem $[x] = [0] \in X/I$. 
  Da $K$ als Ideal definitionsgemäß $[0]$ enthält, folgt $[x] \in K$, also $I \subseteq J$.
  
  Angenommen $J$ sei nicht maximal. Sei $J' \subsetneq X$ ein $J$ umfassendes Ideal.
  Dann gilt 
  \begin{displaymath}
    K \subseteq J'/I := \{[x] \mid x \in J\},
  \end{displaymath}
  denn für $[x] \in K$ ist $x \in J \subseteq J'$ und damit $[x] \in J'/X$.
  Aufgrund der Maximalität von $K$ folgt sofort $K := J/I$ und damit $J' = J$.

  Also ist $J$ maximales $I$ umfassendes Ideal.
\end{proof}

In \textsc{Boole}schen Algebren existiert das zum Ideal duale Konzept des Filters. 

\begin{defn}
  Filter
\end{defn}

Es lässt sich somit eine zu \PIT duale und damit äquivalente Formulierung für Filter finden:
  \begin{addmargin}[2em]{2em}% 1em left, 2em right
    \textit{Jede \textsc{Boole}sche Algebra besitzt einen maximalen Filter.}
  \end{addmargin}

Dass \PIT schwächer ist als \AC, kann man in QUELLE nachlesen.
Die Bedeutung von \PIT ist jedoch nicht zu unterschätzen QUELLE

\begin{thm}
  \label{thm:pitequivalence}
  Äquivalent sind
  \begin{enumerate}[(1)]
    \item \PIT.
    \item \UFT.
    \item \textsc{Tychonoff}-Satz für \textsc{Hausdorff}-Räume: 
      \begin{addmargin}[2em]{2em}% 1em left, 2em right
        Produkte kompakter \textsc{Hausdorff}-Räume sind kompakt.
      \end{addmargin}
    \item \textsc{Tychonoff}-Satz für endliche diskrete Räume: 
      \begin{addmargin}[2em]{2em}% 1em left, 2em right
        Produkte endlicher diskreter Räume sind kompakt.
      \end{addmargin}
    \item \textsc{Hilbert}-Würfel $[0,1]^I$ sind kompakt.
    \item \textsc{Kantor}-Würfel $\mathbf{2}^I$ sind kompakt.
  \end{enumerate}
\end{thm}

  \begin{proof}
    $(1)\Rightarrow(2)$: 
    Interpretiert man \PIT im Sinne von Filtern, so folgt dies sofort, da jede Potenzmengenalgebra eine \textsc{Boole}sche Algebra ist.

    $(2)\Rightarrow(3)$:

  \end{proof}


