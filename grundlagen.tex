\section{Mengentheoretische und topologische Grundlagen}

\subsection{Das Auswahlaxiom}

\subsection{Filter und Ultrafilter}

\begin{defn}
  Filter
\end{defn}

\subsection{Topologische Räume}

Konzept des metrischen Raumes

Topologie (auch ohne Metrik möglich)
Metrische Räume, wenn nicht anders vereinbart immer mit induzierter Topologie. 
Beispiel: Diskrete Metrik

Offenheit und Abgeschlossenheit

Basis und Subbasis einer Topologie

Stetigkeit

Initialtopologie und Produkttopologie

Finaltopologie und Teilraumtopologie

Trennungseigenschaften
Metrische Räume sind Hausdorff
Hausdorff vererbt sich auf Produkte und auf Teilräume

Standard Kompaktheit (offene Überdeckungen)
Stetiges Bild eines Kompaktums
Abgeschlossene Teilmengen von Kompakta sind wiederum kompakt.
Kompakte Teilmengen von Hausdorff-Räumen sind abgeschlossen.

Lokale Kompaktheit

\subsection{Kompaktheitsbegriffe auf topologischen Räumen}

\subsection{Funktionenräume}

Spezielle Topologien auf Funktionenräumen

Mengen-offene speziell Topologie der punktweisen Konvergenz oder punktweise Topologie; kompakt offene Topologie

\subsection{Der Boolsche Primidealsatz}
