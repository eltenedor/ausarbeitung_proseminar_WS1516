\chapter{Mengentheoretische und topologische Grundlagen}

%\subsection{Das Auswahlaxiom}

%\subsection{Der Boolesche Primidealsatz}

%\subsection{Kompaktheitsbegriffe auf topologischen Räumen}
%\subsection{Zum Booleschen Primidealsatz äquivalente Aussagen}

\begin{defn}
 \item \AC, das \textit{Auswahlaxiom}.
 \item Lemma von Zorn.
\end{defn}

\begin{defn}
  \begin{enumerate}[(1)]
    \item \PIT, der \textit{Boolesche Primidealsatz}:
      \begin{addmargin}[2em]{2em}% 1em left, 2em right
        \textit{Jede Boolesche Algebra besitzt ein maximales Ideal.}
      \end{addmargin}
    \item \UFT, der Ultrafiltersatz: 
      
      \begin{addmargin}[2em]{2em}% 1em left, 2em right
        \textit{Jeder Filter auf einer Menge lässt sich zu einem Ultrafilter erweitern.}
      \end{addmargin}
  \end{enumerate}
\end{defn}

\begin{thm}
  \label{thm:acpit}
  \AC impliziert \PIT.
\end{thm}

\begin{proof}
  Sei $X$ eine Boolesche Algebra und $I$ ein Ideal. 
  Man betrachte die Menge
  \begin{displaymath}
    \mathcal{A} := \{J \subsetneq X \mid \text{ $J$ ist ein $I$ umfassendes Ideal } \}.
  \end{displaymath}
  Um das Zornsche Lemma anwenden zu können, betrachtet man nun eine Kette $\mathcal{K} \subseteq \mathcal{A}$.
  Es sei 
  \begin{displaymath}
    B := \bigcup_{J \in \mathcal{K}} J.
  \end{displaymath}

  Es soll nun gezeigt werden, dass $B \in \mathcal{A}$ gilt.
  Da nach Voraussetzung $\mathcal{K}$ nur aus $I$ umfassenden Idealen besteht, gilt auch $I \subseteq B$.
  Sind $x,y \in B$ so existieren $J_x, J_y \in \mathcal{K}$ mit $x \in J_x$ und $y \in J_y$.
  Da $\mathcal{K}$ eine Kette ist, kann man aus Symmetriegründen ohne Beschränkung der Allgemeinheit annehmen $J_x \subseteq J_y$ und damit $x \in J_y$.
  Da $J_y$ nach Voraussetzung ein Ideal ist gilt $x + y \in J_y \subseteq B$.
  Ist andererseits $x \in B$ und $a \in X$ mit $a \leq x$, so existiert wie zuvor ein $J \in \mathcal{K}$ mit $x \in J$ und, da $J$ ein Ideal ist, gilt auch $a \in J$ und damit $a \in B$.
  Also ist $B$ ein Ideal, welches $I$ umfasst, das bedeutet $B \in \mathcal{A}$.

  Nach Konstruktion gilt $J \subseteq B$ für alle $J \in \mathcal{K}$, und damit ist $B$ eine obere Schranke von $\mathcal{K}$ in $\mathcal{A}$.
  Das Zornsche Lemma garantiert nun die Existenz maximaler Elemente in $\mathcal{A}$, also die Existenz eines maximalen, $I$ umfassenden Ideals.
\end{proof}

\begin{bem}
  Auf den ersten Blick scheint der bewiesene Satz \ref{thm:acpit} stärker zu sein als \PIT.
  Es hätte bereits genügt, anstatt $\mathcal{A}$ die Menge aller echten Ideale in $X$ zu betrachten.
  Tatsächlich sind die beiden Aussagen jedoch äquivalent.
\end{bem}

\begin{proof}
  
  Um zu zeigen, dass aus \PIT bereits zu jedem Ideal die Existenz maximaler umfassender Ideale folgt, betrachte man eine Boolesche Algebra $X$ mit Ideal $I$.
  
  Es induziert $I$ folgendermaßen eine Äquivalenzrelation auf $X$.
  Es soll für $x,y \in X$ gelten, dass 
  \begin{displaymath}
    x \sim y \quad \text{genau dann, wenn} \quad (x \cdot -y) + (y \cdot -x) \in I
  \end{displaymath}
  gilt.
  Auf der Menge der Restklassen 
  \begin{displaymath}
    X / I := \{[x] \mid x \in X\}
  \end{displaymath}
  lassen sich nun über die Repräsentanten folgende Operationen definieren:
  \begin{displaymath}
    [x] + [y] := [x+y], \quad
    [x] \cdot [y] := [x \cdot y], \quad
    -[x] := [-x]
  \end{displaymath}
  Damit wird $X/I$ zu einer Booleschen Algebra.
  Setzt man \PIT voraus, so existiert ein maximales Ideal $K \subseteq X/I$
  Betrachtet man nun die Menge
  \begin{displaymath}
    J := \{x \in X \mid [x] \in K\},
  \end{displaymath}
  so zeigt man nun, dass $J$ ein maximales Ideal in $X$ ist, welches $I$ umfasst.
  
  Dass $J$ ein Ideal ist, folgt sofort aus der Definition der Verknüpfungsoperationen auf $X/I$ und der Tatsache, dass $K$ ein Ideal ist.
  Für alle $x \in I$ gilt zudem $[x] = [0] \in X/I$. 
  Da $K$ als Ideal definitionsgemäß $[0]$ enthält, folgt $[x] \in K$, also $I \subseteq J$.
  
  Angenommen $J$ sei nicht maximal. Sei $J' \subsetneq X$ ein $J$ umfassendes Ideal.
  Dann gilt 
  \begin{displaymath}
    K \subseteq J'/I := \{[x] \mid x \in J\},
  \end{displaymath}
  denn für $[x] \in K$ ist $x \in J \subseteq J'$ und damit $[x] \in J'/X$.
  Aufgrund der Maximalität von $K$ folgt sofort $K := J/I$ und damit $J' = J$.

  Also ist $J$ maximales $I$ umfassendes Ideal.
\end{proof}

In Boolschen Algebren existiert das zum Ideal duale Konzept des Filters. 

\begin{defn}
  Filter
\end{defn}

Es lässt sich somit eine zu \PIT duale und damit äquivalente Formulierung für Filter finden:
  \begin{addmargin}[2em]{2em}% 1em left, 2em right
   \textit{Jede Boolesche Algebra besitzt einen maximalen Filter.}
  \end{addmargin}

Dass \PIT schwächer ist als \AC, kann man in QUELLE nachlesen.
Die Bedeutung von \PIT ist jedoch nicht zu unterschätzen QUELLE

\begin{thm}
  Äquivalent sind
\begin{enumerate}[(1)]
    \item \PIT.
    \item \UFT.
    \item Tychonoff-Satz für Hausdorff-Räume: 
      \begin{addmargin}[2em]{2em}% 1em left, 2em right
        Produkte kompakter Hausdorff-Räume sind kompakt.
      \end{addmargin}
    \item Tychonoff-Satz für endliche diskrete Räume: 
      \begin{addmargin}[2em]{2em}% 1em left, 2em right
        Produkte endlicher diskreter Räume sind kompakt.
      \end{addmargin}
    \item Hilbert-Würfel $[0,1]^I$ sind kompakt.
    \item Kantor-Würfel $\mathbf{2}^I$ sind kompakt.
  \end{enumerate}

  \begin{proof}
    $(1)\Rightarrow(2)$:
  \end{proof}<++>

\end{thm}

