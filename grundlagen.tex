\chapter{Grundlagen aus Mengenlehre und Topologie}

Dieses Kapitel schafft die nötigen Grundlagen für das Folgekapitel.
Es soll in allen aufgeführten Aussagen und den dazugehörigen Beweisen stets von \ZF ausgegangen werden, falls eine Erweiterung des Axiomensystems nicht explizit als Voraussetzung aufgeführt wurde.

\section{Kompaktheitsbegriffe auf topologischen Räumen}

Das Auswahlaxiom oder Abschwächungen davon dienen oftmals als Bindeglied zwischen unterschiedlichen Konzepten in der Mathematik, da sie oft eine zentrale Rolle im Beweis der Äquivalenz dieser Konzepte einnehmen. 
Ein für die vorliegende Ausarbeitung zentrales Konzept ist das der Kompaktheit topologischer Räume. 
Es existieren unterschiedliche Kompaktheitsbegriffe, die im Rahmen von \ZF nicht notwendig äquivalent sind.
Diese Konzepte sollen zunächst unabhängig voneinander eingeführt werden. 
Anschließend wird basierend auf \cite{herrlich2006axiom} dargestellt, unter welchen Bedingungen die Äquivalenz der Konzepte gegeben ist.
Zuvor werden die im Folgenden benötigten Begriffe definiert. 

\begin{defn}
  Sei $(X,\tau)$ ein topologischer Raum.
  Ein Punkt $x \in X$ heißt \textit{Adhärenzpunkt} eines Filters $\varphi \in \mathcal{F}(X)$, falls ein bezüglich $\tau$ gegen $x$ konvergenter Oberfilter $\psi \supseteq \varphi$ existiert.
\end{defn}

\begin{prop}
  \label{prop:adherence}
  Sei $(X,\tau)$ ein topologischer Raum.
  Ein Punkt $x \in X$ ist genau dann Adhärenzpunkt eines Filters $\varphi \in \mathcal{F}(X)$, falls
  \begin{displaymath}
    x \in \bigcap_{F \in \varphi} \overline{F}.
  \end{displaymath}
\end{prop}

\begin{proof}
  Sei zunächst $x$ Adhärenzpunkt eines Filters $\varphi$ auf $X$.
  Nach Definition existiert ein gegen $x$ konvergenter Oberfilter $\psi \supseteq \varphi$, also gilt $\psi \supseteq \dot x \cap \tau$.
  Somit gilt aufgrund der Filtereigenschaft von $\psi$ für alle $F \in \varphi$ und alle $U \in \dot x \cap \tau$, dass  $F \cap U \neq \emptyset$.
  Also besitzt jede offene Umgebung $U$ von $x$ mit allen $F \in \varphi$ einen nichtleeren Schnitt, was bedeutet, dass $x \in \overline F$ für alle $F  \in \varphi$.

  Sei andererseits $x \in \bigcap_{F \in \varphi} \overline F$.
  Nach Definition des Abschlusses einer Menge, gilt somit für alle $F \in \varphi$ und $U \in \dot x \cap \tau$, dass $F \cap U \neq \emptyset$.
  Es ist also $\varphi \cup (\dot x \cap \tau)$ eine Filtersubbasis.
  Man bezeichne mit $\psi$ den davon erzeugten Filter.
  Nach Konstruktion gelten dann $ \psi \supseteq \varphi $ und $ \psi \overset{\tau}{\to} x$.
  Also ist $x$ ein Adhärenzpunkt von $\varphi$.
\end{proof}

\begin{defn}
  Sei $(X,\tau)$ ein topologischer Raum.
  Dann heißt $x \in X$ \textit{vollständiger Häufungspunkt} einer Teilmenge $A \subseteq X$, wenn für alle Umgebungen $U$ von $x$ die Mengen $A$ und $A \cap U$ dieselbe Kardinalzahl besitzen, also $|A| = | A \cap U |$.
\end{defn}

Es folgen nun einige Kompaktheitsbegriffe, welche im Rahmen von \ZFC für gewöhnlich synonym verwendet werden.
\begin{defn}
  \label{def:compactness}
  Ein topologischer Raum $(X,\tau)$ heißt
  \begin{enumerate}[(1)]
    \item kompakt, \textit{falls sich aus jeder offenen Überdeckung von $X$ eine Teilüberdeckung auswählen lässt},
    \item filterkompakt, \textit{falls jeder Filter auf $X$ einen Adhärenzpunkt besitzt},
    \item ultrafilterkompakt, \textit{falls jeder Ultrafilter auf $X$ konvergiert},
    \item \textsc{Alexandroff}-\textsc{Urysohn}-kompakt, \textit{falls jede unendliche Teilmenge von $X$ einen vollständigen Häufungspunkt besitzt},
    \item \textsc{Tychonoff}-kompakt, \textit{falls $X$ homöomorph zu einem abgeschlossenen Teilraum eines \textsc{Hilbert}-Würfels $[0,1]^I$ ist}.
  \end{enumerate}
\end{defn}

Im Folgenden sollen die Zusammenhänge der unterschiedlichen Kompaktheitsbegriffe, gegebenenfalls unter Hinzunahme weiterer Annahmen, erörtert werden.

\begin{thm}
\label{thm:compactness}
  Sei $(X,\tau)$ ein topologischer Raum. Dann gelten:
  \begin{enumerate}[(1)]
    \item Der Raum $X$ ist genau dann kompakt, wenn er filterkompakt ist. 
    \item Ist $X$ filterkompakt, dann auch ultrafilterkompakt.
  \end{enumerate}
\end{thm}

\begin{proof}
  $(1)$: 
  Sei $(X,\tau)$ kompakt und $\varphi$ ein Filter auf $X$.
  Angenommen, $\varphi$ besitze keinen Adhärenzpunkt.
  Dann gilt nach Proposition \ref{prop:adherence}
  \begin{displaymath}
    \bigcap_{F \in \varphi} \overline{F} = \emptyset.
  \end{displaymath}
  Folglich ist 
  \begin{displaymath}
    %X = \bigcup_{F \in \varphi} X \setminus \overline{F} \subseteq \bigcup_{F \in \varphi} X \setminus F 
    X = \bigcup_{F \in \varphi} X \setminus \overline{F} 
  \end{displaymath}
  eine offene Überdeckung von $X$.
  Da $X$ als kompakt vorausgesetzt wird, existieren $F_1,\allowbreak \dots , F_n \in \varphi$ mit
  \begin{displaymath}
    %\bigcup_{i = 1}^n X \setminus F_i = X,
    X = \bigcup_{i = 1}^n X \setminus \overline{F_i} \subseteq \bigcup_{i = 1}^n X \setminus F_i,
  \end{displaymath}
  woraus wiederum
  \begin{displaymath}
    X \setminus \bigcup_{i = 1}^n F_i = \bigcap_{i = 1}^n F_i = \emptyset
  \end{displaymath}
  folgt. 
  Dies steht jedoch im Widerspruch zur Filtereigenschaft.

  Sei umgekehrt $X$ filterkompakt und $(O_i)_{i \in I}$ eine offene Überdeckung von $X$.
  Angenommen, es existiere keine endliche Teilüberdeckung.
  Man betrachte nun die Familie
  \begin{displaymath}
    \mathfrak{B} := \left\{ X \setminus \mathcal{O} \mid \mathcal{O} = \bigcup_{k = 1}^n O_{i_k}\right\}.
  \end{displaymath}
  Diese Familie ist nichtleer, da sie $X \setminus O_i$ für alle $i \in I$ enthält, und zudem abgeschlossen unter endlichen Schnitten.
  Folglich ist $\mathfrak{B}$ eine Filterbasis.
  Es bezeichne $\varphi$ den aus dieser Basis durch Obermengenbildung erzeugten Filter.
  Nach Voraussetzung existiert ein Oberfilter $\psi \supseteq \varphi$ der bezüglich der Topologie $\tau$ gegen ein $x \in X$ konvergiert.
  Aus der Überdeckungseigenschaft folgt, dass zusätzlich ein $i_0 \in I$ existiert mit $x \in O_{i_0}$.
  Aufgrund der Konvergenz des Filters $\psi$ folgt $O_{i_0} \in \psi$.
  Dies steht jedoch im Widerspruch zu $\psi \supseteq \mathfrak{B}$, denn deswegen gilt bereits $X \setminus O_{i_0} \in \psi$.

  $(2)$: Da Ultrafilter bereits bezüglich Inklusion maximal sind, also keine echten Oberfilter besitzen, impliziert Filterkompaktheit bereits Ultrafilterkompaktheit.
\end{proof}

Eine wichtige Folgerung aus dem Auswahlaxiom ist der folgende Satz:
\begin{defn}
    \UFT, der Ultrafiltersatz: 
      \begin{addmargin}[2em]{2em}% 1em left, 2em right
        \textit{Jeder Filter auf einer Menge lässt sich zu einem Ultrafilter erweitern.}
      \end{addmargin}
\end{defn}

Ein Beweis dieser Aussage im Rahmen von \ZFC wird in Satz \ref{thm:acpit} für Ideale auf \textsc{Boole}schen Algebren erbracht. 
Zuvor soll jedoch die Rolle von \UFT im Rahmen der Übereinstimmung gewisser Kompaktheitsbegriffe analysiert werden.

\begin{lem}
  \label{lem:uftprod}
  Sei $(X_i,\tau_i)_{i \in I}$ eine Familie ultrafilterkompakter Räume. 
  Dann ist ihr Produkt $\prod_{i \in I} X_i$ wiederum ultrafilterkompakt.
\end{lem}

\begin{proof}
  Es sei $\varphi \in \mathcal{F}_0(\prod_{i \in I} X_i)$.
  Dann sind auch die Bilder $\pi_i(\varphi)$ unter den kanonischen Projektionen $\pi_i$ Ultrafilter auf $X_i$.
  Aufgrund der Ultrafilterkompaktheit der einzelnen Faktoren konvergiert jeder Bildfilter gegen ein $x_i \in X_i$.
  Die Eigenschaft der Initialtopologie impliziert dann jedoch, dass auch $\varphi$ gegen $(x_i)_{i \in I}$ konvergiert.
\end{proof}

\begin{thm}
  \label{thm:uftcompact}
  Es sind äquivalent:
  \begin{enumerate}[(1)]
    \item Ein topologischer Raum ist genau dann kompakt, wenn er ultrafilterkompakt ist.
    \item \UFT.
  \end{enumerate}
\end{thm}

\begin{proof}
  (1)  $\Rightarrow$  (2):
  Es soll hierzu Satz \ref{thm:pitequivalence}((4) $\Rightarrow$ (3)) verwendet werden.
  Man beachte, dass dies gerade die Implikation ist, welche nicht Bezug auf Satz \ref{thm:uftcompact} nimmt.
  Es reicht daher aus zu zeigen, dass Produkte kompakter \textsc{Hausdorff}\hyp{}Räume wiederum kompakt sind. 
  Produkte ultrafilterkompakter \textsc{Hausdorff}\hyp{}Räume sind jedoch nach Lemma \ref{lem:uftprod} wieder ultrafilterkompakt, was nach Voraussetzung bedeutet, dass das Produkt kompakt ist.

  (2) $\Rightarrow$ (1):
  Nach Satz \ref{thm:compactness}(1) genügt es zu zeigen, dass jeder Filter $\varphi$ auf einem ultrafilterkompakten Raum $X$ einen Adhärenzpunkt besitzt.
  Nach Voraussetzung existiert ein $\varphi$ umfassender Ultrafilter, welcher aufgrund der Ultrafilterkompaktheit von $X$ konvergiert.
  Damit ist ein konvergenter Oberfilter von $\varphi$ gefunden, also besitzt $\varphi$ einen Adhärenzpunkt.
\end{proof}

Um die nächste Beziehung zwischen Kompaktheitsbegriffen herzustellen, benötigt man zunächst den folgenden Satz aus der Mengenlehre.

\begin{thm}
  \label{thm:comparablecardinal}
  Die folgenden Aussagen sind äquivalent:
  \begin{enumerate}[(1)]
    \item Je zwei Kardinalzahlen sind bezüglich $\leq$ vergleichbar.
    \item Je zwei Kardinalzahlen sind bezüglich $\leq^*$ vergleichbar.
    \item \AC.
  \end{enumerate}
  Hierbei gelte für zwei Mengen $X,Y$ die Relation $X \leq Y$, falls eine Injektion $f \colon X \to Y$ existiert und es gelte die Relation $X \leq^* Y$, falls $X = \emptyset$ gilt oder eine Surjektion $g \colon Y \to X$ existiert.
\end{thm}

\begin{proof}
  (1) $\Rightarrow$ (2):
  Es sei $f \colon X \to Y$ die nach (1) existierende injektive Abbildung.
  Ist $X \neq \emptyset$, so existiert ein $x_0 \in X$ und es lässt sich für alle $y \in Y$ die surjektive Abbildung
  \begin{displaymath}
    g \colon Y \to X, \quad g(x) := 
    \begin{cases}
      f^{-1}(\{y\}), \text{ falls } y \in f(X)\\
      x_0, \text{ sonst}
    \end{cases}
  \end{displaymath}
  definieren. Aufgrund der Injektivität von $f$ besteht $f^{-1}(\{y\})$ aus nur einem Punkt, daher ist $g$ wohldefiniert.

  (2) $\Rightarrow$ (3):
  Sei $X$ eine Menge.
  Es soll gezeigt werden, dass auf $X$ eine Wohlordnung existiert. 
  Ist $X$ leer, so gibt es nichts zu zeigen. Im Folgenden werde also $X \neq \emptyset$ angenommen.

  Allgemein gilt für eine weitere Menge $Y$, dass $|Y| \leq^* |X|$ impliziert, dass $|Y| \leq |\mathcal{P}(X)|$.
  Denn, ist $f \colon X \to Y$ die entsprechende surjektive Abbildung, so lässt sich durch
  \begin{displaymath}
    g \colon Y \to \mathcal{P}(X), \quad y \mapsto f^{-1}(\{y\})
  \end{displaymath}
  eine Abbildung definieren, welche injektiv ist.
  Falls nämlich für $y,z \in Y$ gilt, dass $g(y) = g(z)$, so folgt daraus zunächst $f^{-1}(\{y\}) = f^{-1}(\{z\})$ und daraus über die Surjektivität von $f$, dass auch $\{y\} = \{z\}$ gilt.


  Es bezeichne nun $\aleph$ die nach dem \textit{Satz von \textsc{Hartogs} }
  \begin{addmargin}[2em]{2em}% 1em left, 2em right
    \textit{Zu jeder Menge $A$ gibt es wenigstens eine wohlgeordnete Menge $B$, deren Kardinalität nicht durch die Kardinalität von $A$ beschränkt wird.} 
  \end{addmargin}
  bezüglich der Potenzmenge $\mathcal{P}(X)$ kleinste existierende Kardinalzahl mit $\aleph \not\leq \mathcal{P}(X)$.
  Man beachte zudem, dass die Existenz von $\aleph$ bereits in \ZF gilt.
  Dann folgt nach dem zuvor Bewiesenen auch $\aleph \not\leq^* |X|$ was zusammen mit (2) $|X|\leq^* \aleph$ impliziert.
  Also existiert im Falle $X \neq \emptyset$ eine Surjektion $f \colon \aleph \to X$.
  Es lässt sich aufgrund der Wohlordnung auf $\aleph$ eine Abbildung
  \begin{displaymath}
    g \colon X \to \aleph, \quad g(x) := \min f^{-1}(\{x\})
  \end{displaymath}
  definieren.
  Fasst man $\aleph$ als Ordinalzahl mit der Inklusion $\subseteq$ als Ordnungsrelation auf, so ist diese Abbildung injektiv, denn für $x,y \in X$ mit $g(x) = g(y)$ gilt
  \begin{displaymath}
    f^{-1}(\{x\}) \subseteq f^{-1}(\{y\}) \quad\text{oder}\quad f^{-1}(\{y\}) \subseteq f^{-1}(\{x\}),
  \end{displaymath}
  woraus mit der Surjektivität von $f$ bereits
  \begin{displaymath}
    \{x\} \subseteq \{y\} \quad\text{oder}\quad \{y\} \subseteq \{x\}
  \end{displaymath}
  folgt.
  Auf $X$ existiert damit eine Ordnung $\leq_X$, indem man 
  \begin{displaymath}
    x \leq_X y, \quad\text{ falls }\quad g(x) \subseteq g(y)
  \end{displaymath}
  gilt, als Ordnungsrelation verwendet. 
  Bezüglich dieser Relation ist $X$ wohlgeordnet.
  Ist nämlich $T \subseteq X$ nichtleer, so ist auch $g(T) \subseteq \aleph$ nichtleer und besitzt ein minimales Element $m \in g(T)$. Aufgrund der Injektivität von $g$ korrespondiert hierzu ein eindeutig bestimmtes $x \in X$, sodass 
  \begin{displaymath}
    g(x) = m = \min g(T).
  \end{displaymath}
  Ist nun $y \in T$ mit $y \leq_X x$, so ist insbesondere $g(y) \in g(T)$, also gilt 
  \begin{displaymath}
    g(x) = \min g(T) \leq g(y),
  \end{displaymath}
  woraus über die Definition der Ordnungsrelation sofort $y \leq_X x$ also $y = x$ folgt.
  Damit ist $x$ das minimale Element von $T$.

  Die Existenz einer Wohlordnung auf einer beliebigen Menge $X$ impliziert \AC.

  (3) $\Rightarrow$ (1):
  Seien $X,Y$ Mengen. Nach (3) lassen sich $X$ und $Y$ wohlordnen.
  Als wohlgeordnete Mengen sind sie nach dem \textit{Vergleichbarkeitssatz} \cite[S. 28]{bartsch2015allgemeine} entweder ordnungsisomorph oder es ist eine der Mengen isomorph zu einem initialen Intervall der anderen Menge.
  Im Ersten Fall gilt $|X| = |Y|$.
  Ist $X$ isomorph zu einem initialen Intervall von $Y$, so gilt $|X| \leq |Y|$ oder, falls $Y$ isomorph zu einem initialen Intervall von $X$ ist, gilt $|Y| \leq |X|$.
\end{proof}

\newpage
\begin{thm}
  \label{thm:alexandroffUrysohnCompactAC}
  Es sind äquivalent:
  \begin{enumerate}[(1)]
    \item Ein topologischer Raum ist genau dann kompakt, wenn er \textsc{Alexandroff}\hyp{}\textsc{Urysohn}\hyp{}kompakt ist.
    \item Ein topologischer Raum ist genau dann ultrafilterkompakt, wenn er \textsc{Alexandroff}\hyp{}\textsc{Urysohn}\hyp{}kompakt ist.
    \item \AC.
  \end{enumerate}
\end{thm}

\begin{proof}
  (1),(2) $\Rightarrow$ (3):
  Man betrachte zwei unendliche Kardinalzahlen $a,b$.
  Dann existieren disjunkte Mengen $A,B$ mit $|A| = a$ und $|B| = b$.
  Man betrachte $X := A \cup B$ als topologischen Raum mit Topologie $\tau := \{\emptyset, A, B, A \cup B\}$.
  Dieser Raum ist sowohl kompakt, da die Topologie nur aus endlich vielen Mengen besteht, als auch ultrafilterkompakt. 
  Somit implizieren sowohl (1) als auch (2), dass $X$ \textsc{Alexandroff}\hyp{}\textsc{Urysohn}\hyp{}kompakt ist.
  Folglich besitzt $A \cup B$ einen vollständigen Häufungspunkt $x$.
  Angenommen, $x \in A$, dann ist $a = |A| = |A \cup B| \geq b$. 
  Ist andererseits $x \in B$, dann ist $b = |B| = |A \cup B| \geq a$.
  Folglich gilt $a \leq b$ oder $b \leq a$.
  Zusammen mit Satz \ref{thm:comparablecardinal} und der Tatsache, dass je zwei endliche Kardinalzahlen vergleichbar sind folgt \AC.

  (3) $\Rightarrow$ (1),(2):
  Mit Satz \ref{thm:uftcompact} folgt aus \UFT bereits die Äquivalenz der Begriffe Kompaktheit und Ultrafilterkompaktheit.
  Es reicht also bereits zu zeigen, dass (1) gilt.

  Angenommen, $X$ sei \textsc{Alexandroff}\hyp{}\textsc{Urysohn}\hyp{}kompakt aber nicht kompakt.
  Dann ist die Menge $\mathfrak{M}$ aller offenen Überdeckungen, die keine endliche Teilüberdeckung enthalten nichtleer und es existiert nach Übergang zu den korrespondierenden Kardinalzahlen eine bezüglich Mächtigkeit minimale offene Überdeckung $\mathcal{O} = (O_i)_{i \in I}$ mit Kardinalzahl $\mathcal{C}$, da die Menge aller Ordinalzahlen wohlgeordnet ist.

  Es bezeichne nun $f$ die zur $\mathcal{C}$ korrespondierende Bijektion $\mathcal{C} \to \mathcal{O}$ und für alle $b \in \mathcal{C}$ sei 
  \begin{displaymath}
    \mathcal{A}_b := \{ f(a) \mid a < b\} \subseteq \mathcal{O}.
  \end{displaymath}
  Es gilt zudem
  \begin{displaymath}
    | \{ a \in \mathcal{C} \mid a < b \} | = b < \mathcal{C}
  \end{displaymath}
  und folglich auch $| \mathcal{A}_b | < \mathcal{C}$, da es sich hierbei um das Bild unter einer bijektiven Abbildung $f$ handelt.

  Dann folgt für alle $b \in \mathcal{C}$, dass $\mathcal{A}_b$ keine Überdeckung von $X$ sein kann.
  Angenommen, $\bigcup \mathcal{A}_b \supseteq X$, dann ergibt sich sofort $\mathcal{A}_b \in \mathfrak{M}$, denn als Teilmenge von $\mathcal{O}$ kann auch $\mathcal{A}_b$ keine endliche Teilüberdeckung von $X$ besitzen.
  Dies widerspricht jedoch der Wahl von $\mathcal{C}$, da $|\mathcal{A}_b| < \mathcal{C}$ für alle $b \in \mathcal{C}$ vorausgesetzt wird.

  Für alle $b \in \mathcal{C}$ und 
  \begin{displaymath}
  U_b := \bigcup \mathcal{A}_b 
  \end{displaymath}
  gilt des Weiteren $|X \setminus U_b| \geq \mathcal{C}$.
  Nimmt man an, es gelte $|X \setminus U_{b'}| < \mathcal{C}$ für ein $b' \in \mathcal{C}$, so lässt sich für alle $x \in X \setminus U_{b'}$ ein $O_x \in \mathcal{O}$ auswählen, welches $x$ enthält.
  Dann ist für $\mathcal{O}' := \{O_x \mid x \in X \setminus U_{b'}\}$ jedoch
  \begin{displaymath}
    \mathcal{A}_{b'} \cup  \mathcal{O}'
  \end{displaymath}
  eine Überdeckung von $X$, die aus demselben Grund wie oben keine endliche Teilüberdeckung enthalten kann.
  Diese Überdeckung besitzt jedoch eine Kardinalität kleiner $\mathcal{C}$, denn $| \mathcal{O}' | = | X \setminus U_{b'} |$, und damit besitzt die Vereinigung von $\mathcal{A}_{b'}$ und $\mathcal{O}'$ die größere der zugehörigen Kardinalitäten, welche weiterhin kleiner ist als $\mathcal{C}$.
  Dies widerspricht jedoch der Wahl von $\mathcal{O}$ als Überdeckung mit minimaler Kardinalität.

  Es ist nun möglich für alle $b \in \mathcal{C}$ ein $x_b \in X \setminus U_b$ mit $x_a \neq x_b$ für alle $a < b$ zu wählen. 
  Für alle $b \in \mathcal{C}$ gilt nämlich $|\{x_a \mid a < b\}| < \mathcal{C}$ und, da $|X \setminus U_b| \geq \mathcal{C}$, folgt dann
  \begin{displaymath}
    S := X \setminus (U_b \cup \{x_a \mid a < b\}) \neq \emptyset,
  \end{displaymath}
  denn es ist
  \begin{displaymath}
    | S | = | (X \setminus U_b) \setminus \{x_a \mid a < b\}) | \geq C.
  \end{displaymath}
  Es kann $| S | < C$ nicht gelten, denn sonst ist $ |( X \setminus U_b )| = |S \cup \{x_a \mid a < b\}| < \mathcal{C}$ im Widerspruch zur Voraussetzung.
  Insbesondere ist also $S \neq \emptyset$ und man kann somit für alle $b \in \mathcal{C}$ ein zu allen $x_a$ mit $a < b$ verschiedenes $x_b \in S$ auswählen.

  Man betrachte nun die Menge
  \begin{displaymath}
    M := \{ x_b \mid b \in \mathcal{C} \}.
  \end{displaymath}
  Es soll nun gezeigt werden, dass sie keinen vollständigen Häufungspunkt besitzt. 
  Aufgrund der Überdeckungseigenschaft von $\mathcal{O}$ existiert zunächst für alle $x \in X$ ein $b \in \mathcal{C}$ mit $x \in U_b$.
  Die Menge aller $b$ mit ebendieser Eigenschaft ist demnach nichtleer und besitzt als Teilmenge der wohlgeordneten Menge $\mathcal{C}$ ein minimales Element.
  Im Folgenden bezeichne nun $b$ dieses minimale Element.
  Es ist $U_b$ dann eine offene Umgebung von $x$ und es gilt 
  \begin{displaymath}
    U_b \cap M \subseteq \{x_a \mid a < b\}
  \end{displaymath}
  also auch
  \begin{displaymath}
    |U_b \cap M| \leq b < \mathcal{C}.
  \end{displaymath}
  Dies Widerspricht jedoch der Annahme $X$ sei \textsc{Alexandroff}\hyp{}\textsc{Urysohn}\hyp{}kompakt.

  Sei nun umgekehrt $X$ kompakt.
  Angenommen, $X$ sei nicht \textsc{Alexandroff}\hyp{}\textsc{Urysohn}\hyp{}kompakt.
  Dann existiert eine unendliche Teilmenge $A \subseteq X$ ohne vollständigen Häufungspunkt.
  Das bedeutet, dass für alle $x \in X$ eine offene Umgebung $U_x$ existiert mit
  \begin{displaymath}
    | A \cap U_x | < | A |. \tag{$\ast$}
  \end{displaymath}
  Die Familie $\{U_x \mid x \in X\}$ ist nach Konstruktion eine Überdeckung von $X$ und aufgrund der vorausgesetzten Kompaktheit existiert eine endliche Teilüberdeckung $\{U_{x_1},\dots,U_{x_n}\}$.
  Es gilt somit
  \begin{displaymath}
    A = \bigcup_{i = 1}^n (A \cap U_{x_i}).
  \end{displaymath}
  Dies widerspricht jedoch ($\ast$), denn dann wäre $A$ die endliche Vereinigung von Mengen echt kleinerer Kardinalität. 
\end{proof}

\section{Der \textsc{Boole}sche Primidealsatz}

Der \textsc{Boole}sche Primidealsatz (\PIT) ist eine im Vergleich zum Auswahlaxiom schwächere Aussage, wie man in \cite{halpern1971boolean} nachlesen kann.
Die Bedeutung von \PIT ist jedoch nicht zu unterschätzen. Eine Reihe zu \PIT äquivalenter Aussagen aus unterschiedlichen Gebieten der Mathematik findet man in \cite{dau1994diplom}.
Auch ist \PIT zu vielen der noch zu besprechenden Kompaktheitsaussagen äquivalent. 
In diesem Kapitel stehen speziell Kompaktheitsaussagen über kartesische Produkte im Vordergrund. 
Diese Kompaktheitsaussagen werden im Folgekapitel die Verbindung der \textsc{Ascoli}\hyp{}Sätze zu \PIT ermöglichen.

Im Folgenden werde wie in \cite{graetzer2003general} eine \textit{\textsc{Boole}sche Algebra} aufgefasst als ein \textit{\textsc{Boole}scher Verband} $(B; \;\land, \;\lor, \;', \;1, \;0)$, also ein komplementärer distributiver Verband. 
Die Verträglichkeit der algebraischen Struktur mit der Verbandsstruktur ist gegeben durch 
\begin{displaymath}
  x \land y = \inf\{x,y\} \quad \text{und} \quad x \lor y = \sup\{x,y\}
\end{displaymath}
beziehungsweise durch die Halbordnungsrelation 
\begin{displaymath}
  x \leq y \quad \text{gilt genau dann, wenn} \quad x \land y = x.
\end{displaymath}
Zusätzlich gilt 
\begin{displaymath}
  x \land y = x \quad \text{genau dann, wenn} \quad x \lor y = y,
\end{displaymath}
denn falls $x \land y = y$, so gilt $x \leq y$ also auch $y = \sup\{x,y\} = x \lor y$. 
Die Umkehrung folgt analog.

\begin{defn}
  Sei $B$ eine \textsc{Boole}sche Algebra. Dann bezeichnet man eine nichtleere echte Teilmenge $I \subsetneq B$ als \textit{Ideal} in $B$, falls folgende Bedingungen gelten:
  \begin{enumerate}[(1)]
    \item Aus $u \in I$ und $v \in I$ folgt $u \lor v \in I$.
    \item Für $u \in I$, $x \in B$ und $x \leq u$ folgt  $x \in I$.
  \end{enumerate}
  Ein Ideal $I$ in $B$ heißt \textit{Primideal}, falls zusätzlich gilt:
  \begin{enumerate}[(3)]
    \item Für alle $x \in B$ ist entweder $x \in I$ oder $x' \in I$.
  \end{enumerate}
\end{defn}

\begin{defn}
    \PIT, der \textit{\textsc{Boole}sche Primidealsatz}:
      \begin{addmargin}[2em]{2em}% 1em left, 2em right
        \textit{Jede \textsc{Boole}sche Algebra besitzt ein maximales Ideal.}
      \end{addmargin}
\end{defn}

\begin{thm}
  \label{thm:acpit}
  \AC impliziert \PIT.
\end{thm}

\begin{proof}
  Sei $B$ eine \textsc{Boole}sche Algebra und $I$ ein Ideal. 
  Man betrachte die Menge
  \begin{displaymath}
    \mathcal{A} := \{J \subsetneq B \mid \text{ $J$ ist ein $I$ umfassendes Ideal } \}.
  \end{displaymath}
  Um das \textsc{Zorn}sche Lemma anwenden zu können, betrachtet man nun eine Kette $\mathcal{K} \subseteq \mathcal{A}$.
  Es sei 
  \begin{displaymath}
    S := \bigcup_{J \in \mathcal{K}} J.
  \end{displaymath}

  Es soll nun gezeigt werden, dass $S \in \mathcal{A}$ gilt.
  Da $\mathcal{K}$ nach Voraussetzung nur aus $I$ umfassenden Idealen besteht, gilt auch $I \subseteq S$.
  Sind $u,v \in S$, so existieren $J_u, J_v \in \mathcal{K}$ mit $u \in J_u$ und $v \in J_v$.
  Da $\mathcal{K}$ eine Kette ist, kann man aus Symmetriegründen ohne Beschränkung der Allgemeinheit $J_u \subseteq J_v$ annehmen, woraus $u \in J_v$ folgt.
  Da $J_v$ nach Voraussetzung ein Ideal ist, gilt $u \lor v \in J_v \subseteq S$.
  Ist andererseits $u \in S$ und $x \in B$ mit $x \leq u$, so existiert wie zuvor ein $J \in \mathcal{K}$ mit $u \in J$ und, da $J$ ein Ideal ist, gilt auch $x \in J$ und damit $x \in S$.
  Also ist $S$ ein Ideal, welches $I$ umfasst, das bedeutet $S \in \mathcal{A}$.

  Nach Konstruktion gilt $J \subseteq S$ für alle $J \in \mathcal{K}$ und damit ist $S$ eine obere Schranke von $\mathcal{K}$ in $\mathcal{A}$.
  Das \textsc{Zorn}sche Lemma garantiert nun die Existenz maximaler Elemente in $\mathcal{A}$, also die Existenz eines maximalen, $I$ umfassenden Ideals.
\end{proof}

\begin{bem}
  Auf den ersten Blick scheint der bewiesene Satz \ref{thm:acpit} stärker zu sein als \PIT.
  Es hätte bereits genügt, anstatt $\mathcal{A}$ die Menge aller echten Ideale in $B$ zu betrachten.
  Tatsächlich sind die beiden Aussagen jedoch äquivalent.
\end{bem}

\begin{proof}
  
  Um zu zeigen, dass aus \PIT bereits für jedes Ideal einer \textsc{Boole}schen Algebra die Existenz maximaler umfassender Ideale folgt, betrachte man eine \textsc{Boole}sche Algebra $X$ mit Ideal $I$.
  
  Es induziert $I$ folgendermaßen eine Äquivalenzrelation auf $X$.
  Es soll für $x,y \in X$ gelten, dass 
  \begin{displaymath}
    x \sim y \quad \text{genau dann, wenn} \quad x \bigtriangleup y := (x \land y') \lor (y \land x') \in I
  \end{displaymath}
  gilt.
  Man bezeichnet die Operation $\bigtriangleup$ auch als \textit{symmetrische Differenz}.

  Es ist klar, dass die so definierte Relation reflexiv und symmetrisch ist.
  Für die Transitivität betrachte man $x,y,z \in X$ mit $x \sim y$ und $y \sim z$.
  Dann gilt
  \begin{align*}
    ((x  \lor y  ) \land ((x \land y') \lor z' ),
    &=  ((x \land y') \lor y  ) \land ((x \land y') \lor z' ) \\
    &= (x \land y') \lor (y \land z') \in I
  \end{align*}
  aufgrund der Distributivität.
  Zudem gelten $x \leq x \lor y$ und $z' \leq ((x \land y') \lor z')$, was wiederum
  \begin{displaymath}
    x \land z' \leq  ((x  \lor y  ) \land ((x \land y') \lor z' )
  \end{displaymath}
  impliziert.
  Aufgrund der Idealeigenschaft (1) folgt damit $x \land z' \in I$.
  Analog zeigt man $z \land x' \in I$.
  Daraus folgt mit der Idealeigenschaft (2) dann $x \sim z$.
  Also ist obige Relation tatsächlich eine Äquivalenzrelation.

  Auf der Menge der Restklassen 
  \begin{displaymath}
    X / I := \{[x] \mid x \in X\}
  \end{displaymath}
  lassen sich nun über die Repräsentanten folgende Operationen definieren:
  \begin{displaymath}
    [x] \lor [y] := [x \lor y], \quad
    [x] \land [y] := [x \land y], \quad
    [x]' := [x']
  \end{displaymath}
  Damit wird $X/I$ zu einer \textsc{Boole}schen Algebra, sofern die Operationen wohldefiniert sind.
  Seien dazu $x \sim a$ und $y \sim b$ gegeben.
  Dann gilt
  \begin{align*}
    (x \lor y) \land (a \lor b)'
    &= (x \lor y) \land (a' \land b') \\
    &= (x \land (a' \land b')) \lor (y \land (a' \land b')) \\
    &\leq (x \land a') \lor (y \land b')  \in I,
  \end{align*}
  also gilt auch $(x \lor y) \land (a \lor b)' \in I$ aufgrund der Idealeigenschaft (2).
  Analog zeigt man $ (a \lor b) \land (x \lor y)' \in I$.
  Damit folgt dann $x \lor y \sim a \lor b$ mit Idealeigenschaft (1).

  Des weiteren gilt
  \begin{align*}
    x' \bigtriangleup y'
    = (x' \land y'') \lor (y' \land x'')
    = (x' \land y) \lor (y' \land x)
    = y \bigtriangleup x
    = x \bigtriangleup y,
  \end{align*}
  also ist auch die Komplementoperation wohldefiniert.

  Damit lässt sich nun auch die Wohldefiniertheit von $\land$ beweisen.
  Denn für $x \sim a$ und $y \sim b$ gilt mit dem bereits Bewiesenen
  \begin{displaymath}
    [x \land y]
    = [x'' \land y'']
    = [(x' \lor y')']
    = [x \lor y]',
  \end{displaymath}
  woraus schließlich $[x \land y] = [a \land b]$ folgt.
  Die auf dem Quotienten definierten Operationen sind also wohldefiniert.

  Setzt man \PIT voraus, so existiert ein maximales Ideal $K \subseteq X/I$.
  Betrachtet man nun die Menge
  \begin{displaymath}
    J := \{x \in X \mid [x] \in K\},
  \end{displaymath}
  so zeigt man nun, dass $J$ ein maximales Ideal in $X$ ist, welches $I$ umfasst.
  
  Dass $J$ ein Ideal ist, folgt sofort aus der Definition der Verknüpfungsoperationen auf $X/I$ und der Tatsache, dass $K$ ein Ideal ist.
  Für alle $x \in I$ gilt zudem $[x] = [0] \in X/I$. 
  Da $K$ als Ideal $[0]$ enthält, folgt $[x] \in K$, also $I \subseteq J$.
  
  Angenommen, $J$ sei nicht maximal. Sei $J' \subsetneq X$ ein $J$ umfassendes Ideal.
  Dann gilt 
  \begin{displaymath}
    K \subseteq J'/I := \{[x] \mid x \in J\},
  \end{displaymath}
  denn für $[x] \in K$ ist $x \in J \subseteq J'$ und damit $[x] \in J'/I$.
  Aufgrund der Maximalität von $K$ folgt sofort $K := J'/I$ und damit $J' = J$.

  Also ist $J$ ein maximales $I$ umfassendes Ideal.
\end{proof}

In \textsc{Boole}schen Algebren existiert das zum Ideal \textit{duale} Konzept des Filters.

\begin{defn}
  Sei $B$ eine \textsc{Boole}sche Algebra. Dann bezeichnet man eine nichtleere echte Teilmenge $F \subsetneq B$ als \textit{Filter} in $B$, falls folgende Bedingungen gelten:
  \begin{enumerate}[(1)]
    \item Aus $u \in F$ und $v \in F$ folgt $u \land v \in F$.
    \item Für $u \in F$, $x \in B$ und $u \leq x$ folgt  $x \in F$.
  \end{enumerate}
  Ein Filter $F$ in $B$ heißt \textit{Ultrafilter}, falls zusätzlich gilt:
  \begin{enumerate}[(3)]
    \item Für alle $x \in B$ ist entweder $x \in F$ oder $x' \in F$.
  \end{enumerate}
\end{defn}

Es sollen als nächstes Ultrafilter in \textsc{Boole}schen Algebren charakterisiert und ihr Verhalten im Hinblick auf \textsc{Boolesche} Homomorphismen beschrieben werden.

\begin{prop}
  \label{prop:ultrafilterchar}
  Es sei $F \subsetneq B$ ein Filter in der \textsc{Boole}schen Algebra $B$. Dann sind die folgenden Aussagen äquivalent
  \begin{enumerate}[(1)]
    \item Der Filter $F$ ist ein Ultrafilter.
    \item Der Filter $F$ ist bezüglich mengentheoretischer Inklusion maximal.
    \item Für alle $x,y \in B$ folgt aus $x \lor y \in F$, dass $x \in F$ oder $y \in F$. Einen solchen Filter bezeichnet man auch als \emph{Primfilter}.
  \end{enumerate}
\end{prop}

\begin{proof}
  (1) $\Rightarrow$ (2):
  Angenommen, der Ultrafilter $F$ sei nicht maximal. 
  So existiert ein Filter $F' \supsetneq F$, also insbesondere ein $x \in F' \setminus F$.
  Dann ist jedoch nach Voraussetzung $x' \in F$ und aufgrund der Inklusionsbeziehung $x' \in F'$.
  Dies widerspricht jedoch der Filtereigenschaft, da $0 = x' \cap x \in F'$, also $F' = B$ gilt.

  (2) $\Rightarrow$ (3):
  Angenommen, es existieren $x,y \in B$ mit $x \lor y \in F$ aber $x \not\in F$ und $y \not\in F$.
  Man kann ohne Beschränkung der Allgemeinheit $x \neq 0$ annehmen.
  Dann gilt $x \leq x \lor y$ und damit wäre $F \cup \{x\}$ ein echt größerer Filter im Widerspruch zur Voraussetzung.

  (3) $\Rightarrow$ (1):
  Für alle $x \in B$ gilt $x \cap x' = 1$.
  Zudem gilt für alle Filter $F$, dass $1 \in F$.
  Nach Voraussetzung gilt dann für jeden Primfilter $x \in F$ oder $x' \in F$.
\end{proof}

\begin{lem}
  \label{lem:boolehomfilters}
  Es sei $f \colon A \to B$ ein Homomorphismus zwischen den \textsc{Boole}schen Algebren $A$ und $B$.
  Dann gelten folgende Aussagen:
  \begin{enumerate}[(1)]
    \item Ist $F \subsetneq B$ ein Filter auf $B$, dann ist auch $f^{-1}(F)$ ein Filter auf $A$.
    \item Ist $F$ sogar ein Ultrafilter, dann auch $f^{-1}$.
  \end{enumerate}
\end{lem}

\begin{proof}
  (1):
  Sind $u,v \in f^{-1}(F)$ so sind $f(u),f(v) \in F$ und es gilt $f(u \land v) = f(u) \land f(v) \in F$, da $F$ ein Filter ist.
  Damit folgt sofort $u \land v \in f^{-1}(F)$.
  Man beachte, dass Homomorphismen aufgrund der Verträglichkeit der algebraischen Struktur mit Verbandsstruktur \textsc{Boole}scher Algebren isotone Abbildungen sind.
  Für $x,y \in A$ mit $x \leq y$ gilt nämlich $x = x \land y$, also auch $f(x) = f(x\land y) = f(x) \land f(y)$ also auch $f(x) \leq f(y)$.
  Daraus folgt für $u \in f^{-1}$ und $x \in A$, dass auch $x \in f^{-1}(F)$ gilt.

  (2):
  Sei $x \in A$ gegeben.
  Dann gilt $f(x) \in F$ oder $f(x') = f(x)' \in F$.
  Dies impliziert $x \in f^{-1}(F)$ oder $x' \in f^{-1}(F)$.
\end{proof}



Es lässt sich somit eine zu \PIT duale und damit äquivalente Formulierung für Filter finden:
  \begin{addmargin}[2em]{2em}% 1em left, 2em right
    \textit{Jede \textsc{Boole}sche Algebra besitzt einen maximalen Filter.}
  \end{addmargin}


Bevor der für das Folgekapitel zentrale Satz vorgestellt wird, soll eine in dieser Arbeit mehrfach verwendete Aussage bewiesen werden.

\begin{prop}
  \label{prop:cartesianclosed}
  Sei $(X_i,\tau_i)_{i \in I}$ eine Familie topologischer Räume. 
  Zudem sei $(A_i)_{i \in I}$ mit  $A_i \subseteq X_i$ eine Familie abgeschlossener Teilmengen.
  Dann ist 
  \begin{displaymath}
    \mathcal{A} := \prod_{i \in I} A_i \subseteq \prod_{i \in I} X_i
  \end{displaymath}
  abgeschlossen bezüglich der Produkttopologie.
  %STÄRKERE AUSSAGE VIELLECHT BEWEISEN?
\end{prop}

\begin{proof}
  Es soll gezeigt werden, dass $\mathcal{A}$ ein offenes Komplement besitzt.
  Sei dazu $x$ ein Punkt, welcher nicht in $\mathcal{A}$ liegt.
  Dann existiert ein $i_0 \in I$, sodass $x_{i_0} \not\in A_{i_0}$.
  Damit ist $O_{i_0} := X_{i_0} \setminus A_{i_0}$ eine offene Umgebung von $x$ in $X_{i_0}$.
  Folglich ist
  \begin{displaymath}
    U:= \prod_{i \in I} O_i
  \end{displaymath}
  mit $O_i = X_i$ für $i \not= i_0$ eine offene Umgebung von $x$ in $\prod_{i \in I} X_i$.
  Nach Konstruktion gilt $U \cap \mathcal{A} = \emptyset$.
  Also ist $x$ ein innerer Punkt und das Komplement von $\mathcal{A}$ somit offen.
  Folglich ist $\mathcal{A}$ abgeschlossen.
\end{proof}

Nun zu dem Satz, welcher \PIT mit unterschiedlichen Kompaktheitsaussagen über topologische Produkträume in Verbindung setzt.

\newpage
\begin{thm}
  \label{thm:pitequivalence}
  Die folgenden Aussagen sind äquivalent:
  \begin{enumerate}[(1)]
    \item \PIT.
    \item \UFT.
    \item \textsc{Tychonoff}-Satz für \textsc{Hausdorff}-Räume: 
      \begin{addmargin}[2em]{2em}% 1em left, 2em right
        Produkte kompakter \textsc{Hausdorff}-Räume sind kompakt.
      \end{addmargin}
    \item \textsc{Hilbert}-Würfel $[0,1]^I$ sind kompakt.
    \item \textsc{Kantor}-Würfel $\mathbf{2}^I$ sind kompakt.
    \item \textsc{Tychonoff}-Satz für endliche diskrete Räume: 
      \begin{addmargin}[2em]{2em}% 1em left, 2em right
        Produkte endlicher diskreter Räume sind kompakt.
      \end{addmargin}
  \end{enumerate}
\end{thm}

\begin{proof}
  (1) $\Rightarrow$ (2): 
  Interpretiert man \PIT im Sinne von Filtern, so folgt dies sofort, da sich jede Potenzmengenalgebra als \textsc{Boole}sche Algebra auffassen lässt.

  (2) $\Rightarrow$ (3):
  Nach Satz \ref{thm:uftcompact} folgt aus \UFT die Übereinstimmung von Kompaktheit mit Ultrafilterkompaktheit.
  Sei nun $(X_i)_{i \in I}$ eine Familie kompakter \textsc{Hausdorff}\hyp{}Räume mit Produktraum $X := \prod_{i \in I} X_i$.
  Sei $\psi$ ein Ultrafilter auf diesem Produktraum.
  Dann ist für alle $i \in I$ der Bildfilter $\psi_i = \pi_i(\psi)$ wiederum ein Ultrafilter, wobei die kanonischen Projektionen mit $\pi_i$ bezeichnet seien.
  Nach Voraussetzung ist jedoch jeder der Faktoren $X_i$ kompakt, es konvergiert also jede Projektion $\psi$ des Ultrafilters gegen ein $x_i \in X_i$.
  Also konvergiert, da die Produkttopologie initial bezüglich $(\pi_i)_{i \in I}$ ist, der Filter $\psi$ gegen $(x_i)_{i \in I}$.
  Damit ist $X$ ultrafilterkompakt und folglich auch kompakt.

  (3) $\Rightarrow$ (4):
  Dies ist eine Spezialisierung von (3), da das abgeschlossene Intervall $[0,1]$ mit euklidischer Topologie insbesondere ein \textsc{Hausdorff}\hyp{}Raum ist.

  (4) $\Rightarrow$ (5):
  Da in \textsc{Hausdorff}\hyp{}Räumen insbesondere das T1\hyp{}Axiom gilt und damit Einpunktmengen abgeschlossen sind, liefert Proposition \ref{prop:cartesianclosed}, dass man \textsc{Kantor}\hyp{}Würfel als abgeschlossene Teilräume von \textsc{Hilbert}\hyp{}Würfeln auffassen kann. 
  Als solche sind sie aber auch kompakt.

  (5) $\Rightarrow$ (6):
  Sei $(X_i)_{i \in I}$ eine Familie endlicher diskreter Räume.
  Es bezeichne
  \begin{displaymath}
    f_i \colon X_i \to \mathbf{2}^{C(X_i,\mathbf{2})}
  \end{displaymath}
  die kanonische Einbettung, die jedem  $y \in X_i$ die entsprechende Evaluationsabbildung
  \begin{displaymath}
    \omega_i(y,\cdot) \colon C(X_i,\mathbf{2}) \to \mathbf{2}, \quad \omega_i(y,g) := g(y)
  \end{displaymath}
  zuordnet.
  Sei $X := \prod_{i \in I} X_i$.
  Dann sind alle $f_i$  und
  \begin{displaymath}
    f(x) := (f_i(x_i))_{i \in I} \colon X \to \prod_{i \in I} \mathbf{2}^{C(X_i,\mathbf{2})}
  \end{displaymath}
  abgeschlossene Einbettungen.
  Denn, da alle $X_i$ nach Voraussetzung diskrete Räume sind, folgt die Stetigkeit der $f_i$ automatisch.
  Bezüglich der Produkttopologie auf $X$ ist nach Konstruktion dann auch $f$ stetig.
  Zudem sind alle $f_i$ abgeschlossen, da jede abgeschlossene Teilmenge $A \subseteq X_i$ insbesondere endlich und damit kompakt ist, andererseits jedoch das Bild einer kompakten Menge unter einer stetigen Abbildung wiederum kompakt und als kompakte Teilmenge eines \textsc{Hausdorff}\hyp{}Raumes abgeschlossen ist.  
  Mit demselben Argument sieht man auch die Abgeschlossenheit von $f$ ein.
  Ist nämlich $A \subseteq X$ eine abgeschlossene Menge, so besteht sie nach Konstruktion aus einem kartesischen Produkt endlicher, also kompakter Mengen.

  Nach dem vorangehenden Argument ist das Bild jedes einzelnen Faktors des kartesischen Produkts eine kompakte, also auch abgeschlossene Menge.
  Dann ist aber auch $f(A)$ als kartesisches Produkt abgeschlossener Mengen nach Proposition \ref{prop:cartesianclosed}.
  Die Injektivität der $f_i$ ist klar, denn man kann, falls $y,z \in X_i$ mit $y \neq z$, eine Abbildung $c \in C(X_i,\mathbf{2})$ definieren durch $c(y) = 1$ und $c(z) = 0$ und $c(a)$ beliebig für alle $a \in X_i \setminus \{y,z\}$.
  Damit ist auch klar, dass $f$ injektiv ist.
   
  Es gilt 
  \begin{displaymath}
    \prod_{i \in I} \mathbf{2}^{C(X_i,\mathbf{2})} 
    \simeq \mathbf{2}^{\{(f,i) \mid f \in C(X_i,\mathbf{2}), i \in I\}}
    = \mathbf{2}^{\bigcup_{i \in I} (X_i,\mathbf{2}) \times \{i\}},
  \end{displaymath}
  da das Produkt $\mathbf{2}^{C(X_i,\mathbf{2})}$ bis auf Homöomorphie bereits durch die Kardinalität $|C( X_i, \mathbf{2} )|$ vollständig bestimmt ist.
  Daraus folgt die Kompaktheit von $X$, weil man $X$ als abgeschlossenen Teilraum eines nach (5) kompakten Raumes auffassen kann.

  (6) $\Rightarrow$ (1):
  Sei $B$ eine \textsc{Boole}sche Algebra.
  Man betrachte die Menge $\mathcal{A}$ aller endlichen \textsc{Boole}schen Unteralgebren auf $B$.
  Für alle $A \in \mathcal{A}$ ist die Menge $X_A$ aller \textsc{Boole}schen Homomorphismen von $A$ nach $\mathbf{2}$ nichtleer, da jede endliche \textsc{Boole}sche Algebra einen maximalen Filter besitzt.
  Ist nämlich $A$ eine endliche \textsc{Boole}sche Algebra und $x \neq 0$.
  Da $A$ nur endlich viele Elemente enthält kann man annehmen, dass aus $a \leq x$ bereits $a = 0$ folgt.
  Betrachtet man nun die Menge $\varphi := \{x \lor y \mid y \in A\}$, so ist $\varphi$ nach Konstruktion ein maximaler Filter.
  Es lässt sich nun ein \textsc{Boole}scher Homomorphismus $f \colon A \to \mathbf{2}$ angeben, indem man 
  \begin{displaymath}
    f(x) := 
    \begin{cases}
      1, \text{ falls } x \in \varphi \\
      0, \text{ falls } x \not\in \varphi
    \end{cases}
  \end{displaymath}
  setzt.
  Man prüft unter Verwendung von Proposition \ref{prop:ultrafilterchar} sofort nach, dass $f$ ein Homomorphismus ist, wenn man $\mathbf{2}$ in naheliegender Weise als \textsc{Boole}sche Algebra betrachtet.

  Man fasse $X_A$ nun auf als diskreten topologischen Raum.
  Dann ist der Produktraum $X = \prod_{A \in \mathcal{A}} X_A$ nach (6) kompakt.
  Zudem ist $X$ nichtleer.
  Dazu betrachte man die Familie diskreter topologischer Räume $(Y_A)_{A \in \mathcal{A}}$ mit $Y_A := X_A \cup \{\infty\}$.
  Nach Voraussetzung ist auch $Y := \prod_{A \in \mathcal{A}} Y_A$ kompakt.
  Für alle $A \in \mathcal{A}$ ist das Urbild $U_A = \pi_A^{-1}(X_A)$ unter der kanonischen Projektion $\pi_A$ eine nichtleere, abgeschlossene Teilmenge von $Y$.
  Es ist nämlich $U_A = \prod_{i \in \mathcal{A}} Z_i$ mit $Z_A = X_A$ und $Z_i = Y_i$ für alle $i \in \mathcal{A} \setminus \{A\}$.
  In $U_A$ ist daher insbesondere eine Abbildung $f$ mit $f(i) = \infty$ für alle $i \in \mathcal{A} \setminus \{A\}$ und $f(A) \in X_A$ enthalten, da $X_A$ nach Voraussetzung nichtleer ist.
  Die Familie $\mathfrak{U} := \{U_A \mid A \in \mathcal{A}\}$ besitzt die endliche Durchschnittseigenschaft, wie man mit demselben Argument zeigt.
  Da $Y$ kompakt ist, folgt daraus $\bigcap_{A \in \mathcal{A}} U_A \neq \emptyset$.
  Zudem gilt $X = \bigcap_{A \in \mathcal{A}} U_A$, also ist auch $X$ nichtleer.

  Für alle Paare $(A,B)$ mit $A,B \in \mathcal{A}$ und $A \subseteq B$ ist die Menge
  \begin{displaymath}
    M(A,B) := \{ (x_M)_{M \in \mathcal{A}} \in X \mid x_A \text{ ist die Einschränkung } {x_B}_{|A} \}
  \end{displaymath}
  abgeschlossen in $X$. 
  Jedes Menge ist nämlich von der Form $M(A,B) = \prod_{i \in \mathcal{A}} Z_i$ mit $Z_i = X_i$ für alle $i \in \mathcal{A} \setminus \{A\}$ und $Z_A = r(X_B)$, wobei $r \colon X_B \to X_A$ die Restriktionsabbildung $r(x_B) := {x_B}_{|A}$ bezeichne.
  Aufgrund der vorausgesetzten Diskretheit von $X_B$ ist $r$ stetig, und aufgrund der Diskretheit von $X_A$ sogar abgeschlossen.
  Daher ist auch $M(A,B)$ nach Proposition \ref{prop:cartesianclosed} eine abgeschlossene Menge in $X$.

  Zudem besitzt die Menge aller $M(A,B)$ die endliche Durchschnittseigenschaft, also existiert aufgrund der Kompaktheit auch ein Element $x := (x_M)_{M \in \mathcal{A}}$ im Durchschnitt aller $M(A,B)$.
  Es bezeichne im Folgenden $B(a)$ die von $a \in B$ erzeugte Unteralgebra von $B$.
  Dann ist die durch $f(a) = x_{B(a)}(a)$ bestimmte Abbildung von $B$ nach $\mathbf{2}$ ein \textsc{Boole}scher Homomorphismus.
  Sind nämlich $a,b \in B$ so gilt zunächst 
  \begin{displaymath}
    x_{B(a \land b)}(a) = x_{B(a)}(a),
  \end{displaymath}
  da nach Voraussetzung $x$ im Schnitt aller $M(A,B)$, insbesondere also auch in der Menge $M(B(a),B(a \land b))$ enthalten ist, da $B(a) \subseteq B(a \land b)$.
  Damit folgt
  \begin{displaymath}
    f(a \land b) 
    = x_{B(a \land b)}(a \land b)
    = x_{B(a \land b)}(a) \land x_{B(a \land b)}(b)
    = x_{B(a)}(a) \land x_{B(b)}(b)
    = f(a) \land f(b).
  \end{displaymath}
  Analog zeigt man $f(a \lor b) = f(a) \lor f(b)$.
  Es gilt $B(a') = B(a)$ da \textsc{Boole}sche Unteralgebren unter anderem abgeschlossen unter Komplementbildung sind.
  Damit folgt sofort $f(a') = f(a)'$.
  Dass $f(1) = 1$ und $f(0) = 0$ folgt daraus, dass jede Komponente von $x$ ein \textsc{Boole}scher Homomorphismus ist.
  Damit ist dann $f^{-1}(\{1\})$ nach Lemma \ref{lem:boolehomfilters} ein Ultrafilter auf $B$.
  Die Dualität von Idealen und Filtern in \textsc{Boole}schen Algebren liefert die Behauptung.
\end{proof}


