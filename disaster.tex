\section{Der Áscoli-Satz und das Auswahlaxiom}

Dieser Abschnitt ist eine ausgestaltete Darstellung des Kapitels \textit{Disasters in Topology III: Function Spaces (The Ascoli Theorem)} aus dem Buch von Horst Herrlich.

\begin{defn}
  Es seien $(X,d_X)$ und $(Y,d_Y)$ metrische Räume. Eine Abbildung $f \colon X \to Y$ heißt \textit{gleichmäßig stetig}, wenn gilt:
  \begin{addmargin}[2em]{2em}% 1em left, 2em right
  Zu jedem $\varepsilon > 0$ existiert ein $\delta > 0$, sodass \\
  für alle $x,x' \in X$ mit $d_X(x,x') < \delta$ gilt, dass $d_Y(f(x),f(x')) < \varepsilon$.
  \end{addmargin}
\end{defn}

\begin{defn}
  Es sei $(X,\tau)$ ein topologischer Raum und $(Y,d)$ ein metrischer Raum. Eine Teilmenge $F \subseteq Y^X$ heißt \textit{gleichgradig stetig}, wenn gilt:
  \begin{addmargin}[2em]{2em}% 1em left, 2em right
    Für alle $x \in X$ und $\varepsilon > 0$ existiert ein $V \in \dot{x} \cap \tau$, sodass \\
    für alle $f \in F$ gilt, dass $f[V] \subseteq U_\varepsilon(f(x))$.
  \end{addmargin}
\end{defn}

Für zwei topologische Räume $(X,\tau)$ und $(Y,\sigma)$ bezeichne im Folgenden $C_{co}(X,Y)$ den topologischen Funktionenraum $(C(X,Y),\tau_{co})$ der stetigen Funktionen von $X$ nach $Y$ ausgestattet mit der kompakt-offenen Topologie.

Áscoli-Sätze gelten bezeichnend für Aussagen, die es gestatten Kompaktheit von Teilmengen von Funktionenräumen bezüglich starker Topologien aus der Kompaktheit bezüglich einer schwächeren Topologie unter zusätzlichen Annahmen zu folgern. Dabei existieren unterschiedliche Formulierungen.

Eine topologische Version des Áscoli Satzes lautet:
\begin{defn}[Topologischer Áscoli Satz]
  Sei $(X,\tau)$ ein lokalkompakter Hausdorff-Raum und $(Y,d)$ ein metrischer Raum.
  Für jeden Teilraum $(F,\tau_{co})$ von $C_{co}(X,Y)$ sind äquivalent:
  \begin{enumerate}
    \item[(a)] $F$ ist kompakt.
    \item[(b)] \begin{enumerate}
        \item[($\alpha$)] Für alle $x \in X$ ist die Menge $F(x) = \{f(x) \mid f \in F \}$ kompakt in $Y$.
        \item[($\beta$)]  $F$ ist abgeschlossen in $Y$ bezüglich der Topologie der punktweisen Konvergenz $\tau_p$.
        \item[($\gamma$)] $F$ ist gleichgradig stetig auf $X$.
      \end{enumerate}
  \end{enumerate}
\end{defn}

Es soll nun gezeigt werden, dass die Gültigkeit dieses Satzes äquivalent zum Booleschen Primideal ist.

\begin{thm}
  \label{thm:ascoliPIT}
  Äquivalent sind
  \begin{enumerate}
    \item Der topologische Áscoli-Satz.
    \item Der Boolesche Primidealsatz.
  \end{enumerate}
\end{thm}

\begin{proof}
  $(1) \Rightarrow (2)$: Sei $X$ eine Menge und $\mathbf{2}$ die zweielementige Menge $\{0,1\}$ und seien beide mit der diskreten Topologie ausgestattet. 
  Die diskrete Topologie lässt sich auch als die von der \textit{diskreten Metrik} induzierte Topologie auffassen, somit kann $\mathbf{2}$ auch als metrischer Raum betrachtet werden. 

  Es sei $F := C(X, \mathbf{2})$.
  Dann gilt $C(X, \mathbf{2}) = \mathbf{2}^X$, da $X$ nach Voraussetzung diskret ist.
  Es folgt, dass $\mathbf{2} = F(x)$ für alle $x \in X$ als endlicher Raum kompakt ist.
  Nach Definition ist $\mathbf{2}^X$ als Gesamtraum abgeschlossen in jeder Topologie, also insbesondere auch in der punktweisen Topologie $\tau_p$.
  Zuletzt ist $F$ auch gleichgradig stetig, da $X$ diskret und somit Einpunktmengen offen sind, d.h. für alle $x \in X$ und $\varepsilon > 0$ wähle man $\{x\}$ als offene Umgebung in $X$. Dann gilt für alle $f \in F$, aufgrund der Definitheit der Metrik für  $d(f(x),f(x)) = 0 \leq \varepsilon$.
  Damit sind die Bedingungen (b) des Áscoli-Satzes erfüllt und es folgt, dass $\mathbf{2}^X$ kompakt ist. Dies ist äquivalent zum Booleschen Primidealsatz.

  $(2) \Rightarrow (1)$: Es seien $(X,\tau)$ ein lokalkompakter Hausdorff-Raum und $(Y,d)$ ein metrischer Raum.
  Des Weiteren sei $F$ ein Teilraum vom $C_{co}(X,Y)$.

  $(a) \Rightarrow (b, \alpha)$: 
  Nach Voraussetzung ist $F$ ein bezüglich der kompakt-offenen Topologie $\tau_{co}$ kompakter Teilraum von $C(X,Y)$.
  Somit ist er insbesondere bezüglich der Spur der schwächeren Topologie der punktweisen Konvergenz $\tau_p$ ein kompakter Teilraum des Hausdorff-Raumes $Y^X$.
  Die Topologie der punktweisen Konvergenz lässt sich als Initialtopologie bezüglich der kanonischen Projektionen $p_x : Y^X \to Y, f \mapsto f(x)$ beschreiben.
  Als stetiges Bild eines Kompaktums ist somit auch $p_x(F) = F(x)$ kompakt in $(Y,\tau_d)$.

  $(a) \Rightarrow (b, \beta)$:  
  Als kompakter Teilraum eines Hausdorff-Raumes ist $F$ bezüglich $\tau_p$ auch abgeschlossen in $Y^X$.

  $(a) \Rightarrow (b, \gamma)$:
  Es soll gezeigt werden, dass $F$ gleichgradig stetig ist.
  Sei dazu $x \in X$ und $\varepsilon > 0$.
  Für jedes $f \in F$ ist die Menge
  \begin{displaymath}
    B_f := B_{\frac{\varepsilon}{2}} = \left\{ y \in Y \mid d(f(x),y) < \tfrac{\varepsilon}{2}\right\} 
  \end{displaymath}
  offen bezüglich der von der Metrik $d$ auf $Y$ induzierten Topologie.
  Da $F \subseteq C(X,Y)$, ist das Urbild $f^{-1}(B_f)$ ebenfalls offen bezüglich $\tau$.
  Nach Voraussetzung ist $X$ lokalkompakt.
  Daher existiert eine kompakte Umgebung $K_f$ von $x$ mit $K_f \subseteq f^{-1}(B_f)$.
  Damit gilt $f(K_f) \subseteq B_f$ und es definiert
  \begin{displaymath}
    U_f := F \cap (K_f, B_f) = \left\{g \in F \mid g(K_f) \subseteq B_f \right\} \tag{$\ast$}
  \end{displaymath}
  eine bezüglich der Teilraumtopologie auf $F$ offene Umgebung von $f$.
  Man betrachte nun die Evaluationsabbildung $\omega: X \times F \to Y, (y,g) \mapsto g(y)$.
  Unter Berücksichtigung von ($\ast$) folgt $\omega(K_f \times U_f) \subseteq B_f$.
  Es definiere $\mathcal{C}$ die Menge aller Tripel $(f,K,U)$, wobei $f \in F$, $K$ eine Umgebung von $x$ in $X$ und $U$ eine offene Umgebung von $f$ in $F$ mit $\omega(K \times U) \subseteq B_f$ sei.
  Dann liefert
  \begin{displaymath}
    \mathfrak{U} = \{ U \subseteq F \mid \text{es existieren $f \in F$, $K \subseteq X$ mit $(f,K,U) \in \mathcal C$} \}
  \end{displaymath}
  eine offene Überdeckung von $F$, da für alle $f \in F$ nach Konstruktion $(f,K_f,U_f) \in \mathfrak{U}$ gilt.
  Nach Voraussetzung ist $F$ kompakt, daher existiert eine endliche Teilüberdeckung durch $U_1,\dots,U_n \in \mathfrak{U}$.
  Für alle $i = 1,\dots,n$ wähle man nun $f_i \in F$ und $K_i \subseteq X$ mit $(f_i, K_i, U_i) \in \mathcal{C}$.
  Des Weiteren definiere man $U := \bigcap_{i = 1}^n K_i$.
  Es ist $U$ eine (nicht notwendig offene) Umgebung von $x \in X$.
  Nun soll gezeigt werden, dass für diese Wahl von $U$ die Voraussetzung der gleichgradigen Stetigkeit erfüllt ist, also dass für alle $f \in F$ folgt, dass $f(U) \subseteq U_\varepsilon(f(x))$ gilt.
  Für jedes $f \in F$ existiert aufgrund der Überdeckungseigenschaft ein $i = 1,\dots,n$ mit $f \in U_i$.
  Sei des Weiteren $y \in U$ gegeben.
  Dann gilt
  \begin{displaymath}
    f(y) = \omega(y,f) \in \omega(U \times U_i) \subseteq \omega(K_i \times U_i) \subseteq B_{f_i},
  \end{displaymath}
  also $d(f_i(x),f(y)) < \tfrac{\varepsilon}{2}$.
  Insbesondere gilt für $x \in U$, da nach Voraussetzung $x \in K_i$ für alle $i = 1,\dots,n$, dass $d(f_i(x),f(x)) < \tfrac{\varepsilon}{2}$.
  Unter Verwendung der Dreiecksungleichung folgt letztlich
  \begin{displaymath}
    d(f(x),f(y)) \leq d(f(x),f_i(x)) + d(f_i(x),f(y)) < \frac{\varepsilon}{2} + \frac{\varepsilon}{2} = \varepsilon,
  \end{displaymath}
  und damit $f(U) \subseteq U_\varepsilon(f(x))$.
  Folglich ist $F$ gleichgradig stetig.

  $(b) \Rightarrow (a)$
  Nach Voraussetzung ist $F(x)$ als Teilraum von $Y^X$ kompakt und besitzt zudem die Hausdorff-Eigenschaft.
  Der Tychonoff-Satz für Kompaktheit besagt, dass dann auch $\prod_{x \in X} F(x)$ kompakt ist.
  Die Voraussetzung, dass $F$ abgeschlossen in $Y^X$ bezüglich der Produkttopologie $\tau_p$ ist, liefert, dass $F$ auch abgeschlossen im Teilraum $\prod_{x \in X} F(x)$ ist.
  Als abgeschlossene Teilmenge eines Kompaktums ist somit $F$ auch kompakt bezüglich der Topologie der punktweisen Konvergenz.
  Es fehlt zu zeigen, dass $F$ ebenfalls bezüglich der stärkeren, kompakt-offenen Topologie $\tau_{co}$ kompakt ist.
  Hierfür soll die Inklusion $\tau_{co} \subseteq \tau_p$ nachgewiesen werden.
  Dazu betrachte man ein Subbasiselement $V = (K,U) \cap F$ der kompakt-offenen Topologie auf dem Teilraum $F$ und ein $f \in V$.
  Es gilt nun zu zeigen, dass $V$ auch in $\tau_p$ offen ist.
  Da $f(K) \subseteq U$ und $U \in \sigma$ gilt, folgt für alle $x \in K$
  \begin{displaymath}
    r_x = \inf\{d(f(x),y) \mid y \in (Y \setminus U) \} > 0.
  \end{displaymath}
  Dann ist $U_x = \{ z \in X \mid d(f(x),f(z)) < \tfrac{r_x}{2}\}$ eine offene Umgebung von $x \in X$.
  Denn für alle $z \in U_x$ mit $\varepsilon := \tfrac{r_x}{2} - d(f(x),f(z)) > 0$ gilt, dass aufgrund der Stetigkeit von $f$ eine Umgebung $U' \in \tau$ von $z$ existiert, sodass $f(U') \subseteq U_\varepsilon(f(z))$ gilt.
  Daraus folgt für alle $u \in U'$ mit der Dreiecksungleichung
  \begin{align*}
    d(f(x),f(u)) &\leq d(f(x),f(z)) + d(f(z),f(u)) \\
                 &< \varepsilon + d(f(z),f(u)) = \frac{r_x}{2} - d(f(x),f(z)) + d(f(x),f(z)) =  \frac{r_x}{2}
  \end{align*}
  und damit schließlich $U' \subseteq U_x$, also die Offenheit von $U_x$ in $X$.
  Zudem bildet $\mathfrak{U} := \{U_x \mid x \in K\}$ eine offene Überdeckung von $K$. 
  Nach Voraussetzung ist $K$ kompakt, daher existieren $x_1,\dots,x_n \in K$, sodass $K \subseteq \bigcup_{i=1}^n U_{x_i}$.
  Es sei $r:= \min\{r_{x_1},\dots,r_{x_n}\}$. 
  Damit ergibt sich für alle $x \in K$ und alle $y \in (Y \setminus U)$ die Ungleichung $d(f(x),y) \geq \tfrac{r}{2}$.
  Ist $x \in K$ und $d(f(x),y) < \tfrac{r}{2}$ so folgt daraus $y \in U$.
  Da $F$ nach Voraussetzung gleichgradig stetig auf $X$ ist, existiert für alle $x \in X$ eine offene Umgebung $W$ von $x$ in $X$, sodass für alle $g \in F$ und alle $z \in W$ gilt, dass $d(g(x),g(z)) < \tfrac{r}{4}$.
  Betrachtet man nun die auf obige Weise erzeugte Menge $\mathcal{C}$ aller geordneten Paare $(x,W)$, so bildet
  \begin{displaymath}
    \mathfrak{W} := \{W \subseteq X \mid \text{Es existiert ein $x \in K$ mit $(x,W) \in \mathcal{C}$} \}
  \end{displaymath}
  eine offene Überdeckung von $K$.
  Aufgrund der Kompaktheit von $K$ lässt sich eine endliche Teilüberdeckung $W_1,\dots,W_m \in \mathfrak{W}$ von $K$ auswählen.
  Für alle $i = 1,\dots,m$ wähle man entsprechend der Definition $\mathfrak{W}$ ein $x_i$ mit $(x_i,W_i) \in \mathcal{C}$.
  Dann ist
  \begin{displaymath}
    B_f := \{ g \in F \mid d(f(x_i),g(x_i)) < \tfrac{r}{4} \text{ für } i = 1,\dots,m\}
  \end{displaymath}
  eine offene Umgebung von $f$ bezüglich $\tau_p$.
  Es soll nun gezeigt werden, dass $B_f \subseteq V$ gilt.
  Dann ist nämlich $V = \bigcup_{f \in V} B_f$ als Vereinigung offener Mengen offen bezüglich $\tau_p$ und die Behauptung $\tau_{co} \subseteq \tau_p$ folgt, da $V$ beliebig war.
  Sei dazu also $g \in B_f$.
  Für jedes $x \in K$ existiert ein $i \in \{1,\dots,m\}$ mit $x \in W_i$.
  Nach Konstruktion der $W_i$ gilt $d(g(x_i),g(x)) < \tfrac{r}{4}$.
  Da $g \in B_f$ gilt zudem die Ungleichung $d(f(x_i),g(x_i)) < \tfrac{r}{4}$.
  Daraus folgert man
  \begin{displaymath}
    d(f(x_i),g(x)) \leq d(f(x_i),g(x_i)) + d(g(x_i),g(x)) < \frac{r}{2},
  \end{displaymath}
  was wiederum $g(x) \in U$ und damit $g(K) \subseteq U$ impliziert. 
  Also ist $g \in (K,U)$, woraus $g \in V$ folgt.
  Damit ist der Beweis vollständig.
\end{proof}

Es soll nun untersucht werden, inwiefern der Áscoli Satz auch unter Verwendung anderer Kompaktheitsbegriffe seine Gültigkeit behält.

\begin{lem}
  \label{lem:openmap}
  Es seien $(X,\tau)$ und $(Y,\sigma)$ topologische Räume und es sei $\mathfrak{S}$ eine Subbasis für $\tau$.  
  Dann ist eine injektive Abbildung $f \colon X \to Y$ genau dann offen, wenn $f(\mathfrak{S}) \subseteq \sigma$.
\end{lem}

\begin{proof}
  Da $\mathfrak{S} \subseteq \tau$ folgt aus der Offenheit von $f$ sofort 
  \begin{displaymath}
    f(\mathfrak{S}) \subseteq f(\tau) \subseteq \sigma.
  \end{displaymath}
  Sei andererseits $f$ eine injektive Abbildung und $O \in \tau$ beliebig.
  So ist $O$ die Vereinigung endlicher Schnitte von Subbasislementen.
  Für zwei beliebige Mengen $O_1,O_2$ gilt nun 
  \begin{displaymath}
    f(O_1 \cap O_2) = f(O_1) \cap f(O_n)
  \end{displaymath}
  aufgrund der Injektivität von $f$, sowie 
  \begin{displaymath}
    f \left(\bigcup_{i \in I} O_i \right) = \bigcup_{i \in I} f(O_i)
  \end{displaymath}
  für eine beliebige Familie offener Mengen.
  Damit ist auch $f(O)$ offen und die Behauptung folgt.
\end{proof}

\begin{lem}
  \label{lem:filterclusterpoint}
  Es sei $(X,\tau)$ ein topologischer Raum, sodass die darin offen-abgeschlossenen Mengen einen Basis für die Topologie $\tau$ bilden.

  \begin{itemize}
    \item[(i)] Ist $\mathcal{F}$ ein Filter ohne Häufungspunkt, so gilt für die Menge $\mathfrak{A} \subseteq \mathcal{F}$ aller offen-abgeschlossenen Elemente 
     \begin{displaymath}
       \bigcap_{F \in \mathfrak{A}} F = \emptyset.
     \end{displaymath}
   \item[(ii)] Erfüllt $(X,\tau)$ zusätzlich das T1-Axiom, so existiert zu je zwei unterschiedlichen Punkten $x,y \in X$ ein $A_{(x,y)} \in \mathfrak{A}$ mit 
     \begin{displaymath}
       \{x,y\} \cap A_{(x,y)} = \{y\}.
     \end{displaymath}
  \end{itemize}
\end{lem}

\begin{proof}
  \begin{itemize}
    \item[(i)]
      Nach Voraussetzung besitzt $\mathcal{F}$ keinen Häufungspunkt.
      Dies bedeutet, dass
      \begin{displaymath}
        \bigcap_{F \in \mathcal{F}} \overline F = \emptyset.
      \end{displaymath}
      Für alle $x \in X$ gilt damit, dass ein $F \in \mathcal{F}$ existiert, sodass $x \not\in \overline{F}$.
      Das bedeutet, dass $X \setminus \overline{F}$ eine offene Umgebung von $x$ ist.
      Da die offen-abgeschlossenen Mengen nach Voraussetzung eine Basis für $\tau$ bilden, existiert eine offen-abgeschlossene Umgebung $U$ von $x$, mit $U \subseteq X \setminus \overline{F}$ und damit
      \begin{displaymath}
        U \cap F = \emptyset.
      \end{displaymath}
      Folglich ist $X \setminus U \supseteq F$ und wegen des Abschlusses gegen Obermengenbildung
      \begin{displaymath}
        X \setminus U \in \mathcal{F}.
      \end{displaymath}
      Es existiert also eine offen-abgeschlossene Menge in $\mathcal{F}$, die $x$ nicht enthält. 
      Dies impliziert
      \begin{displaymath}
        x \not\in \bigcap_{F \in \mathfrak{A}} F.
      \end{displaymath}
      Da $x$ beliebig gewählt war, folgt 
      \begin{displaymath}
       \bigcap_{F \in \mathfrak{A}} F = \emptyset
      \end{displaymath}
      und damit die Behauptung.
    \item[(ii)] Unter der Voraussetzung, dass $(X,\tau)$ das T1-Axiom erfüllt, lässt sich zur offenen Umgebung $X \setminus \overline{F}$ aus Teil (i) des Beweises eine offene Umgebung 
      \begin{displaymath}
        U_x \subseteq X \setminus \overline{F}
      \end{displaymath}
      von $x$ finden, die zusätzlich $y$ nicht enthält. 
      Da die offen-abgeschlossenen Mengen eine Basis von $\tau$ bilden, kann man zudem annehmen, dass $U_x$ offen-abgeschlossen ist.
      Man prüft nach, dass für
      \begin{displaymath}
        A_{(x,y)} := X \setminus U_x
      \end{displaymath}
      die in (ii) geforderten Eigenschaften erfüllt sind:
      Als Komplement einer offen-abgeschlossenen Menge, ist auch $A_{(x,y)}$ offen und abgeschlossen.
      Nach Konstruktion ist 
      \begin{displaymath}
        x \not\in A_{(x,y)} \text{ sowie } y \in A_{(x,y)}.
      \end{displaymath}
      Zudem gilt, wie schon in (i) gezeigt, $A_{(x,y)} \in \mathcal{F}$.
      Daraus ergibt sich die Behauptung. \qedhere
  \end{itemize}
\end{proof}

\begin{thm}
  Äquivalent sind:
  \begin{enumerate}
    \item Der Áscoli Satz bezüglich Ultrafilterkompaktheit.
    \item Der Boolesche Primidealsatz.
  \end{enumerate}
\end{thm}

\begin{proof}
  $(1)\Rightarrow(2) $: Es reicht aus zu zeigen, dass $(1)$ bereits impliziert, dass die Kantor-Würfel $\mathbf{2}^I$ im gewöhnlichen Sinne kompakt sind.
  Angenommen, es existiert eine Menge $I$, sodass $P:= \mathbf{2}^I$ nicht kompakt ist.
  Man betrachte $I$ als einen mit der diskreten Topologie ausgestatteten topologischen Raum und $\mathbf{2}$ als diskreten metrischen Raum.
  Dann existiert ein Filter $\mathcal{F}$ auf $P$ ohne Häufungspunkt.

  Der mit der Produkttopologie ausgestattete Raum $P$ besitzt eine Basis aus offen-ab\-ge\-schlossenen Mengen.
  Dies folgt daraus, dass bereits jedes Subbasiselement $S$ per Definition offen ist und, da $\mathbf{2}$ diskret ist, auch ein offenes Komplement besitzt.
  Damit ist $S$ aber als Komplement einer offenen Menge definitionsgemäß abgeschlossen.
  Insgesamt ist $S$ also offen-abgeschlossen.
  Da die Menge der offen-abgeschlossenen Mengen abgeschlossen unter endlichen Durchschnitten ist, ist auch jedes von der Subbasis erzeugte Basiselement eine offen-abgeschlossene Menge.

  Hiermit lässt sich nun Lemma \ref{lem:filterclusterpoint}(i) auf $\mathcal{F}$ anwenden und es folgt für die Menge $\mathfrak{A}$ aller offen-abgeschlossenen Elemente von $\mathcal{F}$
  \begin{displaymath}
    \bigcap_{F \in \mathfrak{A}} F = \emptyset \tag{$\ast$}.
  \end{displaymath}

  Es lässt sich nun für alle $A \in \mathfrak{A}$ eine Funktion
  \begin{displaymath}
    f_A \colon P \to \mathbf{2}, \quad
    f_A(x) := 
    \begin{cases}
      1 &\text{, falls } x \in A \\
      0 &\text{, falls } x \not\in A
    \end{cases}
  \end{displaymath}
  definieren.
  Da alle $A$ offen-abgeschlossen sind, folgt sofort die Stetigkeit von $f_A$ für alle $A \in \mathfrak{A}$.
  Somit induziert die Familie $(f_A)_{A \in \mathfrak{A}}$ eine Abbildung 
  \begin{displaymath}
    f \colon P \to \mathbf{2}^\mathfrak{A}, \quad
    x \mapsto (f_A(x))_{A \in \mathfrak{A}},
  \end{displaymath}
  welche bezüglich der Produkttopologie auf $P$ stetig ist.

  Es bezeichne $F := f[P]$ das Bild von $P$ unter der obigen Abbildung $f$.
  Es soll nun gezeigt werden, dass $f$ eine Einbettung ist, also ein Homöomorphismus auf $F$:
  Die Abbildung $f$ ist injektiv, denn für $x,y \in P$ mit $x \neq y$ gilt, da $P$ als Hausdorff-Raum insbesondere das T1-Axiom erfüllt mit $A_{(x,y)} \in \mathfrak{A}$ aus Lemma \ref{lem:filterclusterpoint}(ii) 
  \begin{displaymath}
    f_{A_{(x,y)}}(x) = 0 \neq 1 = f_{A_{(x,y)}}(y).
  \end{displaymath}

  Zudem ist $f$ offen bezüglich der Spurtopologie auf $F$.
  Um Lemma \ref{lem:openmap} anwenden zu können sei nun $O = \pi_{i_0}^{-1}[\{j\}]$ mit $i_0 \in I$ und $j \in \{0,1\}$ eine beliebige offene Subbasismenge der Produkttopologie auf $P$,  wobei hier und im Folgenden die kanonischen Projektionen des Produktraumes mit $\pi_i$ bezeichnet seien.
  Man erkennt $O$ als eine offen-abgeschlossene Menge. 

  Gilt $O \in \mathcal{F}$, also insbesondere $O \in \mathfrak{A}$, so ist für alle $x \in O$
  \begin{displaymath}
    \pi_O(f(x)) = f_O(x) = 1,
  \end{displaymath}
  Daraus folgt 
  \begin{displaymath}
    f[O] = \pi_O^{-1}[\{1\}] \cap f[P]
  \end{displaymath}
  und folglich ist $f[O]$ offen bezüglich der Teilraumtopologie auf $f[P]$.

  Ist andernfalls $O \not\in \mathcal{F}$, so gilt für alle $F \in \mathcal{F}$ entweder
  \begin{displaymath}
    F \cap O \neq \emptyset
  \end{displaymath}
  oder $P \setminus O \in \mathcal{F}$.
  Im zweiten Fall ist damit insbesondere $P \setminus O \in \mathfrak{A}$ und analog zum ersten Teil folgt für alle $x \in O$
  \begin{displaymath}
    \pi_{P \setminus O}(f(x)) = 0
  \end{displaymath}
  und damit 
  \begin{displaymath}
    f[O] = \pi_{P \setminus O}^{-1}[\{0\}] \cap f[P].
  \end{displaymath}
  Im anderen Fall gilt zudem für alle $F \in \mathcal{F}$
  \begin{displaymath}
    F \cap P \setminus O \neq \emptyset
  \end{displaymath}
  und damit 
  \begin{displaymath}
    F[O] = F[P].
  \end{displaymath}

  Damit ist gezeigt, dass $f$ eine offene Abbildung auf ihr Bild ist.
  Folglich ist $f$ ein Homöomorphismus auf $f[P]$.

  Jeder endliche diskrete Raum ist ultrafilterkompakt, da dort die einzigen Ultrafilter, die Einpunktfilter sind, welche immer konvergieren.
  Es ist daher $P$ als Produkt ultrafilterkompakter Räume wiederum ultrafilterkompakt.
  Damit ist $F$ als homöomorphes Bild eines ultrafilterkompakten Raumes ebenso ultrafilterkompakt.

  Wendet man nun den Áscoli-Satz bezüglich Ultrafilterkompaktheit auf $F$ an, wobei $\mathfrak{A}$ als diskreter topologischer Raum und $\mathbf{2}$ als diskreter metrischer Raum betrachtet wird, so ist die Aussage $(b, \beta)$ nicht erfüllt:
  Es ist nämlich $p := (1)_{A \in \mathfrak{A}} \in \mathbf{2}^\mathfrak{A}$ im Abschluss von $F$ in $\mathbf{2}^\mathfrak{A}$ enthalten.
  Ist nämlich $U_p \subseteq \mathbf{2}^\mathfrak{A}$ eine beliebige Umgebung von $p$, dann existieren Subbasiselemente $B_{A_1},\dots,B_{A_n}$ mit
  \begin{displaymath}
    p \in \bigcap_{i=1}^n B_{A_i} \subseteq U_p
  \end{displaymath}
  und es ist
  \begin{displaymath}
    \bigcap_{i=1}^n B_{A_i} \cap f[P] \neq \emptyset.
  \end{displaymath}
  Denn für 
  \begin{displaymath}
    x \in \bigcap_{i=1}^n A_i \in \mathcal{F}
  \end{displaymath}
  gilt $f_{A_i}(x) = 1$, da offen-abgeschlossene Mengen sowie Filter eine endliche Durchschnittseigenschaft besitzen.
  
  Andererseits ist $p$ jedoch nicht in $F$ enthalten. 
  Gälte nämlich $p \in F$, so existiert ein $x \in P$ mit $f_A(x) = 1$ für alle $A \in \mathfrak{A}$, was gleichbedeutend ist mit 
  $x \in \cup_{A \in \mathfrak{A}}$.
  Dies steht jedoch im Widerspruch dazu, dass der Schnitt über alle in $\mathcal{F}$ enthaltenen offen-abgeschlossenen Mengen leer ist.
  Also gilt $F \neq \overline{F}$, ein Widerspruch zum Áscoli-Satz bezüglich Ultrafilterkompaktheit.
  Die Annahme, dass $\mathbf{2}^I$ nicht kompakt war, muss also verworfen werden.

  $(2)\Rightarrow(1)$: Der Boolesche Primidealsatz impliziert die Übereinstimmung des gewöhnlichen Kompaktheitsbegriffes mit dem Begriff der Ultrafilterkompaktheit. Somit folgt (1) direkt aus Satz \ref{thm:ascoliPIT}.
\end{proof}

Ähnlich gilt für den nächsten Kompaktheitsbegriff

\begin{thm}
  Äquivalent sind:
  \begin{enumerate}
    \item Der Áscoli Satz bezüglich Tychonoffkompaktheit.
    \item Der Boolesche Primidealsatz.
  \end{enumerate}
\end{thm}

\begin{proof}
  $(1)\Rightarrow(2) $: Es reicht aus zu zeigen, dass $(1)$ bereits impliziert, dass die Hilbert-Würfel $[0,1]^I$ im gewöhnlichen Sinne kompakt sind.
  Angenommen, es existiert eine Menge $I$, sodass $P:= [0,1]^I$ nicht kompakt ist.
  Hierbei sei das abgeschlossene Einheitsintervall $[0,1]$ mit der euklidischen Topologie ausgestattet.
  Dann existiert ein Filter $\mathcal{F}$ auf $P$ ohne Häufungspunkt.

  $(2)\Rightarrow(1)$: Der Boolesche Primidealsatz impliziert die Übereinstimmung des gewöhnlichen Kompaktheitsbegriffes mit dem Begriff der Tychonoffkompaktheit. Somit folgt (1) direkt aus Satz \ref{thm:ascoliPIT}.
\end{proof}


Ähnlich gilt für den nächsten Kompaktheitsbegriff

\begin{thm}
  Äquivalent sind:
  \begin{enumerate}
    \item Der Áscoli Satz bezüglich Alexandroff-Urysohn-Kompaktheit.
    \item Der Boolesche Primidealsatz.
  \end{enumerate}
\end{thm}

  $(1) \Rightarrow (2)$: Sei $X$ eine Menge und $\mathbf{2}$ die zweielementige Menge $\{0,1\}$ und seien beide mit der diskreten Topologie ausgestattet. 
  Die diskrete Topologie lässt sich auch als die von der \textit{diskreten Metrik} induzierte Topologie auffassen, somit kann $\mathbf{2}$ auch als metrischer Raum betrachtet werden. 

  Es sei $F := C(X, \mathbf{2})$.
  Dann gilt $C(X, \mathbf{2}) = \mathbf{2}^X$, da $X$ nach Voraussetzung diskret ist.
  Da $\mathbf{2} = F(x)$ für alle $x \in X$ als endlicher Raum keine unendliche Teilmenge besitzt, folgt sofort die Alexandroff-Urysohn-Kompaktheit von $F(x)$.
  Nach Definition ist $\mathbf{2}^X$ als Gesamtraum abgeschlossen in jeder Topologie, also insbesondere auch in der punktweisen Topologie $\tau_p$.
  Zuletzt ist $F$ auch gleichgradig stetig, da $X$ diskret und somit Einpunktmengen offen sind, d.h. für alle $x \in X$ und $\varepsilon > 0$ wähle man $\{x\}$ als offene Umgebung in $X$. Dann gilt für alle $f \in F$, aufgrund der Definitheit der Metrik für  $d(f(x),f(x)) = 0 \leq \varepsilon$.
  Damit sind die Bedingungen (b) des Áscoli-Satzes erfüllt und es folgt, dass $\mathbf{2}^X$ Alexandroff-Urysohn-kompakt ist. Dies ist äquivalent zum Auswahlaxiom.

  $(2)\Rightarrow(1)$: Das Auswahlaxiom impliziert die Übereinstimmung des gewöhnlichen Kompaktheitsbegriffes mit dem Begriff der Alexandroff-Urysohn-Kompaktheit. Somit folgt (1) direkt aus Satz \ref{thm:ascoliPIT}.

Wie die vorangehenden Ausführungen gezeigt haben, versagt der Áscoli-Satz bei allen vorgestellten Versionen von Kompaktheit. Es soll im Folgenden untersucht werden, inwiefern eine eingeschränkte Formulierung des Áscoli-Satzes die Anforderungen senkt.

\begin{defn}
  Der klassische Áscoli-Satz besagt, dass für eine Funktionenmenge $F$ stetiger Abbildungen $f \colon \R \to \R$ die folgenden Bedingungen äquivalent sind:
  \begin{itemize}
    \item[(a)] Jede Folge $(f_n)_{n \in \N}$ in $F$ besitzt eine Teilfolge $(f_{(\nu(n))})_{n \in \N}$ die stetig gegen eine nicht notwendig in $F$ liegende Funktion $g$ konvergiert, d.h.
      \begin{addmargin}[2em]{2em}% 1em left, 2em right
        Für alle $x \in \R$ und alle $(x_n)_{n \in N} \in \R^\N$ gilt: \\
        Aus $\lim_{n \to \infty} x_n = x$ folgt $\lim_{n \to \infty} f_{\nu(n)}(x_n) = g(x)$.
      \end{addmargin}
    \item[(b)]
      \begin{itemize}
        \item[($\alpha$)] Für alle $x \in \R$ ist die Menge $F(x) = \{ f(x) \mid f \in F\}$ beschränkt.
        \item[($\beta$)] Es ist $F$ gleichgradig stetig.
      \end{itemize}
  \end{itemize}
\end{defn}

Es sollen zunächst einige Eigenschaften von Funktionenfolgen in Bezug auf die stetige Konvergenz genannt werden:

\begin{prop}
  \begin{itemize}
    \item[a)] Falls $(f_n)_{n \in \N}$ stetig gegen $g$ konvergiert, so ist auch $g$ stetig.
    \item[b)] Falls $(f_n)_{n \in \N}$ lokal gleichmäßig gegen $g$ konvergiert, so konviergiert $(f_n)_{n \in \N}$ auch stetig gegen $g$.
    \item[c)] Falls $(f_n)_{n \in \N}$ stetig gegen $g$ konvergiert, so konvergiert $(f_n)_{n \in \N}$ auch punktweise gegen $g$.
  \end{itemize}
\end{prop}

\begin{proof}
  a): Es soll die Stetigkeit der Grenzfunktion $g$ mittels Folgenstetigkeit gezeigt werden. 
  Es sei dazu $(x_n)_{n \in \N}$ eine Folge in $\R$ mit Grenzwert $x$ und $\varepsilon > 0$ gegeben.
  Wie in c) noch gezeigt wird konvergiert $(f_n)_{n \in \mathbb{N}}$ punktweise.
  Daraus lässt sich nun induktiv eine Teilfolge $(f_{n_k})_{k \in \mathbb{N}}$ gewinnen mit
  \begin{displaymath}
    | f_{n_k}(x_k) - f(x_k) | \leq \frac{\varepsilon}{2}, \text{ für alle $k \in \N$}.
  \end{displaymath}
  Denn aufgrund der Punktweisen Konvergenz existiert zu $x_1$ ein $n_1$, sodass
  \begin{displaymath}
    | f_{n}(x_1) - f(x_1) | \leq \frac{\varepsilon}{2}, \text{ für alle $n \geq n_1$}
  \end{displaymath}
  analog findet man für $x_k$ ein $n_k > n_{k-1}$ mit derselben Eigenschaft.
  Die so konstrierte Teilfolge besitzt nachwievor denselben Grenzwert, d.h.
  \begin{displaymath}
    \lim_{k \to \infty} f_{n_k}(x_k) = f(x).
  \end{displaymath}
  Man sieht dies ein, indem man eine neue Folge $(y_k)_{k \in \N}$ definiert über $y_m:=x_1$ für $1 \leq m \leq n_1$ und $y_m:=x_n$ für $n_{m-1} \leq m \leq n_{m}$.
  Damit gilt nun
  \begin{displaymath}
    \lim_{m \to \infty} f_m(x_m) = f(x)
  \end{displaymath}
  und somit
  \begin{displaymath}
    \lim_{k \to \infty} f_{n_k}(y_{n_k}) 
    = \lim_{k \to \infty} f_{n_k}(x_{k}) = f(x).
  \end{displaymath}
  
  Aufgrund der stetigen Konvergenz der folge $(f_n)_{n \in \N}$ existiert nun ein $n_\varepsilon \in \N$, sodass für alle $n \geq n_\varepsilon$ gilt
  \begin{displaymath}
    | f_{n_k}(x_k) - f(x) | \leq \frac{\varepsilon}{2}.
  \end{displaymath}
  Daraus folgt nun unter Anwendung der Dreiecksungleichung
  \begin{displaymath}
    | f(x_k) - f(x) |
    \leq
    | f_{n_k}(x_k) - f(x_k) | + | f_{n_k}(x_k) - f(x) | 
    \leq \varepsilon.
  \end{displaymath}

  b):

  c): Dies folgt direkt aus der Definition unter Betrachtung konstanter Folgen $(x_n)_{n \in \N}$ mit $x_n = x$ für alle $n \in N$.
\end{proof}

\begin{thm}
  Äquivalent sind:
  \begin{enumerate}
    \item Der klassische Áscoli-Satz.
    \item $\mathbf{CC}(\R)$.
  \end{enumerate}
\end{thm}

\begin{proof}
  $(1) \Rightarrow (2)$:
  Es genügt zu zeigen, dass jede unbeschränkte Teilmenge $B$ von $\R$ eine unbeschränkte Folge enthält.
  Dazu sei $B \subseteq \R$ unbeschränkt.
  Für ein $b \in B$ lassen sich die konstante Abbildung $f_b \colon \R \to \R, f(x):= b$ und die Menge $F := \{f_b \mid b \in B\}$ definieren.
  Für $x \in \R$ gilt $F(x) = \{f_b(x) \mid b \in B\} = B$ und diese Menge ist nach Voraussetzung unbeschränkt
  Also ist Teil (b, $\alpha$) des klassischen Áscoli-Satzes verletzt.
  Mit 1. folgt sogleich, dass auch Teil (a) des klassischen Áscoli-Satzes nicht gelten kann und somit die Existenz einer Folge $(f_{b_n})_{n \in \N}$ ohne stetig konvergente Teilfolge.

  $(2) \Rightarrow (1)$:
  Es sei $F$ eine Menge stetiger Abbildungen $f \colon \R \to \R$.

  $(a) \Rightarrow (b,\alpha)$:
  Angenommen $(b,\alpha)$ gilt nicht. So existiert ein $x \in \R$, sodass $F(x)$ unbeschränkt ist.
  Nach Definition existiert also für alle $C \in R$ ein $f \in F$ mit $|f(x)| \geq C$.
  Insbesondere sind also die Mengen 
  \begin{displaymath}
    F_n := \{ f \in F \mid |f(x)| \geq m \}
  \end{displaymath}
  für alle $n \in N$ nichtleer.
  $CC(\R)$ impliziert folglich die Existenz einer Folge 
  \begin{displaymath}
    (f_n)_{n \in \N} \in \prod_{n \in N} F_n.
  \end{displaymath}
  Für obiges $x \in \R$ gilt also insbesondere $|f_{n_k}(x)| \geq n_k \geq n$ für alle $k \in \N$, sodass die Funktionenfolge $(f_n)_{n \in \N}$ in $x$ nicht punktweise, folglich auch nicht stetig konvergiert.
  Dies widerspricht jedoch Bedingung (a).

  $(a) \Rightarrow (b,\beta)$:
  Angenommen $(b, \beta)$ gilt nicht.
  So existiert ein $x \in \R$ und $\varepsilon > 0$, sodass das Bild keiner offenen Umgebung von $x$ in $U_\varepsilon(f(x))$ enthalten ist.
  Insbesondere gilt also auch, dass für kein $\delta > 0$ gilt, dass
  \begin{displaymath}
    f[U_\delta(x)] \subseteq U_\varepsilon(f(x)).
  \end{displaymath}
  Schließlich gilt für alle $n \in N$, dass die Mengen
  \begin{displaymath}
    F_n := \{ f \in F \mid \text{Es existiert ein } y \in U_{\frac{1}{n+1}}(x) \text{ mit } f(y) \not\in U_\varepsilon(f(x)) \}
  \end{displaymath}
  nicht leer sind.
  Wie zuvor impliziert $CC(\R)$ die Existenz einer Folge 
  \begin{displaymath}
    (f_n)_{n \in \N} \in \prod_{n \in N} F_n.
  \end{displaymath}
  Für obiges $x \in \R$ und $\varepsilon$ 
  Dies widerspricht jedoch Bedingung (a).
\end{proof}<++>
