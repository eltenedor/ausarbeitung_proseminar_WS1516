\section{Der Satz von Árzela-Áscoli}

Dieser Abschnitt ist eine ausgestaltete Darstellung des Kapitels ``Disasters in Topology III: Function Spaces (The Ascoli Theorem)'' aus dem Buch von Horst Herrlich.
\begin{defn}
  Gleichmäßige Stetigkeit.
\end{defn}

\begin{defn}[Satz von Árzela-Áscoli]
  Sei $(X,\tau)$ ein lokalkompakter Hausdorff-Raum und $(Y,d)$ ein metrischer Raum.
  Für jeden Teilraum $(F,\tau_{co})$ von $C(X,Y)$ sind äquivalent:
  \begin{enumerate}
    \item[(a)] $F$ ist kompakt.
    \item[(b)] \begin{enumerate}
        \item[($\alpha$)] Für alle $x \in X$ ist die Menge $F(x) = \{f(x) \mid f \in F \}$ kompakt in $Y$.
        \item[($\beta$)]  $F$ ist abgeschlossen in $Y$ bezüglich der Topologie der punktweisen Konvergenz $\tau_p$.
        \item[($\gamma$)] $F$ ist gleichmäßig stetig auf $X$.
      \end{enumerate}
  \end{enumerate}
\end{defn}

\begin{thm}
  Äquivalent sind
  \begin{enumerate}
    \item Der Satz von Árzela-Áscoli.
    \item Der Boolesche Primidealsatz.
  \end{enumerate}
\end{thm}

\begin{proof}
  $(2) \Rightarrow (1)$: Es seien $(X,\tau)$ ein lokalkompakter Hausdorff-Raum und $(Y,d)$ ein metrischer Raum.
  Des Weiteren sei $(F,\tau_{co})$ ein Teilraum vom $C(X,Y)$.

  $(a) \Rightarrow (b, \alpha)$: 
  Nach Voraussetzung ist $F$ ein bezüglich der kompakt-offenen Topologie $\tau_{co}$ kompakter Teilraum von $C(X,Y)$.
  Somit ist er insbesondere bezüglich der Spur der schwächeren Topologie der punktweisen Konvergenz $\tau_p$ ein kompakter Teilraum des Hausdorff-Raumes $Y^X$.
  Die Topologie der punktweisen Konvergenz lässt sich als Initialtopologie bezüglich der kanonischen Projektionen $p_x : Y^X \to Y, f \mapsto f(x)$ beschreiben.
  Als stetiges Bild eines Kompaktums ist somit auch $p_x(F) = F(x)$ kompakt.

  $(a) \Rightarrow (b, \beta)$:  
  Als kompakter Teilraum eines Hausdorff-Raumes ist $F$ bezüglich $\tau_p$ auch abgeschlossen in $Y^X$.

  $(a) \Rightarrow (b, \gamma)$:
  Es soll gezeigt werden, dass $F$ gleichgradig stetig ist.
  Sei dazu $x \in X$ und $\varepsilon > 0$.
  Für jedes $F$ ist die Menge
  \begin{displaymath}
    B_f := B_{\frac{\varepsilon}{2}} = \left\{ y \in Y \mid d(f(x),y) < \tfrac{\varepsilon}{2}\right\} 
  \end{displaymath}
  offen bezüglich der von der Metrik $d$ auf $Y$ induzierten Topologie.
  Da $F \subset C(X,Y)$ ist das Urbild $f^{-1}(B_f)$ ebenfalls offen bezüglich $\tau$.
  Nach Voraussetzung ist $X$ lokalkompakt.
  Daher existiert eine kompakte Umgebung $K_f$ von $x$ mit $K_f \subset f^{-1}(B_f)$.
  Damit gilt $f(K_f) \subset B_f$ und es definiert
  \begin{displaymath}
    U_f = F \bigcap (K_f, B_f) = \left\{g \in F \mid g(K_f) \subset B_f \right\} \tag{$\ast$}
  \end{displaymath}
  eine bezüglich der Teilraumtopologie auf $F$ offene Umgebung von $f$.
  Man betrachte nun die Evaluationsabbildung $\omega: X \times F \to Y, (y,g) \mapsto g(y)$.
  Unter Berücksichtigung von ($\ast$) folgt $\omega(K_f \times U_f) \subset B_f$.
  Es definiere $\mathcal{C}$ die Menge aller Tripel $(f,K,U)$, wobei $f \in F$, $K$ eine Umgebung von $x$ in $X$ und $U$ eine offene Umgebung von $f$ in $F$ mit $\omega(K \times U) \subset B_f$ sei.
  Dann liefert
  \begin{displaymath}
    \mathfrak{U} = \{ U \subset F \mid \text{es existieren $f \in F$, $K \subset X$ mit $(f,K,U) \in \mathcal C$} \}
  \end{displaymath}
  eine offene Überdeckung von $F$.
  Nach Voraussetzung ist $F$ kompakt, daher existiert eine endliche Teilüberdeckung durch $U_1,\dots,U_n \in \mathfrak{U}$.
  Für alle $i = 1,\dots,n$ wähle man nun $f_i \in F$ und $K_i \subset X$ mit $(f_i, K_i, U_i) \in \mathcal{C}$
  Des Weiteren definiere man $U := \bigcap_{i = 1}^n K_i$.
  Es ist $U$ eine (nicht notwendig offene) Umgebung von $x \in X$.
  Nun soll gezeigt werden, dass für diese Wahl von $U$ die Voraussetzung der gleichgradigen Stetigkeit erfüllt ist:
  Für jedes $f \in F$ existiert aufgrund der Überdeckungseigenschaft ein $i = 1,\dots,n$ mit $f \in U_i$.
  Sei des Weiteren $y \in U$ gegeben.
  Dann gilt
  \begin{displaymath}
    f(y) = \omega(y,f) \in \omega(U \times U_i) \subset \omega(K_i \times U_i) \subset B_{f_i},
  \end{displaymath}
  also $d(f_i(x),f(y)) < \tfrac{\varepsilon}{2}$.
  Insbesondere gilt für $x \in U$, nach Voraussetzung gilt $x \in K_i$ für alle $i = 1,\dots,n$, dass $d(f_i(x),f(x)) < \tfrac{\varepsilon}{2}$.
  Unter Verwendung der Dreiecksungleichung folgt
  \begin{displaymath}
    d(f(x),f(y)) \leq d(f(x),f_i(x)) + d(f_i(x),f(y)) < \frac{\varepsilon}{2} + \frac{\varepsilon}{2} = \varepsilon.
  \end{displaymath}
  Folglich ist $F$ gleichgradig stetig.

  $(b) \Rightarrow (a)$
  Nach Voraussetzung ist $F(x)$ als Teilraum von $Y^X$ kompakt und hausdorff'sch.
  Der Tychonoff-Satz für Kompaktheit besagt, dass dann auch $\prod_{x \in X} F(x)$ kompakt ist.
  Die Voraussetzung, dass $F$ abgeschlossen in $Y^X$ bezüglich der Produkttopologie $\tau_p$ ist, liefert, dass $F$ auch abgeschlossen im Teilraum $\prod_{x \in X} F(x)$ ist.
  Als abgeschlossene Teilmenge eines Kompaktums ist somit $F$ auch kompakt bezüglich der Topologie der punktweisen Konvergenz.
  Es fehlt zu zeigen, dass $F$ ebenfalls bezüglich der stärkeren, kompakt-offenen Topologie $\tau_{co}$ kompakt ist.
  Hierfür soll die Inklusion $\tau_{co} \subset \tau_p$ nachgewiesen werden.
  Dazu betrachte man ein Subbasiselement $V = (K,U) \cap F$ der kompakt-offenen Topologie auf dem Teilraum $F$ und ein $f \in V$.
  Es gilt nun zu zeigen, dass $V$ auch in $\tau_p$ offen ist.
  Da $f(K) \subset U$ und $U \in \sigma$, gilt für alle $x \in K$
  \begin{displaymath}
    r_x = \inf\{d(f(x),y) \mid y \in (Y \setminus U) \} > 0.
  \end{displaymath}
  Dann ist $U_x = \{ z \in X \mid d(f(x),f(z)) < \tfrac{r_x}{2}\}$ eine offene Umgebung von $x \in X$.
  Denn für alle $z \in U_x$ mit $\varepsilon := \tfrac{r_x}{2} - d(f(x),f(z)) > 0$ gilt, dass aufgrund der Stetigkeit von $f$ eine Umgebung $U'$ von $z$ existiert, sodass $f(U') \subset U_\varepsilon(f(z))$ gilt.
  Daraus folgt für alle $u \in U'$ mit der Dreiecksungleichung
  \begin{align*}
    d(f(x),f(u)) &< d(f(x),f(z)) + d(f(z),f(u)) \\
                 &< \varepsilon + d(f(z),f(u)) = \tfrac{r_x}{2} - d(f(x),f(z)) + d(f(x),f(z)) =  \tfrac{r_x}{2}
  \end{align*}
  und damit schließlich $U' \subset U_x$, also die Offenheit von $U_x$ in $X$.
  Zudem bildet $\mathfrak{U} := \{U_x \mid x \in K\}$ eine offene Überdeckung von K. 
  Nach Voraussetzung ist $K$ kompakt, daher existieren $x_1,\dots,x_n \in K$, sodass $K \subset \bigcup_{i=1}^n U_{x_i}$.
  Es sei $r:= \min\{r_{x_1},\dots,r_{x_n}\}$. 
  Damit ergibt sich für alle $x \in K$ und alle $y \in (Y \setminus U)$ die Ungleichung $d(f(x),y) \geq \tfrac{r}{2}$.
  Ist $x \in K$ und $d(f(x),y) < \tfrac{r}{2}$ so folgt daraus $y \in U$.
  Da $F$ nach Voraussetzung gleichgradig stetig auf $X$ ist, existiert für alle $x \in X$ eine offene Umgebung $W$ von $x$ in $X$, sodass für alle $g \in F$ und alle $z \in W$ gilt, dass $d(g(x),g(z)) < \tfrac{r}{4}$.
  Betrachtet man nun die auf obige Weise erzeugte Menge $\mathcal{C}$ aller geordneten Paare $(x,W)$, so bildet
  \begin{displaymath}
    \mathfrak{W} := \{W \subset X \mid \text{Es existiert ein $x \in K$ mit $(x,W) \in \mathcal{C}$} \}
  \end{displaymath}
  eine offene Überdeckung von $K$.
  Aufgrund der Kompaktheit von $K$ lässt sich eine endliche Teilüberdeckung $W_1,\dots,W_m \in \mathfrak{W}$ von $K$ auswählen.
  Für alle $i = 1,\dots,m$ wähle man entsprechend der Definition $\mathfrak{W}$ ein $x_i$ mit $(x_i,W_i) \in \mathcal{C}$.
  Dann ist
  \begin{displaymath}
    B_f := \{ g \in F \mid d(f(x_i),g(x_i)) < \tfrac{r}{4} \text{ für } i = 1,\dots,m\}
  \end{displaymath}
  eine offene Umgebung von $f$ bezüglich $\tau_p$.
  Es soll nun gezeigt werden, dass $B_f \subset V$ gilt.
  Dann ist nämlich $V = \bigcup_{f \in V} B_f$ als Vereinigung offener Mengen offen bezüglich $\tau_p$ und die Behauptung $\tau_{co} \subset \tau_p$ folgt, da $V$ beliebig war.
  Sei dazu also $g \in B_f$.
  Für jedes $x \in K$ existiert ein $i \in \{1,\dots,m\}$ mit $x \in W_i$.
  Nach Konstruktion der $W_i$ gilt $d(g(x_i),g(x)) < \tfrac{r}{4}$.
  Da $g \in B_f$ gilt zudem die Ungleichung $d(f(x_i),g(x_i)) < \tfrac{r}{4}$.
  Daraus folgert man
  \begin{displaymath}
    d(f(x_i),g(x)) \leq d(f(x_i),g(x_i)) + d(g(x_i),g(x)) < \tfrac{r}{2}.
  \end{displaymath}
  Daraus folgt $g(x) \in U$ und damit $g(K) \subset U$. 
  Also ist $g \in (K,U)$ woraus $g \in V$ folgt.
  Damit ist der Beweis vollständig.
\end{proof}
