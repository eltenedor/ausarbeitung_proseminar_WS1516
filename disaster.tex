%\section{Der \textsc{Áscoli}-Satz und das Auswahlaxiom}
\chapter{Der \textsc{Áscoli}-Satz und das Auswahlaxiom}

\textsc{Áscoli}-Sätze gelten bezeichnend für Aussagen, die es gestatten Kompaktheit von Teilmengen von Funktionenräumen bezüglich starker Topologien aus der Kompaktheit bezüglich einer schwächeren Topologie unter zusätzlichen Annahmen zu folgern. Dabei existieren unterschiedliche Formulierungen.

Dieser Abschnitt ist eine ausgestaltete Darstellung des Kapitels  
\begin{addmargin}[2em]{2em}% 1em left, 2em right
  \textit{Disasters in Topology III: Function Spaces (The Ascoli Theorem)}
\end{addmargin}
aus \cite{herrlich2006axiom}. 

%\begin{defn}
%  Es seien $(X,d_X)$ und $(Y,d_Y)$ metrische Räume. Eine Abbildung $f \colon X \to Y$ heißt \textit{gleichmäßig stetig}, wenn gilt:
%  \begin{addmargin}[2em]{2em}% 1em left, 2em right
%  Zu jedem $\varepsilon > 0$ existiert ein $\delta > 0$, sodass \\
%  für alle $x,x' \in X$ mit $d_X(x,x') < \delta$ gilt, dass $d_Y(f(x),f(x')) < \varepsilon$.
%  \end{addmargin}
%\end{defn}

\begin{defn}
  Es sei $(X,\tau)$ ein topologischer Raum und $(Y,d)$ ein metrischer Raum versehen mit der durch die Metrik induzierten Topologie $\tau_d$. Eine Teilmenge $F \subseteq Y^X$ heißt \textit{gleichgradig stetig}, wenn gilt:
  \begin{addmargin}[2em]{2em}% 1em left, 2em right
    Für alle $x \in X$ und $\varepsilon > 0$ existiert ein $V \in \dot{x} \cap \tau$, sodass \\
    für alle $f \in F$ gilt, dass $f(V) \subseteq U_\varepsilon(f(x))$.
  \end{addmargin}
\end{defn}

Für zwei topologische Räume $(X,\tau)$ und $(Y,\sigma)$ bezeichne im Folgenden $C_{\co}(X,Y)$ den topologischen Funktionenraum $(C(X,Y),\tau_{\co})$ der stetigen Funktionen von $X$ nach $Y$ ausgestattet mit der kompakt-offenen Topologie.

Eine topologische Version des \textsc{Áscoli}-Satzes lautet:
\begin{defn}[Topologischer \textsc{Áscoli}-Satz]
  Sei $(X,\tau)$ ein lokalkompakter \textsc{Hausdorff}-Raum und $(Y,d)$ ein metrischer Raum.
  Für jeden Teilraum $(F,\tau_{\co})$ von $C_{\co}(X,Y)$ sind äquivalent:
  \begin{enumerate}
    \item[(a)] $F$ ist $\tau_{\co}$-kompakt.
    \item[(b)] \begin{enumerate}
        \item[($\alpha$)] Für alle $x \in X$ ist die Menge $F(x) = \{f(x) \mid f \in F \}$ kompakt in $Y$.
        \item[($\beta$)]  $F$ ist abgeschlossen in $Y^X$ bezüglich der Topologie der punktweisen Konvergenz $\tau_p$.
        \item[($\gamma$)] $F$ ist gleichgradig stetig auf $X$.
      \end{enumerate}
  \end{enumerate}
\end{defn}

Man beachte, dass die auf dem Teilraum $F$ verwendete Topologie die Spur $\tau_{\co}$ der kompakt-offenen Topologie auf $C(X,Y)$ ist.
Es soll nun gezeigt werden, dass die Gültigkeit dieses Satzes äquivalent zum \textsc{Boole}schen Primidealsatz (\PIT) ist.

\begin{thm}
  \label{thm:ascoliPIT}
  Äquivalent sind
  \begin{enumerate}
    \item[(1)] Der topologische \textsc{Áscoli}-Satz.
    \item[(2)] \PIT.
  \end{enumerate}
\end{thm}

\begin{proof}
  (1) $\Rightarrow$ (2): Sei $X$ eine Menge und $\mathbf{2}$ die zweielementige Menge $\{0,1\}$. Beide Mengen seien  mit der diskreten Topologie ausgestattet. 
  Die diskrete Topologie lässt sich auch als die von der \textit{diskreten Metrik} induzierte Topologie auffassen, somit kann $\mathbf{2}$ insbesondere als metrischer Raum betrachtet werden. 

  Es sei $F := C(X, \mathbf{2})$.
  Dann gilt $C(X, \mathbf{2}) = \mathbf{2}^X$, da $X$ nach Voraussetzung diskret ist.
  Es folgt, dass $\mathbf{2} = F(x)$ für alle $x \in X$ als endlicher Raum kompakt ist.
  Nach Definition ist $\mathbf{2}^X$ als Gesamtraum abgeschlossen in jeder Topologie, also insbesondere auch in der punktweisen Topologie $\tau_p$.
  Zuletzt ist $F$ auch gleichgradig stetig, da $X$ diskret und somit Einpunktmengen offen sind und man daher für alle $x \in X$ und $\varepsilon > 0$ die Menge $\{x\}$ als offene Umgebung von $x$ wählen kann.
  Dann gilt nämlich für alle $f \in F$, aufgrund der Definitheit der Metrik für  $d(f(x),f(x)) = 0 < \varepsilon$.
  Damit sind alle Bedingungen aus $(b)$ des \textsc{Áscoli}-Satzes erfüllt und es folgt, dass $\mathbf{2}^X$ kompakt ist. 
  Dies ist nach Satz \ref{thm:pitequivalence} äquivalent zu \PIT.

  (2) $\Rightarrow$ (1): Es seien $(X,\tau)$ ein lokalkompakter \textsc{Hausdorff}-Raum und $(Y,d)$ ein metrischer Raum.
  Des Weiteren sei $F$ ein Teilraum vom $C_{\co}(X,Y)$.

  (a) $\Rightarrow$ (b, $\alpha$): 
  Nach Voraussetzung ist $F$ ein bezüglich der kompakt-offenen Topologie $\tau_{\co}$ kompakter Teilraum von $C(X,Y)$.
  Somit ist er insbesondere bezüglich der Spur der schwächeren Topologie der punktweisen Konvergenz $\tau_p$ ein kompakter Teilraum des \textsc{Hausdorff}-Raumes $Y^X$.
  Die Topologie der punktweisen Konvergenz lässt sich als Initialtopologie bezüglich der für alle $x \in X$ definierten kanonischen Projektionen 
  \begin{displaymath}
    \pi_x : Y^X \to Y, \quad f \mapsto f(x)
  \end{displaymath}
  beschreiben.
  Als stetiges Bild eines Kompaktums ist somit auch $\pi_x(F) = F(x)$ kompakt in $(Y,\tau_d)$.

  (a) $\Rightarrow$ (b, $\beta$):  
  Als kompakter Teilraum eines \textsc{Hausdorff}-Raumes ist $F$ bezüglich $\tau_p$ auch abgeschlossen in $Y^X$.

  (a) $\Rightarrow$ (b, $\gamma$):
  Es soll gezeigt werden, dass $F$ gleichgradig stetig ist.
  Sei dazu $x \in X$ und $\varepsilon > 0$.
  Für jedes $f \in F$ ist die Menge
  \begin{displaymath}
    B_f := B_{\frac{\varepsilon}{2}} = \left\{ y \in Y \mid d(f(x),y) < \tfrac{\varepsilon}{2}\right\} 
  \end{displaymath}
  offen bezüglich der von der Metrik $d$ auf $Y$ induzierten Topologie.
  Da $F \subseteq C(X,Y)$, ist das Urbild $f^{-1}(B_f)$ ebenfalls offen bezüglich $\tau$.
  Nach Voraussetzung ist $X$ lokalkompakt.
  Daher existiert eine kompakte Umgebung $K_f$ von $x$ mit $K_f \subseteq f^{-1}(B_f)$.
  Damit gilt $f(K_f) \subseteq B_f$ und es definiert
  \begin{displaymath}
    U_f := F \cap (K_f, B_f) = \left\{g \in F \mid g(K_f) \subseteq B_f \right\} \tag{$\ast$}
  \end{displaymath}
  eine bezüglich der Teilraumtopologie auf $F$ offene Umgebung von $f$.
  Man betrachte nun die Evaluationsabbildung 
  \begin{displaymath}
    \omega: X \times F \to Y, \quad (y,g) \mapsto g(y).
  \end{displaymath}
  Unter Berücksichtigung von ($\ast$) folgt $\omega(K_f \times U_f) \subseteq B_f$.
  Es definiere $\mathcal{C}$ die Menge aller Tripel $(f,K,U)$, wobei $f \in F$, $K$ eine Umgebung von $x$ in $X$ und $U$ eine offene Umgebung von $f$ in $F$ mit $\omega(K \times U) \subseteq B_f$ sei.
  Dann liefert
  \begin{displaymath}
    \mathfrak{U} := \{ U \subseteq F \mid \text{es existieren $f \in F$, $K \subseteq X$ mit $(f,K,U) \in \mathcal C$} \}
  \end{displaymath}
  eine offene Überdeckung von $F$, da für alle $f \in F$ nach Konstruktion $(f,K_f,U_f) \in \mathfrak{U}$ gilt.
  Nach Voraussetzung ist $F$ kompakt, daher existiert eine endliche Teilüberdeckung durch $U_1,\dots,U_n \in \mathfrak{U}$.
  Für alle $i = 1,\dots,n$ wähle man nun $f_i \in F$ und $K_i \subseteq X$ mit $(f_i, K_i, U_i) \in \mathcal{C}$.
  Des Weiteren definiere man 
  \begin{displaymath}
    U := \bigcap_{i = 1}^n K_i.
  \end{displaymath}
  Es ist $U$ eine (nicht notwendig offene) Umgebung von $x \in X$.
  Nun soll gezeigt werden, dass für diese Wahl von $U$ die Voraussetzung der gleichgradigen Stetigkeit erfüllt ist, also dass für alle $f \in F$ folgt, dass $f(U) \subseteq U_\varepsilon(f(x))$ gilt.
  Für jedes $f \in F$ existiert aufgrund der Überdeckungseigenschaft ein $i = 1,\dots,n$ mit $f \in U_i$.
  Sei des Weiteren $y \in U$ gegeben.
  Dann gilt
  \begin{displaymath}
    f(y) = \omega(y,f) \in \omega(U \times U_i) \subseteq \omega(K_i \times U_i) \subseteq B_{f_i},
  \end{displaymath}
  also $d(f_i(x),f(y)) < \tfrac{\varepsilon}{2}$.
  Insbesondere gilt für $x \in U$, da nach Voraussetzung $x \in K_i$ für alle $i = 1,\dots,n$, dass $d(f_i(x),f(x)) < \tfrac{\varepsilon}{2}$.
  Unter Verwendung der Dreiecksungleichung folgt letztlich
  \begin{displaymath}
    d(f(x),f(y)) \leq d(f(x),f_i(x)) + d(f_i(x),f(y)) < \frac{\varepsilon}{2} + \frac{\varepsilon}{2} = \varepsilon
  \end{displaymath}
  und damit auch $f(U) \subseteq U_\varepsilon(f(x))$.
  Folglich ist $F$ gleichgradig stetig.

  (b) $\Rightarrow$ (a)
  Nach Voraussetzung ist $F(x)$ als Teilraum von $Y$ kompakt und besitzt zudem die \textsc{Hausdorff}-Eigenschaft.
  %%%%%%%% PIT %%%%%%%%
  Der \textsc{Tychonoff}-Satz für \textsc{Hausdorff}-Räume \ref{thm:pitequivalence}(3) besagt, dass dann auch $\prod_{x \in X} F(x) \subseteq Y^X$ kompakt ist.
  %%%%%%%%%%%%%%%%%%%%%
  Die Voraussetzung, dass $F$ abgeschlossen in $Y^X$ bezüglich der Produkttopologie $\tau_p$ ist, liefert, dass $F$ auch abgeschlossen im Teilraum $\prod_{x \in X} F(x)$ ist, da $F \subseteq \prod_{x \in X} F(x)$ gilt.
  Als abgeschlossene Teilmenge eines Kompaktums ist somit $F$ auch kompakt bezüglich der Topologie der punktweisen Konvergenz.

  Es fehlt zu zeigen, dass $F$ ebenfalls bezüglich der kompakt-offenen Topologie $\tau_{\co}$ kompakt ist.
  Hierfür soll die Inklusion $\tau_{\co} \subseteq \tau_p$ nachgewiesen werden.
  Dazu betrachte man ein Subbasiselement 
  \begin{displaymath}
    V := (K,U) \cap F \in \tau_{\co}
  \end{displaymath}
  der kompakt-offenen Topologie auf dem Teilraum $F$ und ein $f \in V$.
  Es gilt nun zu zeigen, dass $V$ auch in $\tau_p$ offen ist.
  Da $f(K) \subseteq U$ und $U \in \sigma$ gilt, folgt für alle $x \in K$
  \begin{displaymath}
    r_x := \inf\{d(f(x),y) \mid y \in (Y \setminus U) \} > 0.
  \end{displaymath}
  Dann ist 
  \begin{displaymath}
    U_x := \{ z \in X \mid d(f(x),f(z)) < \tfrac{r_x}{2}\} = f^{-1}\left( U_{\tfrac{r_x}{2}}(f(x))\right)
  \end{displaymath} 
  eine offene Umgebung von $x \in X$ aufgrund der Stetigkeit von $f$.
% Denn für alle $z \in U_x$ mit $\varepsilon := \tfrac{r_x}{2} - d(f(x),f(z)) > 0$ gilt, dass aufgrund der Stetigkeit von $f$ eine Umgebung $U' \in \tau$ von $z$ existiert, sodass $f(U') \subseteq U_\varepsilon(f(z))$ gilt.
% Daraus folgt für alle $u \in U'$ mit der Dreiecksungleichung
% \begin{align*}
%   d(f(x),f(u)) &\leq d(f(x),f(z)) + d(f(z),f(u)) \\
%                &< \varepsilon + d(f(z),f(u)) = \frac{r_x}{2} - d(f(x),f(z)) + d(f(x),f(z)) =  \frac{r_x}{2}
% \end{align*}
% und damit schließlich $U' \subseteq U_x$, also die Offenheit von $U_x$ in $X$.
  Zudem bildet $\mathfrak{U} := \{U_x \mid x \in K\}$ eine offene Überdeckung von $K$. 
  Nach Voraussetzung ist $K$ kompakt, daher existieren $x_1,\dots,x_n \in K$, sodass $K \subseteq \bigcup_{i=1}^n U_{x_i}$.
  Es sei $r:= \min\{r_{x_1},\dots,r_{x_n}\}$. 
  Damit ergibt sich für alle $x \in K$ und alle $y \in (Y \setminus U)$ die Ungleichung $d(f(x),y) \geq \tfrac{r}{2}$.
  Ist andererseits $x \in K$ und $d(f(x),y) < \tfrac{r}{2}$, so folgt daraus $y \in U$.

  Da $F$ nach Voraussetzung gleichgradig stetig auf $X$ ist, existiert insbesondere für alle $x \in K$ eine offene Umgebung $W$ von $x$, sodass 
  \begin{displaymath}
    d(g(x),g(z)) < \tfrac{r}{4}, \quad \text{für alle $g \in F$ und $z \in W$}.
  \end{displaymath}
  Betrachtet man nun die auf obige Weise erzeugte Menge $\mathcal{C}$ aller geordneten Paare $(x,W)$, so bildet
  \begin{displaymath}
    \mathfrak{W} := \{W \subseteq X \mid \text{Es existiert ein $x \in K$ mit $(x,W) \in \mathcal{C}$} \}
  \end{displaymath}
  eine offene Überdeckung von $K$.
  Aufgrund der Kompaktheit von $K$ lässt sich eine endliche Teilüberdeckung $W_1,\dots,W_m \in \mathfrak{W}$ auswählen.
  Für alle $i = 1,\dots,m$ wähle man entsprechend der Definition $\mathfrak{W}$ ein $x_i$ mit $(x_i,W_i) \in \mathcal{C}$.
  Dann ist
  \begin{displaymath}
    B_f := \{ g \in F \mid d(f(x_i),g(x_i)) < \tfrac{r}{4} \text{ für } i = 1,\dots,m\}
  \end{displaymath}
  als Basisumgebung von $\tau_p$ insbesondere eine offene Umgebung von $f$.

  Es soll nun gezeigt werden, dass $B_f \subseteq V$ gilt.
  Dann ist nämlich $V = \bigcup_{f \in V} B_f$ als Vereinigung offener Mengen offen bezüglich $\tau_p$ und die Behauptung $\tau_{\co} \subseteq \tau_p$ folgt, da $V$ beliebig war.
  Sei dazu also $g \in B_f$.
  Für jedes $x \in K$ existiert ein $i \in \{1,\dots,m\}$ mit $x \in W_i$.
  Nach Konstruktion der $W_i$ gilt $d(g(x_i),g(x)) < \tfrac{r}{4}$.
  Da $g \in B_f$ gilt zudem die Ungleichung $d(f(x_i),g(x_i)) < \tfrac{r}{4}$.
  Daraus folgert man
  \begin{displaymath}
    d(f(x_i),g(x)) \leq d(f(x_i),g(x_i)) + d(g(x_i),g(x)) < \frac{r}{2},
  \end{displaymath}
  was wiederum $g(x) \in U$ und damit $g(K) \subseteq U$ impliziert. 
  Also ist $g \in (K,U)$, woraus $g \in V$ folgt.
  Damit ist der Beweis vollständig.
\end{proof}

Es fällt auf, dass die Implikation (a) $\Rightarrow$ (b) bereits im Rahmen von \ZF gültig ist.
Der Beweis des vorangehenden Satzes verwendet lediglich einer Stelle der Beweisrichtung (b) $\Rightarrow$ (a) den laut Satz \ref{thm:pitequivalence} zu \PIT äquivalenten \textsc{Tychonoff}-Satz für kompakte \textsc{Hausdorff}-Räume.
Unter der Verwendung des Kompaktheitsbegriffs der Ultrafilterkompaktheit ist diese Schlussweise gemäß Lemma \ref{lem:uftprod} jedoch bereits in \ZF möglich.  
Dies motiviert den Versuch zu untersuchen, inwiefern der topologische \textsc{Áscoli}-Satz unter Verwendung anderer Kompaktheitsbegriffe seine Gültigkeit behält.

\begin{lem}
  \label{lem:filterclusterpoint}
  Es sei $(X,\tau)$ ein topologischer Raum, sodass die darin offen-abgeschlossenen Mengen einen Basis für die Topologie $\tau$ bilden.
  \begin{itemize}
    \item[(1)] Ist $\mathcal{F}$ ein Filter ohne Häufungspunkt, so gilt für die Menge $\mathfrak{A} \subseteq \mathcal{F}$ aller offen-abgeschlossenen Elemente 
     \begin{displaymath}
       \bigcap_{A \in \mathfrak{A}} A = \emptyset.
     \end{displaymath}
   \item[(2)] Erfüllt $(X,\tau)$ zusätzlich das $T1$-Axiom, so existiert zu je zwei unterschiedlichen Punkten $x,y \in X$ ein $A_{(x,y)} \in \mathfrak{A}$ mit 
     \begin{displaymath}
       \{x,y\} \cap A_{(x,y)} = \{y\}.
     \end{displaymath}
  \end{itemize}
  %FUNKTIONIERT AUCH BEREITS MIT T0!!! S.a. Bartsch S.246
\end{lem}

\begin{proof}
  (1):
  Nach Voraussetzung besitzt $\mathcal{F}$ keinen Häufungspunkt.
  Dies bedeutet, dass
  \begin{displaymath}
    \bigcap_{F \in \mathcal{F}} \overline F = \emptyset.
  \end{displaymath}
  Für alle $x \in X$ gilt damit, dass ein $F \in \mathcal{F}$ existiert, sodass $x \not\in \overline{F}$.
  Das bedeutet, dass $X \setminus \overline{F}$ eine offene Umgebung von $x$ ist.
  Da die offen-abgeschlossenen Mengen nach Voraussetzung eine Basis für $\tau$ bilden, existiert eine offen-abgeschlossene Umgebung $U$ von $x$, mit $U \subseteq X \setminus \overline{F}$ und damit
  \begin{displaymath}
    U \cap F = \emptyset.
  \end{displaymath}
  Folglich ist $X \setminus U \supseteq F$ und wegen des Abschlusses gegen Obermengenbildung
  \begin{displaymath}
    X \setminus U \in \mathcal{F}.
  \end{displaymath}
  Es existiert also eine offen-abgeschlossene Menge in $\mathcal{F}$, die $x$ nicht enthält. 
  Dies impliziert
  \begin{displaymath}
    x \not\in \bigcap_{A \in \mathfrak{A}} A.
  \end{displaymath}
  Da $x$ beliebig gewählt war, folgt 
  \begin{displaymath}
   \bigcap_{A \in \mathfrak{A}} A = \emptyset
  \end{displaymath}
  und damit die Behauptung.

  (2):
  Unter der Voraussetzung, dass $(X,\tau)$ das T1-Axiom erfüllt, lässt sich zur offenen Umgebung $X \setminus \overline{F}$ aus Teil (1) des Beweises eine offene Umgebung 
  \begin{displaymath}
    U_x \subseteq X \setminus \overline{F}
  \end{displaymath}
  von $x$ finden, die zusätzlich $y$ nicht enthält. 
  Da die offen-abgeschlossenen Mengen eine Basis von $\tau$ bilden, kann man zudem annehmen, dass $U_x$ offen-abgeschlossen ist.
  Man prüft nach, dass für
  \begin{displaymath}
    A_{(x,y)} := X \setminus U_x
  \end{displaymath}
  die in (2) geforderten Eigenschaften erfüllt sind:
  Als Komplement einer offen-abgeschlossenen Menge, ist auch $A_{(x,y)}$ offen und abgeschlossen.
  Nach Konstruktion ist 
  \begin{displaymath}
    x \not\in A_{(x,y)} \text{ sowie } y \in A_{(x,y)}.
  \end{displaymath}
  Zudem gilt, wie schon in (1) gezeigt, $A_{(x,y)} \in \mathcal{F}$.
  Daraus ergibt sich die Behauptung. \qedhere
\end{proof}

\begin{lem}
  \label{lem:openmap}
  Es seien $(X,\tau)$ und $(Y,\sigma)$ topologische Räume und es sei $\mathfrak{S}$ eine Subbasis für $\tau$.  
  Dann ist eine injektive Abbildung $f \colon X \to Y$ genau dann offen, wenn $f(\mathfrak{S}) \subseteq \sigma$.
\end{lem}

\begin{proof}
  Da $\mathfrak{S} \subseteq \tau$, folgt aus der Offenheit von $f$ sofort 
  \begin{displaymath}
    f(\mathfrak{S}) \subseteq f(\tau) \subseteq \sigma.
  \end{displaymath}

  Sei andererseits $f$ eine injektive Abbildung und $O \in \tau$ beliebig.
  Dann ist $O$ die Vereinigung endlicher Schnitte von Subbasislementen.
  Für zwei beliebige Mengen $O_1,O_2$ gilt nun 
  \begin{displaymath}
    f(O_1 \cap O_2) = f(O_1) \cap f(O_n)
  \end{displaymath}
  aufgrund der Injektivität von $f$, sowie 
  \begin{displaymath}
    f \left(\bigcup_{i \in I} O_i \right) = \bigcup_{i \in I} f(O_i)
  \end{displaymath}
  für eine beliebige Familie offener Mengen.
  Damit ist auch $f(O)$ offen und die Behauptung folgt.
\end{proof}

\begin{thm}
  \label{thm:ultrafiltercompact}
  Äquivalent sind:
  \begin{enumerate}
    \item[(1)] Der \textsc{Áscoli}-Satz bezüglich Ultrafilterkompaktheit.
    \item[(2)] \PIT.
  \end{enumerate}
\end{thm}

\begin{proof}
  $(1)\Rightarrow(2) $: Nach Satz \ref{thm:pitequivalence} reicht es aus zu zeigen, dass $(1)$ bereits impliziert, dass die Kantor-Würfel $\mathbf{2}^I$ im gewöhnlichen Sinne kompakt sind.

  Angenommen, es existiere eine Menge $I$, sodass $P:= \mathbf{2}^I$ nicht kompakt ist.
  Man betrachte $I$ als einen mit der diskreten Topologie ausgestatteten topologischen Raum und $\mathbf{2}$ als diskreten metrischen Raum.
  Dann existiert nach Satz \ref{thm:compactness} ein Filter $\mathcal{F}$ auf $P$ ohne Häufungspunkt.

  Der mit der Produkttopologie ausgestattete Raum $P$ besitzt eine Basis aus offen-ab\-ge\-schlossenen Mengen.
  Dies folgt daraus, dass bereits jedes Subbasiselement $S$ per Definition offen ist und, da $\mathbf{2}$ diskret ist, auch ein offenes Komplement besitzt.
  Damit ist $S$ aber als Komplement einer offenen Menge definitionsgemäß abgeschlossen.
  Insgesamt ist $S$ also offen-abgeschlossen.
  Da die Menge der offen-abgeschlossenen Mengen die endliche Durchschnittseigenschaft besitzt, ist auch jedes von der Subbasis erzeugte Basiselement eine offen-abgeschlossene Menge.

  Hiermit lässt sich nun Lemma \ref{lem:filterclusterpoint}(1) auf $\mathcal{F}$ anwenden.
  Damit folgt für die Menge $\mathfrak{A}$ aller offen-abgeschlossenen Elemente von $\mathcal{F}$
  \begin{displaymath}
    \bigcap_{A \in \mathfrak{A}} A = \emptyset \tag{$\ast$}.
  \end{displaymath}

  Man kann nun für alle $A \in \mathfrak{A}$ eine Funktion
  \begin{equation}
    \label{eq:clopenmap}
    f_A \colon P \to \mathbf{2}, \quad
    f_A(x) := 
    \begin{cases}
      1 &\text{, falls } x \in A \\
      0 &\text{, falls } x \not\in A
    \end{cases}
  \end{equation}
  definieren.
  Da alle $A$ offen-abgeschlossen sind, folgt sofort die Stetigkeit von $f_A$ für alle $A \in \mathfrak{A}$.
  Somit induziert die Familie $(f_A)_{A \in \mathfrak{A}}$ eine Abbildung 
  \begin{displaymath}
    f \colon P \to \mathbf{2}^\mathfrak{A}, \quad
    x \mapsto (f_A(x))_{A \in \mathfrak{A}},
  \end{displaymath}
  welche bezüglich der Produkttopologie, der initialen Topologie bezüglich der kanonischen Projektionen, auf $P$ stetig ist.

  Es bezeichne $F := f(P)$ das Bild von $P$ unter der obigen Abbildung $f$.
  Es soll nun gezeigt werden, dass $f$ eine Einbettung ist, also ein Homöomorphismus auf $F$:
  Die Abbildung $f$ ist injektiv, denn für $x,y \in P$ mit $x \neq y$ gilt, da $P$ als \textsc{Hausdorff}-Raum insbesondere das $T1$-Axiom erfüllt,
  \begin{displaymath}
    f_{A_{(x,y)}}(x) = 0 \neq 1 = f_{A_{(x,y)}}(y),
  \end{displaymath}
  mit $A_{(x,y)} \in \mathfrak{A}$ aus Lemma \ref{lem:filterclusterpoint}(2).

  Zudem ist $f$ offen bezüglich der Spurtopologie auf $F$, wie sich mit Lemma \ref{lem:openmap} beweisen lässt. 
  Dazu sei nun $O = \pi_{i_0}^{-1}(\{j\})$ mit $i_0 \in I$ und $j \in \{0,1\}$ eine beliebige offene Subbasismenge der Produkttopologie auf $P$,  wobei die kanonischen Projektionen des Produktraumes mit $\pi_i$ bezeichnet seien.
  Man erkennt $O$ als eine offen-abgeschlossene Menge aufgrund der Stetigkeit der kanonischen Projektionen.

  Gilt $O \in \mathcal{F}$, also insbesondere $O \in \mathfrak{A}$, so ist für alle $x \in O$
  \begin{displaymath}
    \pi_O(f(x)) = f_O(x) = 1.
  \end{displaymath}
  Daraus folgt 
  \begin{displaymath}
    f(O) = \pi_O^{-1}(\{1\}) \cap f(P)
  \end{displaymath}
  und folglich ist $f(O)$ offen bezüglich der Teilraumtopologie auf $f(P)$.

  Ist andernfalls $O \not\in \mathcal{F}$, so gilt für alle $M \in \mathcal{F}$ entweder
  \begin{displaymath}
    M \cap O \neq \emptyset
  \end{displaymath}
  oder $P \setminus O \in \mathcal{F}$.
  Im zweiten Fall ist damit insbesondere $P \setminus O \in \mathfrak{A}$ und analog zum ersten Teil folgt für alle $x \in O$
  \begin{displaymath}
    \pi_{P \setminus O}(f(x)) = 0
  \end{displaymath}
  und damit 
  \begin{displaymath}
    f(O) = \pi_{P \setminus O}^{-1}(\{0\}) \cap f(P).
  \end{displaymath}
  Im anderen Fall gilt zudem für alle $M \in \mathcal{F}$
  \begin{displaymath}
    M \cap (P \setminus O) \neq \emptyset
  \end{displaymath}
  und damit 
  \begin{displaymath}
    f(O) = f(P),
  \end{displaymath}
  denn in $O$ existiert zu jedem $A \in \mathfrak{A}$ ein $x_1 \in O$ mit $f_A(x) = 1$ sowie ein $x_2 \in O$ mit$f_A(x) = 0$

  Damit ist Lemma \ref{lem:openmap} anwendbar und impliziert, dass $f$ eine offene Abbildung auf ihr Bild ist.
  Folglich ist $f$ ein Homöomorphismus auf $f(P)$.

  Jeder endliche diskrete Raum ist ultrafilterkompakt, da dort die einzigen Ultrafilter, die Einpunktfilter sind, welche immer konvergieren.
  Es ist daher $P$ nach Satz \ref{lem:uftprod} als Produkt ultrafilterkompakter Räume wiederum ultrafilterkompakt.
  Es folgt, dass $F = f(P)$ als homöomorphes Bild eines ultrafilterkompakten Raumes ebenso ultrafilterkompakt ist.

  Wendet man nun den \textsc{Áscoli}-Satz bezüglich Ultrafilterkompaktheit auf $F$ an, wobei $\mathfrak{A}$ als diskreter topologischer Raum und $\mathbf{2}$ als diskreter metrischer Raum betrachtet werden, so ist Bedingung (a) von Satz \ref{thm:ascoliPIT} erfüllt.
  Andererseits ist die Aussage $(b, \beta)$ nicht erfüllt:
  Es ist zwar $p := (1)_{A \in \mathfrak{A}} \in \mathbf{2}^\mathfrak{A}$ im Abschluss von $F$ in $\mathbf{2}^\mathfrak{A}$ enthalten, denn für eine beliebige Umgebung $U_p \subseteq \mathbf{2}^\mathfrak{A}$ von $p$, existieren Subbasiselemente $B_{A_1},\dots,B_{A_n}$ mit 
  \begin{displaymath}
    p \in \bigcap_{i=1}^n B_{A_i} \subseteq U_p,
  \end{displaymath}
  wobei $B_{A_i} = \prod_{A \in \mathfrak{A}} R_{A}$ mit  $R_{A_i} = \{1\}$ und $R_A = \mathbf{2}$ sonst gelte.
  Dies wiederum impliziert
  \begin{displaymath}
    \bigcap_{i=1}^n B_{A_i} \cap F \neq \emptyset,
  \end{displaymath}
  denn für 
  \begin{displaymath}
    x \in \bigcap_{i=1}^n A_i \in \mathcal{F}
  \end{displaymath}
  gilt $f_{A_i}(x) = 1$, weil letzlich offen-abgeschlossene Mengen sowie Filter die endliche Durchschnittseigenschaft besitzen.
  
  Andererseits ist $p$ jedoch nicht in $F$ enthalten. 
  Gälte nämlich $p \in F$, so existiert ein $x \in P$ mit $f_A(x) = 1$ für alle $A \in \mathfrak{A}$, was gleichbedeutend ist mit $x \in \bigcap_{A \in \mathfrak{A}} A$.
  Dies steht jedoch im Widerspruch dazu, dass nach ($\ast$) der Schnitt über alle in $\mathcal{F}$ enthaltenen offen-abgeschlossenen Mengen leer ist.
  Also gilt $F \neq \overline{F}$, ein Widerspruch zum \textsc{Áscoli}-Satz bezüglich Ultrafilterkompaktheit.
  Die Annahme, dass $\mathbf{2}^I$ nicht kompakt ist, muss also verworfen werden.

  $(2)\Rightarrow(1)$: \PIT impliziert nach Satz \ref{thm:uftcompact} die Übereinstimmung des gewöhnlichen Kompaktheitsbegriffes mit dem Begriff der Ultrafilterkompaktheit. Somit folgt (1) direkt aus Satz \ref{thm:ascoliPIT}.
\end{proof}

Ähnliches gilt nun auch für den nächsten Kompaktheitsbegriff der \textsc{Tychonoff}-Kompaktheit wie mit den nächsten Aussagen gezeigt werden soll.

\begin{thm}
  Äquivalent sind:
  \begin{enumerate}
    \item[(1)] Der \textsc{Áscoli}-Satz bezüglich \textsc{\textsc{Tychonoff}}-Kompaktheit.
    \item[(2)] \PIT.
  \end{enumerate}
\end{thm}

\begin{proof}
  (1) $\Rightarrow$ (2): Es reicht aus zu zeigen, dass (1) bereits impliziert, dass die Kantor-Würfel $\mathbf{2}^I$ im gewöhnlichen Sinne kompakt sind.
  Angenommen, es existiert eine Menge $I$, sodass $P:= \mathbf{2}^I$ nicht kompakt ist.
  Man betrachte $I$ als einen mit der diskreten Topologie ausgestatteten topologischen Raum und $\mathbf{2}$ als diskreten metrischen Raum.
  Dann existiert ein Filter $\mathcal{F}$ auf $P$ ohne Häufungspunkt.

  Da das Intervall $[0,1]$ ausgestattet mit euklidischer Topologie insbesondere ein $T1$-Raum ist, sind Einpunktmengen abgeschlossen.
  Betrachtet man nun $P$ als Teilraum des \textsc{Hilbert}-Würfels $[0,1]^I$, so folgt nach Proposition \ref{prop:cartesianclosed}, dass $P$ abgeschlossen im \textsc{Hilbert}-Würfel ist.
  Definitionsgemäß ist also $P$ \textsc{Tychonoff}-kompakt.

  Wie aus dem Beweis von Satz \ref{thm:ultrafiltercompact} folgt, ist $P$ homöomorph zu $F:= f(P)$, wobei $f$ die auf der Menge der offen-abgeschlossenen Elemente von $\mathcal{F}$ definierte Funktion aus Gleichung (\ref{eq:clopenmap}) sei.
  Somit ist $F$ ebenfalls \textsc{Tychonoff}-kompakt.

  Wendet man nun den \textsc{Áscoli}-Satz bezüglich \textsc{Tychonoff}-Kompaktheit auf $F$ an, so folgt anlog zu Satz \ref{thm:ultrafiltercompact}, dass zwar die Bedingung (a) gilt aber die Aussage (b, $\beta$) nicht erfüllt ist.

  (2) $\Rightarrow$ (1): \PIT impliziert die Übereinstimmung des gewöhnlichen Kompaktheitsbegriffes mit dem Begriff der \textsc{Tychonoff}-Kompaktheit. Somit folgt (1) direkt aus Satz \ref{thm:ascoliPIT}.
\end{proof}

Ähnlich gilt für den nächsten Kompaktheitsbegriff

\begin{thm}
  Äquivalent sind:
  \begin{enumerate}
    \item[(1)] Der \textsc{Áscoli}-Satz bezüglich \textsc{Alexandroff}-\textsc{Urysohn}-Kompaktheit.
    \item[(2)] $\mathbf{AC}$.
  \end{enumerate}
\end{thm}

\begin{proof}
  $(1) \Rightarrow (2)$: Sei $X$ eine Menge und $\mathbf{2}$ die zweielementige Menge $\{0,1\}$ und seien beide mit der diskreten Topologie ausgestattet. 
  Die diskrete Topologie lässt sich auch als die von der \textit{diskreten Metrik} induzierte Topologie auffassen, somit kann $\mathbf{2}$ auch als metrischer Raum betrachtet werden. 

  Es sei $F := C(X, \mathbf{2})$.
  Dann gilt $C(X, \mathbf{2}) = \mathbf{2}^X$, da $X$ nach Voraussetzung diskret ist.
  Da $\mathbf{2} = F(x)$ für alle $x \in X$ als endlicher Raum keine unendliche Teilmenge besitzt, folgt sofort die \textsc{Alexandroff}-\textsc{Urysohn}-Kompaktheit von $F(x)$.
  Nach Definition ist $\mathbf{2}^X$ als Gesamtraum abgeschlossen in jeder Topologie, also insbesondere auch in der punktweisen Topologie $\tau_p$.
  Zuletzt ist $F$ auch gleichgradig stetig, da $X$ diskret und somit Einpunktmengen offen sind, d.h. für alle $x \in X$ und $\varepsilon > 0$ wähle man $\{x\}$ als offene Umgebung in $X$. Dann gilt für alle $f \in F$, aufgrund der Definitheit der Metrik für  $d(f(x),f(x)) = 0 \leq \varepsilon$.
  Damit sind die Bedingungen (b) des \textsc{Áscoli}-Satzes erfüllt und es folgt, dass $\mathbf{2}^X$ \textsc{Alexandroff}-\textsc{Urysohn}-kompakt ist. Dies ist äquivalent zum Auswahlaxiom.

  $(2)\Rightarrow(1)$: Das Auswahlaxiom impliziert die Übereinstimmung des gewöhnlichen Kompaktheitsbegriffes mit dem Begriff der \textsc{Alexandroff}-\textsc{Urysohn}-Kompaktheit. Somit folgt (1) direkt aus Satz \ref{thm:ascoliPIT}.
\end{proof}

Wie die vorangehenden Ausführungen gezeigt haben, versagt der \textsc{Áscoli}-Satz bei allen vorgestellten Versionen von Kompaktheit. Es soll im Folgenden untersucht werden, inwiefern eine eingeschränkte Formulierung des \textsc{Áscoli}-Satzes die Anforderungen senkt.

\begin{defn}[Klassischer \textsc{Áscoli}-Satz] 
  \label{def:klassisch}
  Für eine Funktionenmenge $F$ stetiger Abbildungen $f \colon \R \to \R$ sind die folgenden Bedingungen äquivalent:
  \begin{itemize}
    \item[(a)] Jede Folge $(f_n)_{n \in \N}$ in $F$ besitzt eine Teilfolge $(f_{(\nu(n))})_{n \in \N}$, die stetig gegen eine nicht notwendig in $F$ liegende Funktion $g$ konvergiert. Das bedeutet:
      \begin{addmargin}[2em]{2em}% 1em left, 2em right
        Für alle $x \in \R$ und alle $(x_n)_{n \in \N} \in \R^\N$ gilt: \\
        Aus $\lim_{n \to \infty} x_n = x$ folgt $\lim_{n \to \infty} f_{\nu(n)}(x_n) = g(x)$.
      \end{addmargin}
    \item[(b)]
      \begin{itemize}
        \item[($\alpha$)] Für alle $x \in \R$ ist die Menge $F(x) = \{ f(x) \mid f \in F\}$ beschränkt.
        \item[($\beta$)] Es ist $F$ gleichgradig stetig.
      \end{itemize}
  \end{itemize}
\end{defn}

Dass die in Definition \ref{def:klassisch} auftretende Grenzfunktion $g$ eindeutig bestimmt ist, sieht man über ein Folgenmischungsargument ein:
Sind nämlich $(x_n)_{n \in \N}$ und $(y_n)_{n \in \N}$ zwei Folgen in $\R$ mit Grenzwert $x$, so lässt sich eine dritte Folge $(z_n)_{n \in \N}$ konstruieren mit $z_{2n-1} := x_{2n-1}$ und $z_{2n} := y_{2n}$.
Diese konvergiert weiterhin gegen $x$ und es gilt 
\begin{displaymath}
   \lim_{n \to \infty} f_{n}(x_n) 
  =\lim_{n \to \infty} f_{2n-1}(x_{2n-1}) 
  =\lim_{n \to \infty} f_{2n-1}(z_{2n-1}) 
  =\lim_{n \to \infty} f_n(z_n) = g(x)
\end{displaymath}
und eine für $(y_n)_{n \in \N}$ analoge Gleichung, da jede Teilfolge einer konvergenten Folge gegen denselben Grenzwert konvergiert.

Es sollen zunächst einige Eigenschaften von Funktionenfolgen in Bezug auf die stetige Konvergenz festgehalten werden:

\begin{prop}
  \label{prop:stetigkonv}
  \begin{itemize}
    \item[(1)] Falls $(f_n)_{n \in \N}$ stetig gegen $g$ konvergiert, so konvergiert $(f_n)_{n \in \N}$ auch punktweise gegen $g$.
    \item[(2)] Falls $(f_n)_{n \in \N}$ stetig gegen $g$ konvergiert, so ist auch $g$ stetig.
    \item[(3)] Falls $(f_n)_{n \in \N}$ lokal gleichmäßig gegen $g$ konvergiert, so konvergiert $(f_n)_{n \in \N}$ auch stetig gegen $g$.
  \end{itemize}
\end{prop}

\begin{proof}
  (1): 
  Dies folgt direkt aus der Definition unter Betrachtung konstanter Folgen $(x_n)_{n \in \N}$ mit $x_n = x$ für alle $n \in \N$.

  (2): 
  Es soll die Stetigkeit der Grenzfunktion $g$ mittels Folgenstetigkeit gezeigt werden. 
  Es sei dazu $(x_n)_{n \in \N}$ eine Folge in $\R$ mit Grenzwert $x$ und $\varepsilon > 0$ gegeben.
  Nach (1) konvergiert $(f_n)_{n \in \mathbb{N}}$ punktweise.
  Daraus lässt sich nun induktiv eine Teilfolge $(f_{n_k})_{k \in \mathbb{N}}$ gewinnen mit
  \begin{displaymath}
    | f_{n_k}(x_k) - f(x_k) | \leq \frac{\varepsilon}{2}, \text{ für alle $k \in \N$},
  \end{displaymath}
  denn aufgrund der punktweisen Konvergenz existiert zu $x_1$ ein $n_1$, sodass
  \begin{displaymath}
    | f_{n}(x_1) - f(x_1) | \leq \frac{\varepsilon}{2}, \text{ für alle $n \geq n_1$}.
  \end{displaymath}
  Analog findet man für $x_k$ ein $n_k > n_{k-1}$ mit einer analogen Eigenschaft.

  Die so konstruierte Teilfolge besitzt nach wie vor denselben Grenzwert
  \begin{displaymath}
    \lim_{k \to \infty} f_{n_k}(x_k) = f(x).
  \end{displaymath}
  Man sieht dies ein, indem man eine neue Folge $(y_m)_{m \in \N}$ definiert über $y_m:=x_1$ für $1 \leq m \leq n_1$ und $y_m:=x_k$ für $n_{k-1} < m \leq n_{k}$ für $k > 1$, sodass $\lim_{m \to \infty} y_m = x$.
  Damit gilt nun
  \begin{displaymath}
    \lim_{m \to \infty} f_m(y_m) = f(x)
  \end{displaymath}
  aufgrund der stetigen Konvergenz der Folge $(f_n)_{n \in \N}$ und somit
  \begin{displaymath}
    \lim_{k \to \infty} f_{n_k}(x_{k}) 
    = \lim_{k \to \infty} f_{n_k}(y_{n_k}) 
    = f(x).
  \end{displaymath} 

  Es existiert also ein $n_\varepsilon \in \N$, sodass für alle $n_k \geq n_\varepsilon$ gilt
  \begin{displaymath}
    | f_{n_k}(x_k) - f(x) | \leq \frac{\varepsilon}{2}.
  \end{displaymath}
  Daraus folgt nun unter Anwendung der Dreiecksungleichung für alle $n \geq n_\varepsilon$
  \begin{displaymath}
    | f(x_k) - f(x) |
    \leq
    | f_{n_k}(x_k) - f(x_k) | + | f_{n_k}(x_k) - f(x) | 
    \leq \varepsilon.
  \end{displaymath}

  (3): 
  Es sei $(x_n)_{n \in \N}$ eine Folge in $\R$ mit Grenzwert $x$.
  Des Weiteren sei $\varepsilon > 0$ gegeben.
  Da nach Voraussetzung $(f_n)_{n \in \N}$ lokal gleichmäßig gegen $f$ konvergiert existiert eine Umgebung $U$ von $x$, sodass ${f_n}_{| U}$ gleichmäßig konvergiert.
  Es existiert somit ein $n_\varepsilon$, sodass für alle $y \in U$ gilt 
  \begin{displaymath}
    | f_n(y) - f_n(x) | \leq \frac{\varepsilon}{2} \quad \text{ für alle $n \geq n_\varepsilon$}.
  \end{displaymath}
  Da $(x_n)_{n \in \N}$ voraussetzungsgemäß gegen $x$ konvergiert, existiert ein $N \in \N$, sodass $x_n \in U$ für alle $n \geq N$.
  Zuletzt existiert aufgrund der punktweisen Konvergenz von $(f_n)_{n \in \N}$, welche aus (1) folgt, ein $n_p \in \N$, sodass
  \begin{displaymath}
    | f_n(x) - f(x) | \leq \frac{\varepsilon}{2} \quad \text{ für alle $n \geq n_p$}.
  \end{displaymath}

  Setzt man nun $n':= \max(n_\varepsilon, N, n_p)$, so folgt unter Anwendung der Dreiecksungleichung
  \begin{displaymath}
    | f_n(x_n) - f(x) |
    \leq
    | f_n(x_n) - f(x_n) | + | f(x_n) - f(x) | 
    \leq \varepsilon \text{ für alle $n \geq n'$}. \qedhere
  \end{displaymath}

\end{proof}

\begin{thm}
  \label{thm:classicalascoli}
  Äquivalent sind:
  \begin{enumerate}
    \item[(1)] Der klassische \textsc{Áscoli}-Satz.
    \item[(2)] $\mathbf{CC}(\R)$.
  \end{enumerate}
\end{thm}

\begin{proof}
  (1) $\Rightarrow$ (2):
  Nach \ref{} genügt es zu zeigen, dass jede unbeschränkte Teilmenge $B$ von $\R$ eine unbeschränkte Folge enthält.
  Dazu sei $B \subseteq \R$ unbeschränkt.
  Für ein $b \in B$ lassen sich die konstante Abbildung $f_b \colon \R \to \R, f(x):= b$ und die Menge $F := \{f_b \mid b \in B\}$ definieren.
  Für $x \in \R$ gilt $F(x) = \{f_b(x) \mid b \in B\} = B$ und diese Menge ist nach Voraussetzung unbeschränkt.
  Also ist Teil (b, $\alpha$) des klassischen \textsc{Áscoli}-Satzes verletzt.
  Mit (1) folgt sogleich, dass auch Teil (a) des klassischen \textsc{Áscoli}-Satzes nicht gelten kann und somit die Existenz einer Folge $(f_{b_n})_{n \in \N}$ mit $f_{b_n} = b_n$ ohne stetig konvergente Teilfolge.
  Daraus folgt, dass die Folge $(b_n)_{n \in \N}$ unbeschränkt ist.

  (2) $\Rightarrow$ (1):
  Es sei $F$ eine Menge stetiger Abbildungen $f \colon \R \to \R$.

  (a) $\Rightarrow$ (b,$\alpha$):
  Angenommen (b,$\alpha$) gelte nicht. So existiert ein $x \in \R$, sodass $F(x)$ unbeschränkt ist.
  Nach Definition existiert also für alle $C \in \R$ ein $f \in F$ mit $|f(x)| \geq C$.
  Insbesondere sind also die Mengen 
  \begin{displaymath}
    F_n := \{ f \in F \mid |f(x)| \geq m \}
  \end{displaymath}
  für alle $n \in \N$ nichtleer.
  $\mathbf{CC}(\R)$ impliziert folglich die Existenz einer Folge 
  \begin{displaymath}
    (f_n)_{n \in \N} \in \prod_{n \in \N} F_n.
  \end{displaymath}
  Für obiges $x \in \R$ gilt also insbesondere $|f_{n_k}(x)| \geq n_k \geq n$ für alle $k \in \N$, sodass die Funktionenfolge $(f_n)_{n \in \N}$ in $x$ nicht punktweise, folglich auch nicht stetig konvergiert.
  Dies widerspricht jedoch Bedingung (a).

  (a) $\Rightarrow$ (b,$\beta$):
  Angenommen (b, $\beta$) gelte nicht.
  So existiert ein $x \in \R$ und $\varepsilon > 0$, sodass das Bild keiner offenen Umgebung von $x$ in $U_\varepsilon(f(x))$ enthalten ist.
  Insbesondere gilt also auch, dass für kein $\delta > 0$ gilt, dass
  \begin{displaymath}
    f(U_\delta(x)) \subseteq U_\varepsilon(f(x)).
  \end{displaymath}
  Schließlich gilt für alle $n \in \N$, dass die Mengen
  \begin{displaymath}
    F_n := \{ f \in F \mid \text{Es existiert ein } y \in U_{\frac{1}{n+1}}(x) \text{ mit } f(y) \not\in U_\varepsilon(f(x)) \}
  \end{displaymath}
  nicht leer sind.
  Wie zuvor impliziert $\mathbf{CC}(\R)$ die Existenz einer Folge 
  \begin{displaymath}
    (f_n)_{n \in \N} \in \prod_{n \in \N} F_n.
  \end{displaymath}
  Nimmt man an, dass diese Folge eine stetig konvergente Teilfolge $(f_{\nu(n)})_{n \in \N}$ besitzt, so konvergiert diese nach Proposition \ref{prop:stetigkonv}(a) auch punktweise.
  Zu dem anfangs gegebenen $x$ und $\varepsilon > 0$ existiert also insbesondere ein $N \in \N$, sodass
  \begin{displaymath}
    | f_{\nu(n)}(x) - g(x) | \leq \frac{\varepsilon}{2}.
  \end{displaymath}
  Andererseits existiert nach der Definition von $F_n$ eine Folge $(y_n)_{n \in \N}$ mit Grenzwert $x$ und
  \begin{displaymath}
    | f_{\nu(n)}(x) -  f_{\nu(n)}(y_n) | \geq \varepsilon.
  \end{displaymath}
  Unter Verwendung der inversen Dreiecksungleichung folgt damit für alle $n \geq N$
  \begin{displaymath}
    | f_{\nu(n)}(y_n) - g(x) | 
    \geq 
    | f_{\nu(n)}(y_n) - f_{\nu(n)}(x) | - | f_{\nu(n)}(x) - g(x)  | 
    \geq \frac{\varepsilon}{2}.
  \end{displaymath}
  Dies steht jedoch im Widerspruch zur stetigen Konvergenz von $(f_{\nu(n)})_{n \in \N}$.

  (b) $\Rightarrow$ (a):
  Sei $(f_n)_{n \in \N}$ eine Folge in $F$.
  Des Weiteren sei $(r_n)_{n \in \N}$ eine Abzählung der rationalen Zahlen.
  Im Folgenden soll induktiv eine Folge von geordneten Paaren $(a_n, s_n)_{n \in \N}$ mit $a_n \in \R$ für alle $n \in \N$ und eine Folge $s_n = (g_m^n)_{m \in \N}$ in $F$ definiert werden:
  \begin{enumerate}
    \item 
      Nach $(b)$ ist $F(x)$ für alle $x \in \R$ beschränkt. 
      Somit ist auch 
      \begin{displaymath}
        \{f_n(r_0) \mid n \in \N\} \subseteq F(r_0)
      \end{displaymath}
      beschränkt.
      Man setze nun $a_0 := \liminf_{n \to \infty} f_n(r_0)$.
      Weiterhin definiere man $s_0 = (g_n^0)_{n \in \N}$ induktiv als Teilfolge $(f_{\nu(n)})_{n \in \N}$ von $(f_n)_{n \in \N}$:
      \begin{itemize}

        \item[$a)$] $\nu(0) := \min\{m \in \N \mid |f_m(r_0) - a_0 | < 1\}$

          $\nu(0)$ ist wohldefiniert, denn die Menge auf der rechten Seite der Gleicheit ist nicht leer, da nach Voraussetzung $a_0$ ein Häufungspunkt von $(f_n(r_0))_{n \in \N}$ ist und somit eine gegen $r_0$ konvergente Teilfolge existiert.
          Daher besitzt sie aufgrund der natürlichen Wohlordnung der natürlichen Zahlen als nichtleere Teilmenge ein Minimum.

        \item[$b)$] $\nu(n+1) :=  \min\{m \in \N \mid \nu(n) < m \text{ und }  |f_m(r_0) - a_0| < \frac{1}{n+1}\}$

          Wie in $a)$ sieht man ein, dass $\nu(n+1)$ wohldefiniert ist.
      \end{itemize}
      Folglich ist $s_0:= (g_n^0)_{n \in \N} = (f_{\nu(n)})_{n \in \N}$ eine Teilfolge von $(f_n)_{n \in \N}$ und es gilt $\lim_{n \to \infty}(g_n^0(r_0)) = a_0$.

    \item 
      Seien nun $a_n$ und $s_n = (g_m^n)_{m \in \N}$ definiert.
      Es sei nun $a_{n + 1} := \liminf_{m \to \infty} g_m^n(r_{n+1})$.
      Analog zu $1.$ definiere man nun induktiv $s_{n+1} := (g_m^{n + 1})_{m \in \N}$ als Teilfolge von $s_n = (g_m^n)_{m \in \N}$, sodass $\lim_{m \to \infty} g_m^{n + 1}(r_{n+1}) = a_{n + 1}$.
  \end{enumerate}

  Hierauf aufbauend betrachte man nun die Diagonalfolge $s := (g_n^n)_{n \in \N}$.
  Dann ist $s$ eine Teilfolge von $(f_n)_{n \in \N}$ und kofinal zu jeder der Folgen $s_n$.
  Also konvergiert für jedes $n \in \N$ die Folge $s(r_n) = (g_m^m(r_n))_{m \in \N}$ gegen $a_n$.
  Folglich konvergiert für jedes $x \in \Q$ die Folge $s(x) = (g_m^m(x))_{m \in \N}$.

  Es soll nun gezeigt werden, dass $s$ lokal gleichmäßig auf $\R$ konvergiert.
  Sei dazu $x \in \R$ und $\varepsilon > 0$ gegeben.
  Da $G$ gleichgradig stetig ist, existiert eine offene Umgebung $U$ von $x$, sodass
  \begin{displaymath}
    f(U) \subseteq U_\frac{\varepsilon}{3}(f(x))
  \end{displaymath}
  für alle $f \in G$ gilt.
  Da $\Q$ dicht in $\R$ liegt existiert ein $y \in U \cap \Q$.
  Nach Konstruktion konvergiert für alle $g_m^m \in G$ die Folge $g_m^m(y)$.
  Sie ist damit also insbesondere eine \textsc{Cauchy}-Folge.
  Es existiert also ein $N \in \N$, sodass für alle $m,n \geq N$
  \begin{displaymath}
    | g_m^m(y) - g_n^n(y) | \leq \frac{\varepsilon}{3}
  \end{displaymath}
  gilt.
  Damit gilt für alle $m, n \geq N$
  \begin{displaymath}
    | g_m^m(x) - g_n^n(x) |
    \leq 
    | g_m^m(x) - g_m^m(y) | +  | g_m^m(y) - g_n^n(y) | +  | g_n^n(y) - g_n^n(x) |
    \leq
    \frac{\varepsilon}{3}.
  \end{displaymath}
  Also ist $s$ eine lokal gleichmäßige \textsc{Cauchy}-Folge auf $\R$.
  Da $\R$ vollständig ist, konvergiert $s$ auch lokal gleichmäßig auf $\R$. 
  Aufgrund von Proposition \ref{prop:stetigkonv}(c) konvergiert $s$ auch stetig gegen eine Abbildung $g \colon \R \to \R$ mit $g(r_n) = a_n$ für alle $n \in \N$.
  Daraus folgt $(a)$.
\end{proof}

Zusammenfassend lässt sich also erkennen, dass weder der topologische \textsc{Áscoli}-Satz noch der klassische \textsc{Áscoli} Satz in $\mathbf{ZF}$ gelten. Man kann jedoch eine modifizierte Variante des klassischen \textsc{Áscoli}-Satzes angeben, welcher sich im Rahmen von $\mathbf{ZF}$ beweisen lässt:

\begin{thm}[Modifizierter \textsc{Áscoli}-Satz]
  Für eine Funktionenmenge $F$ stetiger Abbildungen $f \colon \R \to \R$ sind die folgenden Bedingungen äquivalent:
  \begin{itemize}
    \item[(a)] Jede Folge $(f_n)_{n \in \N}$ in $F$ besitzt eine Teilfolge $(f_{(\nu(n))})_{n \in \N}$ die stetig gegen eine nicht notwendig in $F$ liegende Funktion $g$ konvergiert.

    \item[(b)]
      \begin{itemize}
        \item[($\alpha$)] Für alle $x \in \R$ und jede abzählbare Teilmenge $G \subseteq F$ ist die Menge $G(x) = \{ g(x) \mid g \in G \}$ beschränkt.
        \item[($\beta$)] Jede abzählbare Teilmenge von $F$ ist gleichgradig stetig.
      \end{itemize}
  \end{itemize}
\end{thm}

\begin{proof}
  (a) $\Rightarrow$ (b,$\alpha$):
  Sei $G := \{f_n \in F \mid n \in \N\}$ eine abzählbare Teilmenge von $F$.
  Angenommen $G(x)$ sei unbeschränkt für ein $x \in X$.
  Dann lässt sich für alle $n \in \N$
  \begin{displaymath}
    \nu(n) := \min \{ m \in \N \mid n < |f_m(x)| \}
  \end{displaymath}
  definieren, da zu jeder nichtleeren Teilmenge natürlicher Zahlen ein Minimum existiert.
  Dann ist jedoch die Folge $(f_{\nu(n)})_{n \in \N}$ dergestalt, dass keine der Teilfolgen von $(f_{\nu(n)}(x))_{n \in \N}$ konvergiert.
  Die Folge $(f_{\nu(n)})_{n \in \N}$ konvergiert somit in $x$ nicht punktweise also nach Proposition \ref{prop:stetigkonv}(1) auch nicht stetig.
  Dies widerspricht jedoch (1).

  (a) $\Rightarrow$ (b,$\beta$):
  Es sei $G$ die der Folge $(f_n)_{n \in \N}$ in $F$ zugrunde liegende Menge.
  Angenommen $G$ für ein $x \in \R$ nicht gleichgradig stetig.
  Dann existiert ein $\varepsilon > 0$, sodass für alle $\delta > 0$ ein $n \in \N$ und $y \in \R$ existiert mit
  \begin{displaymath}
     |x - y| < \delta \quad\text{und}\quad |f_n(x) - f_n(y)| \geq \varepsilon.
  \end{displaymath}
  Für alle $n \in \N$ lassen sich nun
  \begin{align*}
    \nu(n) &:= \min \{ m \in \N \mid \text{ Es existiert ein } y \in [x - 2^{-n}, x + 2^{-n}] \text{ mit } |f_m(x) - f_m(y)| \geq \varepsilon\}, \\
    g_n &:= f_{\nu(n)} \quad\text{und}\\
    x_n &:= \min\{ y \in [x - 2^{-n}, x + 2^{-n}] \mid |f_m(x) - f_m(y)| \geq \varepsilon \}
  \end{align*}
  definieren.
  Dann gelten
  \begin{displaymath}
    \lim_{n \in \N} x_n = x \quad\text{und}\quad|g_n(x) - g_n(x_n)| \geq \varepsilon
  \end{displaymath}
  für alle $n \in \N$.
  Somit kann keine Teilfolge von $(g_n)_{n \in \N}$ stetig in $x$ konvergieren, was jedoch (1) widerspricht.

  (b) $\Rightarrow$ (a):
  Dies folgt bereits aus der entsprechenden Implikation in Satz \ref{thm:classicalascoli}.
\end{proof}
