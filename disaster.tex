\section{Áscoli-Sätze und das Auswahlaxiom}

Dieser Abschnitt ist eine ausgestaltete Darstellung des Kapitels \textit{Disasters in Topology III: Function Spaces (The Ascoli Theorem)} aus dem Buch von Horst Herrlich.

\begin{defn}
  Es seien $(X,d_X)$ und $(Y,d_Y)$ metrische Räume. Eine Abbildung $f \colon X \to Y$ heißt \textit{gleichmäßig stetig}, wenn gilt:
  \begin{addmargin}[2em]{2em}% 1em left, 2em right
  Zu jedem $\varepsilon > 0$ existiert ein $\delta > 0$, sodass \\
  für alle $x,x' \in X$ mit $d_X(x,x') < \delta$ gilt, dass $d_Y(f(x),f(x')) < \varepsilon$.
  \end{addmargin}
\end{defn}

\begin{defn}
  Es sei $(X,\tau)$ ein topologischer Raum und $(Y,d)$ ein metrischer Raum. Eine Teilmenge $F \subseteq Y^X$ heißt \textit{gleichgradig stetig}, wenn gilt:
  \begin{addmargin}[2em]{2em}% 1em left, 2em right
    Für alle $x \in X$ und $\varepsilon > 0$ existiert ein $V \in \dot{x} \cap \tau$, sodass \\
    für alle $f \in F$ gilt, dass $f(V) \subseteq U_\varepsilon(f(x))$.
  \end{addmargin}
\end{defn}

Für zwei topologische Räume $(X,\tau)$ und $(Y,\sigma)$ bezeichne im Folgenden $C_{co}(X,Y)$ den topologischen Funktionenraum $(C(X,Y),\tau_{co})$ der stetigen Funktionen von $X$ nach $Y$ ausgestattet mit der kompakt-offenen Topologie.

Áscoli-Sätze gelten bezeichnend für Aussagen, die es gestatten Kompaktheit von Teilmengen von Funktionenräumen bezüglich starker Topologien aus der Kompaktheit bezüglich einer schwächeren Topologie unter zusätzlichen Annahmen zu folgern. Dabei existieren unterschiedliche Formulierungen.

Eine topologische Version des Áscoli Satzes lautet:
\begin{defn}[Topologischer Áscoli Satz]
  Sei $(X,\tau)$ ein lokalkompakter Hausdorff-Raum und $(Y,d)$ ein metrischer Raum.
  Für jeden Teilraum $(F,\tau_{co})$ von $C_{co}(X,Y)$ sind äquivalent:
  \begin{enumerate}
    \item[(a)] $F$ ist kompakt.
    \item[(b)] \begin{enumerate}
        \item[($\alpha$)] Für alle $x \in X$ ist die Menge $F(x) = \{f(x) \mid f \in F \}$ kompakt in $Y$.
        \item[($\beta$)]  $F$ ist abgeschlossen in $Y$ bezüglich der Topologie der punktweisen Konvergenz $\tau_p$.
        \item[($\gamma$)] $F$ ist gleichgradig stetig auf $X$.
      \end{enumerate}
  \end{enumerate}
\end{defn}

Es soll nun gezeigt werden, dass die Gültigkeit dieses Satzes äquivalent zum Booleschen Primideal ist.

\begin{thm}
  Äquivalent sind
  \begin{enumerate}
    \item Der topologische Áscoli-Satz.
    \item Der Boolesche Primidealsatz.
  \end{enumerate}
\end{thm}

\begin{proof}
  $(1) \Rightarrow (2)$: Sei $X$ eine Menge und $\mathbf{2}$ die zweielementige Menge $\{0,1\}$ und seien beide mit der diskreten Topologie ausgestattet. 
  Die diskrete Topologie lässt sich auch als die von der \textit{diskreten Metrik} induzierte Topologie auffassen, somit kann $\mathbf{2}$ auch als metrischer Raum betrachtet werden. 

  Es sei $F := C(X, \mathbf{2})$.
  Dann gilt $C(X, \mathbf{2}) = \mathbf{2}^X$, da $X$ nach Voraussetzung diskret ist.
  Es folgt, dass $\mathbf{2} = F(x)$ für alle $x \in X$ als endlicher Raum kompakt ist.
  Nach Definition ist $\mathbf{2}^X$ als Gesamtraum abgeschlossen in jeder Topologie, also insbesondere auch in der punktweisen Topologie $\tau_p$.
  Zuletzt ist $F$ auch gleichgradig stetig, $X$ diskret und somit Einpunktmengen offen sind, d.h. für alle $x \in X$ und $\varepsilon > 0$ wähle man $\{x\}$ als offene Umgebung in $X$. Dann gilt für alle $f \in F$, aufgrund der Definitheit der Metrik für  $d(f(x),f(x)) = 0 \leq \varepsilon$.
  Damit sind die Bedingungen (b) des Áscoli-Satzes erfüllt und es folgt, dass $\mathbf{2}^X$ kompakt ist. Dies ist äquivalent zum Booleschen Primidealsatz.

  $(2) \Rightarrow (1)$: Es seien $(X,\tau)$ ein lokalkompakter Hausdorff-Raum und $(Y,d)$ ein metrischer Raum.
  Des Weiteren sei $F$ ein Teilraum vom $C_{co}(X,Y)$.

  $(a) \Rightarrow (b, \alpha)$: 
  Nach Voraussetzung ist $F$ ein bezüglich der kompakt-offenen Topologie $\tau_{co}$ kompakter Teilraum von $C(X,Y)$.
  Somit ist er insbesondere bezüglich der Spur der schwächeren Topologie der punktweisen Konvergenz $\tau_p$ ein kompakter Teilraum des Hausdorff-Raumes $Y^X$.
  Die Topologie der punktweisen Konvergenz lässt sich als Initialtopologie bezüglich der kanonischen Projektionen $p_x : Y^X \to Y, f \mapsto f(x)$ beschreiben.
  Als stetiges Bild eines Kompaktums ist somit auch $p_x(F) = F(x)$ kompakt in $(Y,\tau_d)$.

  $(a) \Rightarrow (b, \beta)$:  
  Als kompakter Teilraum eines Hausdorff-Raumes ist $F$ bezüglich $\tau_p$ auch abgeschlossen in $Y^X$.

  $(a) \Rightarrow (b, \gamma)$:
  Es soll gezeigt werden, dass $F$ gleichgradig stetig ist.
  Sei dazu $x \in X$ und $\varepsilon > 0$.
  Für jedes $f \in F$ ist die Menge
  \begin{displaymath}
    B_f := B_{\frac{\varepsilon}{2}} = \left\{ y \in Y \mid d(f(x),y) < \tfrac{\varepsilon}{2}\right\} 
  \end{displaymath}
  offen bezüglich der von der Metrik $d$ auf $Y$ induzierten Topologie.
  Da $F \subseteq C(X,Y)$, ist das Urbild $f^{-1}(B_f)$ ebenfalls offen bezüglich $\tau$.
  Nach Voraussetzung ist $X$ lokalkompakt.
  Daher existiert eine kompakte Umgebung $K_f$ von $x$ mit $K_f \subseteq f^{-1}(B_f)$.
  Damit gilt $f(K_f) \subseteq B_f$ und es definiert
  \begin{displaymath}
    U_f := F \cap (K_f, B_f) = \left\{g \in F \mid g(K_f) \subseteq B_f \right\} \tag{$\ast$}
  \end{displaymath}
  eine bezüglich der Teilraumtopologie auf $F$ offene Umgebung von $f$.
  Man betrachte nun die Evaluationsabbildung $\omega: X \times F \to Y, (y,g) \mapsto g(y)$.
  Unter Berücksichtigung von ($\ast$) folgt $\omega(K_f \times U_f) \subseteq B_f$.
  Es definiere $\mathcal{C}$ die Menge aller Tripel $(f,K,U)$, wobei $f \in F$, $K$ eine Umgebung von $x$ in $X$ und $U$ eine offene Umgebung von $f$ in $F$ mit $\omega(K \times U) \subseteq B_f$ sei.
  Dann liefert
  \begin{displaymath}
    \mathfrak{U} = \{ U \subseteq F \mid \text{es existieren $f \in F$, $K \subseteq X$ mit $(f,K,U) \in \mathcal C$} \}
  \end{displaymath}
  eine offene Überdeckung von $F$, da für alle $f \in F$ nach Konstruktion $(f,K_f,U_f) \in \mathfrak{U}$ gilt.
  Nach Voraussetzung ist $F$ kompakt, daher existiert eine endliche Teilüberdeckung durch $U_1,\dots,U_n \in \mathfrak{U}$.
  Für alle $i = 1,\dots,n$ wähle man nun $f_i \in F$ und $K_i \subseteq X$ mit $(f_i, K_i, U_i) \in \mathcal{C}$.
  Des Weiteren definiere man $U := \bigcap_{i = 1}^n K_i$.
  Es ist $U$ eine (nicht notwendig offene) Umgebung von $x \in X$.
  Nun soll gezeigt werden, dass für diese Wahl von $U$ die Voraussetzung der gleichgradigen Stetigkeit erfüllt ist, also dass für alle $f \in F$ folgt, dass $f(U) \subseteq U_\varepsilon(f(x))$ gilt.
  Für jedes $f \in F$ existiert aufgrund der Überdeckungseigenschaft ein $i = 1,\dots,n$ mit $f \in U_i$.
  Sei des Weiteren $y \in U$ gegeben.
  Dann gilt
  \begin{displaymath}
    f(y) = \omega(y,f) \in \omega(U \times U_i) \subseteq \omega(K_i \times U_i) \subseteq B_{f_i},
  \end{displaymath}
  also $d(f_i(x),f(y)) < \tfrac{\varepsilon}{2}$.
  Insbesondere gilt für $x \in U$, da nach Voraussetzung $x \in K_i$ für alle $i = 1,\dots,n$, dass $d(f_i(x),f(x)) < \tfrac{\varepsilon}{2}$.
  Unter Verwendung der Dreiecksungleichung folgt letztlich
  \begin{displaymath}
    d(f(x),f(y)) \leq d(f(x),f_i(x)) + d(f_i(x),f(y)) < \frac{\varepsilon}{2} + \frac{\varepsilon}{2} = \varepsilon,
  \end{displaymath}
  und damit $f(U) \subseteq U_\varepsilon(f(x))$.
  Folglich ist $F$ gleichgradig stetig.

  $(b) \Rightarrow (a)$
  Nach Voraussetzung ist $F(x)$ als Teilraum von $Y^X$ kompakt und zudem hausdorff'sch.
  Der Tychonoff-Satz für Kompaktheit besagt, dass dann auch $\prod_{x \in X} F(x)$ kompakt ist.
  Die Voraussetzung, dass $F$ abgeschlossen in $Y^X$ bezüglich der Produkttopologie $\tau_p$ ist, liefert, dass $F$ auch abgeschlossen im Teilraum $\prod_{x \in X} F(x)$ ist.
  Als abgeschlossene Teilmenge eines Kompaktums ist somit $F$ auch kompakt bezüglich der Topologie der punktweisen Konvergenz.
  Es fehlt zu zeigen, dass $F$ ebenfalls bezüglich der stärkeren, kompakt-offenen Topologie $\tau_{co}$ kompakt ist.
  Hierfür soll die Inklusion $\tau_{co} \subseteq \tau_p$ nachgewiesen werden.
  Dazu betrachte man ein Subbasiselement $V = (K,U) \cap F$ der kompakt-offenen Topologie auf dem Teilraum $F$ und ein $f \in V$.
  Es gilt nun zu zeigen, dass $V$ auch in $\tau_p$ offen ist.
  Da $f(K) \subseteq U$ und $U \in \sigma$ gilt, folgt für alle $x \in K$
  \begin{displaymath}
    r_x = \inf\{d(f(x),y) \mid y \in (Y \setminus U) \} > 0.
  \end{displaymath}
  Dann ist $U_x = \{ z \in X \mid d(f(x),f(z)) < \tfrac{r_x}{2}\}$ eine offene Umgebung von $x \in X$.
  Denn für alle $z \in U_x$ mit $\varepsilon := \tfrac{r_x}{2} - d(f(x),f(z)) > 0$ gilt, dass aufgrund der Stetigkeit von $f$ eine Umgebung $U' \in \tau$ von $z$ existiert, sodass $f(U') \subseteq U_\varepsilon(f(z))$ gilt.
  Daraus folgt für alle $u \in U'$ mit der Dreiecksungleichung
  \begin{align*}
    d(f(x),f(u)) &\leq d(f(x),f(z)) + d(f(z),f(u)) \\
                 &< \varepsilon + d(f(z),f(u)) = \frac{r_x}{2} - d(f(x),f(z)) + d(f(x),f(z)) =  \frac{r_x}{2}
  \end{align*}
  und damit schließlich $U' \subseteq U_x$, also die Offenheit von $U_x$ in $X$.
  Zudem bildet $\mathfrak{U} := \{U_x \mid x \in K\}$ eine offene Überdeckung von $K$. 
  Nach Voraussetzung ist $K$ kompakt, daher existieren $x_1,\dots,x_n \in K$, sodass $K \subseteq \bigcup_{i=1}^n U_{x_i}$.
  Es sei $r:= \min\{r_{x_1},\dots,r_{x_n}\}$. 
  Damit ergibt sich für alle $x \in K$ und alle $y \in (Y \setminus U)$ die Ungleichung $d(f(x),y) \geq \tfrac{r}{2}$.
  Ist $x \in K$ und $d(f(x),y) < \tfrac{r}{2}$ so folgt daraus $y \in U$.
  Da $F$ nach Voraussetzung gleichgradig stetig auf $X$ ist, existiert für alle $x \in X$ eine offene Umgebung $W$ von $x$ in $X$, sodass für alle $g \in F$ und alle $z \in W$ gilt, dass $d(g(x),g(z)) < \tfrac{r}{4}$.
  Betrachtet man nun die auf obige Weise erzeugte Menge $\mathcal{C}$ aller geordneten Paare $(x,W)$, so bildet
  \begin{displaymath}
    \mathfrak{W} := \{W \subseteq X \mid \text{Es existiert ein $x \in K$ mit $(x,W) \in \mathcal{C}$} \}
  \end{displaymath}
  eine offene Überdeckung von $K$.
  Aufgrund der Kompaktheit von $K$ lässt sich eine endliche Teilüberdeckung $W_1,\dots,W_m \in \mathfrak{W}$ von $K$ auswählen.
  Für alle $i = 1,\dots,m$ wähle man entsprechend der Definition $\mathfrak{W}$ ein $x_i$ mit $(x_i,W_i) \in \mathcal{C}$.
  Dann ist
  \begin{displaymath}
    B_f := \{ g \in F \mid d(f(x_i),g(x_i)) < \tfrac{r}{4} \text{ für } i = 1,\dots,m\}
  \end{displaymath}
  eine offene Umgebung von $f$ bezüglich $\tau_p$.
  Es soll nun gezeigt werden, dass $B_f \subseteq V$ gilt.
  Dann ist nämlich $V = \bigcup_{f \in V} B_f$ als Vereinigung offener Mengen offen bezüglich $\tau_p$ und die Behauptung $\tau_{co} \subseteq \tau_p$ folgt, da $V$ beliebig war.
  Sei dazu also $g \in B_f$.
  Für jedes $x \in K$ existiert ein $i \in \{1,\dots,m\}$ mit $x \in W_i$.
  Nach Konstruktion der $W_i$ gilt $d(g(x_i),g(x)) < \tfrac{r}{4}$.
  Da $g \in B_f$ gilt zudem die Ungleichung $d(f(x_i),g(x_i)) < \tfrac{r}{4}$.
  Daraus folgert man
  \begin{displaymath}
    d(f(x_i),g(x)) \leq d(f(x_i),g(x_i)) + d(g(x_i),g(x)) < \frac{r}{2},
  \end{displaymath}
  was wiederum $g(x) \in U$ und damit $g(K) \subseteq U$ impliziert. 
  Also ist $g \in (K,U)$, woraus $g \in V$ folgt.
  Damit ist der Beweis vollständig.
\end{proof}

Es soll nun untersucht werden, inwiefern der Áscoli Satz auch unter Verwendung anderer Kompaktheitsbegriffe seine Gültigkeit behält.

\begin{lem}
  \label{lem:filterclusterpoint}
  Es sei $(X,\tau)$ ein topologischer Raum, sodass die darin offen-abgeschlossenen Mengen einen Basis für die Topologie $\tau$ bilden.

  Ist $\mathcal{F}$ ein Filter ohne Häufungspunkt, so gilt für die Menge $\mathfrak{A} \subseteq \mathcal{F}$ aller offen-abgeschlossenen Elemente 
  \begin{displaymath}
    \bigcap_{F \in \mathfrak{A}} F = \emptyset.
  \end{displaymath}
\end{lem}

\begin{proof}
  Nach Voraussetzung besitzt $\mathcal{F}$ keinen Häufungspunkt.
  Dies bedeutet, dass
  \begin{displaymath}
    \bigcap_{F \in \mathcal{F}} \overline F = \emptyset.
  \end{displaymath}
  Für alle $x \in X$ gilt nun, da $x$ kein Häufungspunkt von $\mathcal{F}$ ist, dass ein $F \in \mathcal{F}$ existiert, sodass $x \not\in \overline{F}$.
  Da die offen-abgeschlossenen Mengen eine Basis für $\tau$ bilden, existiert eine Umgebung $U_x \in \mathfrak{A}$ von $x$, mit 
  \begin{displaymath}
    U \cap F = \emptyset.
  \end{displaymath}
  Damit ist $X \setminus U \supset F$ und wegen des Abschlusses gegen Obermengenbildung
  \begin{displaymath}
    X \setminus U \in \mathcal{F}.
  \end{displaymath}
  Es existiert also eine offen-abgeschlossene Menge in $\mathcal{F}$, die $x$ nicht enthält. 
  Dies impliziert
  \begin{displaymath}
    x \not\in \bigcap_{F \in \mathfrak{A}} F
  \end{displaymath}
  Da $x$ beliebig gewählt war folgt 
  \begin{displaymath}
   \bigcap_{F \in \mathfrak{A}} F = \emptyset
  \end{displaymath}
  und damit die Behauptung.
\end{proof}

\begin{thm}
  Äquivalent sind:
  \begin{enumerate}
    \item Der Áscoli Satz bezüglich Ultrafilter-Kompaktheit.
    \item Der Boolesche Primidealsatz.
  \end{enumerate}
\end{thm}

\begin{proof}
  $(1)\Rightarrow(2) $: Es reicht aus zu zeigen, dass $(1)$ bereits impliziert, dass die Kantor-Würfel $\mathbf{2}^I$ kompakt sind.
  Angenommen, es existiert eine Menge $I$, sodass $P:= \mathbf{2}^I$ nicht kompakt ist.
  Man betrachte $I$ als einen mit der diskreten Topologie ausgestatteten topologischen Raum und $\mathbf{2}$ als diskreten metrischen Raum.
  Dann existiert ein Filter $\mathcal{F}$ auf $P$ ohne Häufungspunkt.

  %Die Menge aller offen-abgeschlossenen Elemente von $\mathcal{F}$ besitzt zunächst eine endliche Durchschnittseigenschaft.
  Der mit der Produkttopologie ausgestattete Raum $\mathbf{2}^I$ besitzt eine Basis aus offen-ab\-ge\-schlossenen Mengen.
  Dies folgt daraus, dass bereits jedes Subbasiselement $S$ per Definition offen ist und, da $\mathbf{2}$ diskret ist, auch ein offenes Komplement besitzt.
  Damit ist $S$ aber als Komplement einer offenen Menge definitionsgemäß abgeschlossen.
  Insgesamt ist $S$ also offen-abgeschlossen.
  Da die Menge der der offen-abgeschlossenen Mengen eine endlichen Durchschnittseigenschaft besitzt, ist auch jedes von der Subbasis erzeugte Basiselement eine offen-abgeschlossene Menge.
  Damit lässt sich nun Lemma \ref{lem:filterclusterpoint} auf $\mathcal{F}$ anwenden und es folgt für die Menge $\mathfrak{A}$ aller offen-abgeschlossenen Elemente von $\mathcal{F}$
  \begin{displaymath}
   \bigcap_{F \in \mathfrak{A}} F = \emptyset.
  \end{displaymath}

  Damit lässt sich nun für alle $A \in \mathfrak{A}$ eine Funktion
  \begin{displaymath}
    f_A \colon P \to \mathbf{2}, \quad
    f_A(x) := 
    \begin{cases}
      1 &\text{, falls } x \in A \\
      0 &\text{, falls } x \not\in A
    \end{cases}
  \end{displaymath}
  Da alle $A$ offen-abgeschlossen sind, folgt sofort die Stetigkeit von $f_A$ für alle $A \in \mathfrak{A}$.
  Somit induziert die Familie $(f_A)_{A \in \mathfrak{A}}$ eine stetige Abbildung $f \colon P \to \mathbf{2}^I$.

\end{proof}

