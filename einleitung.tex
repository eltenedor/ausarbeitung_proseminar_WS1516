\section*{Einleitung}\index{Einleitung}
\addcontentsline{toc}{section}{Einleitung}\index{Einleitung}

Hier wird die Einleitung stehen. Dabei sollten Sie einen Kurzüberblick des Inhalts Ihrer Arbeit geben. Es ist sinnvoll, eingangs die bearbeitete Fragestellung zu erläutern, welches Ziel Sie und Ihr Betreuer hatten und welche Ergebnisse Sie schließlich erzielt haben.

Es ist hilfreich, auf die Struktur Ihrer Arbeit einzugehen, zum Beispiel indem Sie die einzelnen Kapitel kurz zusammenfassen. Stellen Sie auch klar heraus, welche Resultate schon aus der Literatur bekannt sind und welche Ergebnisse eigene Beiträge darstellen.

Die Tücke bei \AC. Es versteckt sich in vielen Aussagen, die als allgemeingültig hingenommen werden. Zudem taucht das Auswahlaxiom, oder schwächere und dennoch nicht in \ZF beweisbare Aussagen in vielen Beweisen, so wie in der hier dargestellten \textsc{Áscoli}-Sätzen nur an einem einzigen Punkt auf, ist also leicht zu übersehen.

