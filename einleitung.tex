\chapter*{Einleitung}\index{Einleitung}
\addcontentsline{toc}{chapter}{Einleitung}\index{Einleitung}

\begin{addmargin}[2em]{2em}% 1em left, 2em right
  \textit{Zu jeder Menge $X$ von nichtleeren, zueinander disjunkten Mengen gibt es eine Menge, die von jedem Element von $X$ genau ein Element enthält.} 
  \flushright(Auswahlaxiom)
\end{addmargin}

Das Auswahlaxiom \AC ist eines der kontroversesten Axiome der Mengenlehre.
Ein Grund dafür liegt wohl darin, dass sich aus der anscheinlich intuitiven Aussage von \AC, infolge des inkonstruktiven Charakters dieses Axioms oftmals unintuitive Aussagen folgern lassen.
Bekannte Beispiele hierfür liefern der Wohlordnungssatz oder das \textsc{Banach}-\textsc{Tarski}-Paradoxon.
Trotzdem bildet \AC einen wichtigen Bestandteil mathematischer Theorien, wie auch der Theorie topologischer Räume.
In dieser Arbeit stehen basierend auf \cite{herrlich2006axiom} speziell topologische Funktionenräume im Vordergrund unter dem Aspekt der Gültigkeit verschiedener Versionen von \textsc{Ascoli}-Sätzen.
Dabei handelt es sich um Kompaktheitsaussagen, die es ermöglichen unter geeigneten Zusatzannahmen aus der Kompaktheit einer Teilmenge bezüglich einer schwachen Topologie die Kompaktheit bezüglich einer stärkeren Topologie zu folgern.

In diesem Text werden Grundlagen, wie sie in Rahmen einer Vorlesung zu allgemeiner Topologie vermittelt werden, vorausgesetzt.
Sämtliche unbewiesene Aussagen dieses Typus lassen sich jedoch problemlos in der Lehrbuchliteratur auffinden.
Für diesen Text werden hierzu insbesondere \cite{bartsch2015allgemeine}, \cite{ebbinghaus2003einfuhrung}, \cite{kelley1975general} und \cite{preuss1972topologische} empfohlen.

Im Einführungskapitel dieser Arbeit werden grundlegende Aussagen der Topologie und Mengenlehre bewiesen, die es ermöglichen den Charakter ihrer Beziehung zu \AC und dem \textsc{Boole}schen Primidealsatz (\PIT) zu beschreiben. 
Es werden unterschiedliche, im Rahmen von \ZF zunächst voneinander unabhängige Kompaktheitsbegriffe auf topologischen Räumen eingeführt und anschließend in Relation zueinander gesetzt.
Als zentrale Aussage dieses Kapitels steht ein Satz, welcher \PIT als eine Abschwächung von \AC, mit Kompaktheitsaussagen über Produkttopologien verbindet.

Das zweite Kapitel behandelt die Topologie von Funktionenräumen im Hinblick auf \textsc{Ascoli}-Sätze.
Es werden unterschiedliche Formulierungen dieser Kompaktheitsaussagen vorgestellt und deren Äquivalenz zu Axiomen der Mengenlehre bewiesen.
Insbesondere wird aufgezeigt inwieweit sich \textsc{Ascoli}-Sätze im Rahmen von \ZF beweisen lassen, wenn man alternative Kompaktheitsbegriffe zugrunde legt.

Abschließend fasst ein Fazit die wichtigsten Ergebnisse zusammen. 
