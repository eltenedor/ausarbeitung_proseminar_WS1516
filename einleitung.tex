\section*{Einleitung}\index{Einleitung}
\addcontentsline{toc}{section}{Einleitung}\index{Einleitung}

\begin{addmargin}[2em]{2em}% 1em left, 2em right
  \textit{Zu jeder Menge $X$ von nicht leeren, zueinander disjunkten Mengen gibt es eine Menge, die von jedem Elemnt von $X$ genau ein Element enthält.} 
  \flushright(Auswahlaxiom)
\end{addmargin}

Das Auswahlaxiom \AC ist eines der kontroversesten Axiomen der Mengenlehre.
Es wurde schon 1904 nach der ersten expliziten Formulierung zur \textsc{Zermelo} nur zögernd aufgenommen. 
Ein Grund dafür liegt wohl darin, dass sich aus der anscheinlich intuitiven Vorstellung, die man mit \AC verbinden kann, infolge des inkonstruktiven Charakters dieses Axioms oftmals unintuitive Aussagen folgern lassen.
Bekannte Beispiele hierfür liefern der Wohlordnungssatz oder das \textsc{Banach}-\textsc{Tarski}-Paradoxon.
Trotzdem bildet \AC einen wichtigen Bestandteil mathematischer Theorien, wie auch der Theorie von topologischen Räumen.
In dieser Arbeit stehen basierend auf \cite{herrlich2006axiom} speziell topologische Funktionenräume im Vordergrund unter dem Aspekt der Gültigkeit verschiedener Versionen von \textsc{Áscoli}-Sätzen.

Es werden Grundlagen, wie sie in Rahmen einer Vorlesung zu allgemeiner Topologie vermittelt werden, vorausgesetzt.
Sämtliche unbewiesene Aussagen dieses Typus lassen sich jedoch in der vorhandenen Lehrbuchliteratur auffinden.
Für diesen Text wurden hierzu insbesondere Bartsch,Preuß,Kelley,Ebbinghaus herangezogen.

Im Einführungskapitel dieser Arbeit werden grundlegende Aussagen der Topologie und Mengenlehre bewiesen, die es ermöglichen den Charakter ihrer Beziehung zu \AC zu beschreiben. 
Es werden unterschiedliche, im Rahnmen von \ZF zunächst voneinander unabhängige Kompaktheitsbegriffe auf topologischen Räumen eingeführt und anschließend in Relation zueinander gesetzt.
Als zentrales Aussage dieses Kapitels steht ein Satz, welcher eine Abschwächung von \AC, namlich den \textsc{Boole}schen Primidealsatz \PIT, mit Kompaktheitsaussagen über Produkttopologien verbindet.

Das zweite Kapitel behandelt die Topologie von Funktionenräumen im Hinblick auf \textsc{Áscoli}-Sätze.
Es werden unterschiedliche Formulierungen dieser Kompaktheitsaussagen vorgestellt und deren Äquivalenz zu Axiomen der Mengenlehre bewiesen.
Insbesondere wird dargestellt inwieweit sich \textsc{Áscoli}-Sätze im Rahmen von \ZF beweisen lassen, wenn man alternative Kompaktheitsbegriffe zugrunde legt.

Abschließend fasst ein Fazit die wichtigsten Ergebnisse zusammen.
