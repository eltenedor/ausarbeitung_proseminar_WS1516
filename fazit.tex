\chapter{Fazit}

In dieser Proseminarausarbeitung wurden verschiedene Versionen des \textsc{Ascoli}\hyp{}Satzes in Hinblick auf für deren Gültigkeit hinreichende und notwendige Erweiterungen von \ZF vorgestellt.
Es wurde aufgezeigt, dass die topologische und die klassische Version des \textsc{Ascoli}\hyp{}Satzes nicht im Rahmen von \ZF beweisbar ist, auch nicht, wenn man auf im Rahmen von \ZF verschiedene Kompaktheitsbegriffe ausweicht.
Stattdessen muss das Axiomensystem um \PIT oder \CCR erweitert werden, falls eine modifizierte und dadurch weiter eingeschränkte Variante des klassischen \textsc{Ascoli}\hyp{}Satzes nicht ausreicht.

Die im Rahmen dieser Ausarbeitung vorgestellten Beweise demonstrierten des Weiteren eine der Tücken von \AC und seinen Abschwächungen. Sie tauchen in den länglichen Beweisen lediglich an einer einzigen Stelle auf, was dazu führen kann, dass man fälschlicherweise annimmt, die bewiesene Aussage gelte bereits in \ZF. Dazu kommt dass \AC und Abschwächungen selten direkt, sondern in Form einer äquivalenten Aussage, hier unter anderem in Form von Abschwächungen des Kompaktheitssatzes von \textsc{Tychonoff}, zum Einsatz kamen.
